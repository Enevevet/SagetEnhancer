\documentclass{article}
\usepackage{geometry}
\geometry{a4paper,margin=2cm}

\usepackage{wrapfig}

\everymath{\displaystyle}
\renewcommand{\epsilon}{\varepsilon}
%\renewcommand{\epsilon}{\mathchar"122}

\usepackage{esvect}
\usepackage{wrapfig}
\usepackage{mathrsfs}

\usepackage{physics}

% (?<!\\Bigg)\)\_
% <(.*?)>
% \_\{(?=em)(.*?)\}
% \\fbox{\$\left((.|\n\right)*?)\$}
% : ?\\\\ ?\n\$\left((.(?!(\\mathcolorbox\right))|\n)*?)\$ ?\.? ?\n?\\\\

\usepackage{xcolor}
\usepackage{soul}
\newcommand{\mathcolorbox}[2]{\fcolorbox{black}{#1}{$#2$}}

\DeclareSymbolFont{legacyletters}{OML}{cmm}{m}{it}
\let\j\relax
\DeclareMathSymbol{\j}{\mathord}{legacyletters}{"7C}


\newcommand{\oneast}{\bigskip\par{\large\centerline{*\medskip}}\par}


\usepackage[T1]{fontenc}
%% \usepackage[french]{babel}
\usepackage{epsfig}
\usepackage{graphicx}
\usepackage{amsmath}
%\setlength{\mathindent}{0cm}
\usepackage{amsfonts}
\usepackage{amssymb}
\usepackage{float}
\usepackage{esint}
\usepackage{enumitem}
\usepackage{frcursive}
%\usepackage{fourier}
%% \usepackage{amsrefs}
\reversemarginpar
\newcommand{\asterism}{%
\leavevmode\marginpar{\makebox[10em][c]{$10^{500}$\makebox[1em][c]{%
\makebox[0pt][c]{\raisebox{-0.3ex}{\smash{$\star\star$}}}%
\makebox[0pt][c]{\raisebox{0.8ex}{\smash{$\star$}}}%
}}}}

\setlength\parindent{0pt}

\let\oldiint\iint
\renewcommand{\iint}{\oldiint\limits}

\let\oldiiint\iiint
\renewcommand{\iiint}{\oldiiint\limits}

\let\oldoint\oint
\renewcommand{\oint}{\oldoint\limits}

\let\oldoiint\oiint
\renewcommand{\oiint}{\oldoiint\limits}


\renewcommand\overrightarrow{\vv}
\let\oldexp\exp
\renewcommand{\exp}[1]{\oldexp\left(#1\right)}

\newcommand{\ext}{\text{ext}}
\newcommand{\cste}{\text{cste}}

%\renewcommand{\div}{\mathrm{div}}
\let\div\relax
\DeclareMathOperator{\div}{\mathrm{div}}
\let\rot\relax
\DeclareMathOperator{\rot}{\overrightarrow{\mathrm{rot}}}
\let\grad\relax
\DeclareMathOperator{\grad}{\overrightarrow{\mathrm{grad}}}

\title{\huge{\textbf{Modèle scalaire de la lumière (O1)}}}
\author{par Guillaume Saget, professeur de Sciences Physiques en MP, Lycée Champollion}
\date{}

\begin{document}
\maketitle


\begin{abstract}
Je propose ici le cours O1 qui vient apporter les réponses au
support de cours éponyme.
\end{abstract}


\section*{III. Le modèle du train d'onde(s) non amorti}
\subsection*{3) Définitions}
\subsection*{d) Relation entre largeur spectrale $\Delta \lambda_{0}$ et longueur de cohérence $\ell_{c}$}
Dans le vide, $\nu = \frac{c}{\lambda_{0}}$ ; on
différentie
cette relation : $\mathrm{d}\nu = -c\frac{\mathrm{d}\lambda_{0}}{\lambda_{0}^{2}}$. \\
On passe enfin en variations finies : $\mathrm{d}\nu \rightarrow \Delta \nu$
; $-c\frac{\mathrm{d}\lambda_{0}}{\lambda_{0}^{2}} \rightarrow c\frac{\Delta
\lambda_{0}}{\lambda_{0}^{2}}$. \\
Il s'ensuit que : $\Delta \nu = c\frac{\Delta
\lambda_{0}}{\lambda_{0}^{2}}$. \\
On exploite enfin la relation caractéristique des paquets
d'onde(s) : $\Delta t \Delta \nu \approx 1$. \\
On en déduit immédiatement que : \mathcolorbox{gray!20}{\Delta t \approx
\frac{\lambda_{0}^{2}}{c\Delta \lambda_{0}}} et la longueur de
cohérence par : \mathcolorbox{gray!20}{\ell_{c} = c\Delta t =
\frac{\lambda_{0}^{2}}{\Delta \lambda_{0}}}. \\
\underline{Ordre de grandeur de $\Delta \lambda_{0}$ pour une source
spectrale} ($\Delta t \approx 10^{-11}$ s) : pour $\lambda_{0} \sim
6\times 10^{-7}$ m, $\Delta \lambda_{0} \sim \frac{\left(6\times
10^{-7}\right)^{2}}{3\times 10^{8}\times 10^{-11}} \sim \frac{4\times
10^{-13}}{3\times 10^{-3}} \sim 10^{-10}$ m.

\section*{IV. Notion de rayon lumineux}
\subsection*{3) Rayon lumineux}
Un rayon lumineux $\left(\Gamma\right)$ issu de la source ponctuelle $S$ et
atteignant le point $M$ est représenté sur la figure \ref{Fig.2}.


\begin{wrapfigure}{r}{0.25\textwidth}
\epsfig{file=Fig2.PNG,height=1.7cm, width=6.5cm}
\caption\protect{Représentation d'un rayon lumineux $\left(\Gamma\right)$ dans un
milieu transparent (inhomogène).}\label{Fig.2}
\end{wrapfigure}



\section*{V. Modèle scalaire}
\subsection*{2) Représentation mathématique d'un champ scalaire décrivant une onde lumineuse}
Le champ vibratoire issu de la source ponctuelle $S$ atteint à une
date $t$ un quelconque point $M$ de l'espace ; en ce point, le champ
vibratoire est : \\
\centerline{\mathcolorbox{gray!20}{a(M,t) = A(M)\cos\left(\omega t -\varphi_{M}\right)}}
avec
$\varphi_{M}$ la phase instantanée à l'origine en $M$. \\
Cette expression est licite pour les sources ponctuelles
monochromatiques ou quasi-monochromatiques.

\section*{VI. Chemin optique}
\subsection*{1) Définition}
Le chemin optique entre la source ponctuelle $S$ et un quelconque
point $M$ de l'espace est défini par : \\
\centerline{\mathcolorbox{gray!20}{L_{\mathrm{SM}} = (SM) = \int_{P \in \left(\Gamma\right)}n(P)\overline{\mathrm{d}\ell}(P)
= \int_{S}^{M}n(P)\overline{\mathrm{d}\ell}(P)}} avec $n(P)$ l'indice optique
du matériau en $P$ et $\overline{\mathrm{d}\ell}$ un petit déplacement
élémentaire issu de $P$ le long du rayon lumineux $\left(\Gamma\right)$.


\begin{wrapfigure}{r}{0.25\textwidth}
\epsfig{file=Fig1.PNG,height=1.7cm, width=6.5cm}
\caption\protect{Représentation d'un rayon lumineux $\left(\Gamma\right)$ dans un
milieu transparent (inhomogène). Sur la figure,
$\overline{\mathrm{d}\ell}(P)=\overline{PP'}>0$.}\label{Fig.1}
\end{wrapfigure}


Dans un milieu homogène, $\forall P$, $n(P)=n$, d'où : \mathcolorbox{gray!20}{(SM)
= nSM}. \\
En outre, dans un milieu homogène, la propagation de la lumière se
faisant de façon rectiligne, le chemi optique $(SM) = nSM$ est donc
le produit de l'indice du milieu par le segment $\left[SM]$.
\\
\underline{Théorèmes fondamentaux:}
\begin{itemize}
\item Démontrons le théorème n$^{\circ}1$ (théorème de Chasles sur le
chemin) optique. \\
On écrit : $(SM) = \int_{P \in \left(\Gamma\right)}n(P)\overline{\mathrm{d}\ell}(P)$.
Considérons un point $P' \in \left(\Gamma\right)$ atteint par la lumière entre
$S$ et $M$ ; on écrit dès lors : $(SM) =\int_{S}^{M}
n(P)\overline{\mathrm{d}\ell}(P) = \int_{S}^{P'}
n(P)\overline{\mathrm{d}\ell}(P)+\int_{P'}^{M} n(P)\overline{\mathrm{d}\ell}(P) =
(SP')+(P'M)$.
\item \underline{Conséquence du théorème n$^{\circ}2$ :} \\
On écrit $(SM) = \int_{S}^{M} n(P)\overline{\mathrm{d}\ell}(P) = \int_{S}^{M}
n(P)\overline{PP'}$ avec $P'$
point voisin appartenant au rayon lumineux. \\
Or, $(SM) = \int_{S}^{M} n(P)\overline{\mathrm{d}\ell}(P) = -\int_{M}^{S}
n(P)\overline{PP'}$. \\
En vertu du principe inverse de la lumière, $M$ devient le point
source et $S$ un point de l'espace atteint par la lumière depuis $M$
en suivant le rayon lumineux $\left(\Gamma\right)$. Changeons le sens positif
retenu pour la propagation de la lumière ; le sens positif retenu
maintenant est celui allant de $M$ à $S$. \\
On écrit ($\overline{PP'} = -\overline{P'P}$) : $(SM) = \int_{S}^{M}
n(P)\overline{\mathrm{d}\ell}(P) = -\int_{M}^{S} n(P)\overline{PP'} =
\int_{M}^{S}
n(P)\overline{P'P}$. \\
Posons $\overline{P'P} = \overline{\mathrm{d}\ell}(P') >0$. Comme $P'$ est
voisin de $P$, on a $n(P) \approx n(P')$. Il vient :
\\
$(SM) = \int_{S}^{M} n(P)\overline{\mathrm{d}\ell}(P) = \int_{M}^{S}
n(P)\overline{P'P}$ \\ $= \int_{M}^{S}
n(P)\overline{\mathrm{d}\ell}(P')\approx \int_{M}^{S}
n(P')\overline{\mathrm{d}\ell}(P')=(MS)$.
\end{itemize}

\subsection*{2)  Application au champ scalaire décrivant l'onde
lumineuse}
\begin{itemize}
\item \underline{Cas général :} La phase de l'onde en $M$ à la date
$t$ est celle de la source ponctuelle $S$ à la date $t -$ \textit{la
durée mis par l'onde pour aller de} $S$ à $M$ ; notons $\tau$ cette
durée. On écrit : \\
$\Phi(M,t) = \Phi\left(S,t-\tau\right)$. Or, $\Phi(M,t) = \omega t
-\varphi_{M}$ et $\Phi(S,t) = \omega t -\varphi_{S}$. On en déduit
que : $\Phi(M,t) = \omega t -\varphi_{M} = \Phi\left(S,t-\tau\right) = \omega t
-\omega \tau -\varphi_{S}$. On en déduit que :  \mathcolorbox{gray!20}{\varphi_{M} =
\varphi_{S}+\omega \tau}. \\
Le champ vibratoire (la source $S$ est monochromatique ou
quasi-monochromatique) en $M$ à $t$ est : $a(M,t) = A(M)\cos\left(\omega
t -\varphi_{M}\right) = A(M)\cos\left(\omega t -
\omega \tau -\varphi_{S}\right)$. \\
Or, on montrera en exercice que : \mathcolorbox{gray!20}{\tau = \frac{(SM)}{c}}. \\
Finalement, $a(M,t) = A(M)\cos\left(\omega t -
\frac{\omega}{c}(SM\right)-\varphi_{S})$. En posant $k_{0} =
\frac{\omega}{c}$ le nombre d'onde dans le vide, on a :
\mathcolorbox{gray!20}{a(M,t) = A(M)\cos\left(\omega t -
\frac{\omega}{c}(SM\right)-\varphi_{S})}. \\
Introduisons encore la longueur d'onde dans le vide (ou période
spatiale du champ vibratoire dans le vide) définie à partir de
$k_{0}$ par : $k_{0} = \frac{\omega}{c} = \frac{2\pi}{\lambda_{0}}$.
\item Dans un milieu transparent homogène, $(SM) = nSM$. Il vient :
\mathcolorbox{gray!20}{a(M,t) = A(M)\cos\left(\omega t - nk_{0}SM-\varphi_{S}\right)}. \\
Posons $k = k_{0}n = \frac{\omega}{c}n$ le nombre d'onde dans le
milieu. Il vient : \mathcolorbox{gray!20}{a(M,t) = A(M)\cos\left(\omega t -
kSM-\varphi_{S}\right)}. \\
Introduisons enfin $\lambda$ la longueur d'onde dans le milieu
définie à partir de $k$ par : $k = \frac{2\pi}{\lambda}$. On en
déduit que : $\lambda = \frac{2\pi}{k} =
\frac{2\pi}{nk_{0}}=\frac{\lambda_{0}}{n}$.
\end{itemize}

\section*{VII. Surface d'onde et théorème de Malus-Dupin}
\subsection*{2) Théorème de Malus-Dupin (admis)}
Exemple d'application du théorème de Malus-Dupin (en lien avec
l'exercice n$^{\circ}$8) du TD O1. \\
En veru du principe du retour inverse de la lumière, $M_{\mathrm{\infty}}$
est un point source. Loin de ce point, les surfaces d'ondes sont des
plans. En vertu du théorème de Malus-Dupin, les rayons lumineux sont
orthogonaux aux plans d'ondes : $\left(\Sigma\right)$ est un plan d'onde issu
de $M$ : $\left(M_{\mathrm{\infty}}H\right) = \left(M_{\mathrm{\infty}}K\right)$ mais $(SIH) \neq (SIJK)$ :
$\left(\Sigma\right)$ bien qu'orthogonal aux rayons émergents de la lame de
verre n'est pas une surface d'ondes issue de $S$. En effet, le rayon
noir émergeant de la lame de verre subit une réflexion tandis que le
rayon en rouge sortant de la lame de verre a subi deux réfractions
et une réflexion : les points $H$ et $K$ $\in \left(\Sigma\right)$
n'appartiennent pas à la même surface d'ondes issue de $S$.

\section*{VIII. Intensité lumineuse ou éclairement}
\subsection*{2)  Définition de l'éclairement ou de l'intensité
lumineuse} Décomposons l'intervalle $\tau_{\mathrm{d}}$ en un nombre
quasi-entier $N$ d'intervalles de durée identique $T$ (période
temporelle des signaux ; $T\sim 10^{-15}$ s, $\tau_{\mathrm{d}} \sim 10^{-6}$
s pour une photodiode, donc $N \sim 10^{9}$) : $\tau_{\mathrm{d}} \approx
NT$. Il vient :
\begin{eqnarray}\label{Eq.1}
I(M) &=& 2A^{2}(M)<\cos^{2}\left(\omega t
-k_{0}(SM\right)-\varphi_{S})>_{\mathrm{\tau_{\mathrm{d}}}} \notag \\
&=& 2A^{2}(M)\times
\frac{1}{\tau_{\mathrm{d}}}\int_{0}^{\tau_{\mathrm{d}}}\cos^{2}\left(\omega t
-k_{0}(SM\right)-\varphi_{S})\mathrm{d}t \notag \\
\end{eqnarray}
Il faut distinguer le cas des sources moncohromatiques et
quasi-monochromatiques.
\begin{itemize}
\item Considérons tout d'abord le cas des sources monochromatiques. La
phase $\varphi_{S} = \mathrm{cste}$ pour tout $t$. L'équation (\ref{Eq.1})
devient :
\begin{eqnarray*}
I(M) \approx \frac{2A^{2}(M)}{NT}\times
\sum_{k=1}^{N}\int_{(k-1)T}^{kT}\cos^{2}(\omega t
-k_{0}(SM)-\varphi_{S})dt.
\end{eqnarray*}
Or, du fait de la constante de $\varphi_{S}$, $\forall k \in
\left[1,\,N\right]$, $\int_{(k-1)T}^{kT}\cos^{2}\left(\omega t
-k_{0}(SM\right)-\varphi_{S})\mathrm{d}t = \int_{0}^{T}\cos^{2}\left(\omega t
-k_{0}(SM\right)-\varphi_{S})\mathrm{d}t$. Il vient que :
\begin{eqnarray}\label{Eq.2}
I(M) &\approx& \frac{2A^{2}(M)}{NT}\times
\left(N\int_{0}^{T}\cos^{2}(\omega t -k_{0}(SM\right)-\varphi_{S})\mathrm{d}t) \notag \\
&=& \frac{2}{T}\int_{0}^{T}A^{2}(M)\cos^{2}\left(\omega t
-k_{0}(SM\right)-\varphi_{S})\mathrm{d}t \notag \\
&=& <2a^{2}(M,t)>_{T}.
\end{eqnarray}
\item Considérons maintenant le cas des sources quasi-mono-chromatiques. La
phase $\varphi_{S} = \mathrm{cste}$ seulement à l'échelle de $\Delta t$
(temps de cohérence temporelle de la source). Ceci dit, comme $T \ll
\Delta t$, à l'échelle d'une période temporelle, $\varphi_{S}$
demeure quasi-constant. L'équation (\ref{Eq.2}) est donc encore
licite ici et le résultat est le même que dans le cas des sources
monochromatiques.
\end{itemize}
Montrons maintenant que $I(M) = A^{2}(M)$. Pour ce faire, il suffit
de linéariser : \\
$\cos^{2}\left(\omega t - k_{0}(SM\right) - \varphi_{S}) =
\frac{1+\cos\left(2\omega t - 2k_{0}(SM\right) - 2\varphi_{S})}{2}$. \\
Or, $\int_{0}^{T}\cos\left(2\omega t - 2k_{0}(SM\right) - 2\varphi_{S})\mathrm{d}t =
\frac{\left[\sin\left(2\omega t - 2k_{0}(SM\right) -
2\varphi_{S})\right]_{0}^{T}}{2\omega} = 0$. On en déduit que : \\
$I(M)\approx 2<a^{2}(M,t)>_{T} = \frac{2}{T}\times
A^{2}(M)\int_{0}^{T}\frac{\mathrm{d}t}{2} = A^{2}(M)$.

\section*{IX. Onde sphérique et onde plane dans un milieu homogène}
\subsection*{ 1)  Onde sphérique}
Le champ vibratoire en un point $M$ (l'onde émise est issue d'une
source ponctuelle sphérique monochromatique ou
quasi-monochromatique) est : \\
\centerline{\mathcolorbox{gray!20}{a(M,t) = \frac{A_{0}}{r}\cos\left(\omega t -nk_{0}r
-\varphi_{S}\right)}} avec $SM = r$. On peut poser $ k = k_{0}n$
le nombre d'onde dans le vide. $A_{0}$ est une constante et $A(M) = A(r) = \frac{A_{0}}{r}$ est l'amplitude de l'onde. On a alors : \\
\centerline{\mathcolorbox{gray!20}{a(M,t) = \frac{A_{0}}{r}\cos\left(\omega t -kr -\varphi_{S}\right)}}. \\
La démonstration sera faite en TD.

\subsection*{2)  Onde plane}
Loin de la source ponctuelle $S$ ponctuelle  monochromatique ou
quasi-monochromatique, les surfaces d'ondes sont quasiment planes ;
le champ vibratoire de l'onde plane en un point $M$ est : \\
\centerline{\mathcolorbox{gray!20}{a(M,t) = A_{0}\cos\left(\omega t
-n\overrightarrow{k}_{0}.\overrightarrow{OM} - \varphi_{O}\right)}}. \\
$\varphi_{O}$ est la phase isntantanée à l'origine en un point $O$
de l'espace très éloigné de la source $S$ (la démonstration sera
faite en TD). En introduisant le vecteur d'onde dans le milieu
défini par : $\overrightarrow{k} = n\overrightarrow{k}_{0} =
n\frac{\omega}{c}\overrightarrow{u}$ avec $\overrightarrow{u}$ un
vecteur unitaire donnant le sens de propagation des rayons lumineux
(la propagation est rectiligne dans un milieu homogène), le champ
vibratoire est : \\
\centerline{\mathcolorbox{gray!20}{a(M,t) = A_{0}\cos\left(\omega t
-\overrightarrow{k}.\overrightarrow{OM} - \varphi_{O}\right)}}.

\section*{X. Applications du théorème de Malus-Dupin (A SAVOIR)}
\subsection*{1)  Effet d'une lentille mince dans l'approximation de Gauss (stigmatisme
approché)} Soit une source ponctuelle $S_{\mathrm{\infty}}$ à l'infini dans
la direction de l'axe optique. La lentille mince (L) conjugue ce
point au foyer image $F'$ ; on écrit : $S_{\mathrm{\infty}}
$ $\underleftrightarrow{(L)}$ $F'$. \\
Loin de $S_{\mathrm{\infty}}$, les surfaces d'ondes sphériques sont quasiment
planes : en vertu du théorème de Malus-Dupin, les rayons lumineux
sont orthogonaux aux plans d'ondes. \\
Après passage par la lentille, les rayons convergent vers le foyer
image $F'$. \\
En vertu du principe du retour inverse de la lumière,
$F'$ est une source ponctuelle émettant des ondes sphériques. Les
rayons lumineux sont orthogonaux aux surfaces d'ondes (via le
théorème de Malus-Dupin).
\\
Ainsi, la lentille mince dans cette configuration transforme des
plans d'ondes en ondes sphériques et réciproquement (via le principe
du retour inverse de la lumière).

\subsection*{3) Exemples de travail}
\underline{Premier exemple :} Soit une source ponctuelle
$S_{\mathrm{\infty}}$ à l'infini. La lentille mince (L) conjugue ce point au
point  $M$ ; on écrit : $S_{\mathrm{\infty}} $ $\underleftrightarrow{(L)}$
$M$. Dans les conditions de Gauss (on est dans le cadre du
stigmatisme approché), $\left(S_{\mathrm{\infty}}HIM\right) = \left(S_{\mathrm{\infty}}JM\right)$. \\
Décomposons les chemins optiques, $\left(S_{\mathrm{\infty}}HIM\right) =
\left(S_{\mathrm{\infty}}H\right)+(HIM)$ et $\left(S_{\mathrm{\infty}}JM\right) = \left(S_{\mathrm{\infty}}J\right)+(JM)$. \\
Il vient : $\left(S_{\mathrm{\infty}}H\right)+(HIM) = \left(S_{\mathrm{\infty}}J\right)+(JM)$. \\
Or, loin de $S_{\mathrm{\infty}}$, les surfaces d'ondes sont des plans d'onde
issus de $S_{\mathrm{\infty}}$ : $\left(S_{\mathrm{\infty}}H\right) = \left(S_{\mathrm{\infty}}J\right)$. On en
déduit que : \mathcolorbox{gray!20}{(HIM) = (JM)}. \\
En vertu du principe du retour inverse de la lumière, $M$ est une
source ponctuelle avec \mathcolorbox{gray!20}{(MIH) = (MJ)}. \\
\underline{Second exemple :} Par la lentille mince
convergente (LMC), le point $M$ est conjugué à un point $S_{\mathrm{\infty}}$
: $S_{\mathrm{\infty}} $
$\underleftrightarrow{(L)}$ $M$. \\
Dans les conditions de Gauss (on
est dans le cadre du
stigmatisme approché), $\left(S_{\mathrm{\infty}}S_{1}H'HIM\right) = \left(S_{\mathrm{\infty}}S_{2}KJ'JM\right)$. \\
Or, loin de $S_{\mathrm{\infty}}$, les surfaces d'ondes sont des plans d'onde
issus de $S_{\mathrm{\infty}}$ : en particulier $\left(\Sigma''\right)$, ($\Sigma')$ et
$\left(\Sigma\right)$ sont des plans d'onde issues de $S_{\mathrm{\infty}}$. Les rayons
lumineux sont orthogonaux à tous ces plans via le théorème de
Malus-Dupin ($S_{1}$ et $K \in \left(\Sigma''\right)$, $H'$ et $J' \in
\left(\Sigma'\right)$, $H$ et $J \in \left(\Sigma\right)$) :
\begin{eqnarray}undefined
\left(S_{\mathrm{\infty}}S_{1}\right) &=& \left(S_{\mathrm{\infty}}S_{2}K\right), \label{Eq.101} \\
\left(S_{\mathrm{\infty}}S_{1}H'\right) &=& \left(S_{\mathrm{\infty}}S_{2}KJ'\right), \label{Eq.102} \\
\left(S_{\mathrm{\infty}}S_{1}H'H\right) &=& \left(S_{\mathrm{\infty}}S_{2}KJ'J\right). \label{Eq.103}
\end{eqnarray}
En décomposant les chemins optiques $\left(S_{\mathrm{\infty}}S_{1}H'HIM\right) =
\left(S_{\mathrm{\infty}}S_{1}\right)+\left(S_{1}H'HIM\right)$ et $\left(S_{\mathrm{\infty}}S_{2}KJ'JM\right) =
\left(S_{\mathrm{\infty}}K\right)+\left(KJ'JM\right)$, on a  : $\left(S_{\mathrm{\infty}}S_{1}\right)+\left(S_{1}H'HIM\right) =
\left(S_{\mathrm{\infty}}S_{2}K\right)+\left(KJ'JM\right)$. Or, avec l'équation (\ref{Eq.101}), il
vient : $\left(S_{1}H'HIM\right) = \left(KJ'JM\right)$. \\
De même, en décomposant les chemins optiques $\left(S_{\mathrm{\infty}}S_{1}H'HIM\right)
= \left(S_{\mathrm{\infty}}S_{1}H'\right)+\left(H'HIM\right)$ et $\left(S_{\mathrm{\infty}}S_{2}KJ'JM\right) =
\left(S_{\mathrm{\infty}}S_{2}KJ'\right)+(J'JM)$, on a  : $\left(S_{\mathrm{\infty}}S_{1}H'\right)+\left(H'HIM\right) =
\left(S_{\mathrm{\infty}}S_{2}KJ'\right)+(J'JM)$. Or, avec l'équation (\ref{Eq.102}), il
vient : $\left(H'HIM\right) = (J'JM)$. \\
Enfin, en décomposant les chemins optiques $\left(S_{\mathrm{\infty}}S_{1}H'HIM\right) =
\left(S_{\mathrm{\infty}}S_{1}H'H\right)+(HIM)$ et $\left(S_{\mathrm{\infty}}S_{2}KJ'JM\right) =
\left(S_{\mathrm{\infty}}KJ'J\right)+(JM)$, on a  : $\left(S_{\mathrm{\infty}}S_{1}H'H\right)+(HIM) =
\left(S_{\mathrm{\infty}}S_{2}KJ'J\right)+(JM)$. Or, avec l'équation (\ref{Eq.103}), il
vient : $(HIM) = (JM)$.
\\
En résumé, on vient de montrer que :
\begin{eqnarray}undefined
(HIM) &=& (JM), \label{Eq.107} \\
\left(H'HIM\right) &=& (J'JM), \label{Eq.108} \\
\left(S_{1}H'HIM\right) &=& \left(KJ'JM\right). \label{Eq.109}
\end{eqnarray}
En vertu du principe du retour inverse de la lumière, $M$ est un
point source : $M$ $\underleftrightarrow{(L)}$ $S_{\mathrm{\infty}}$. On a :
\begin{eqnarray}undefined
(MIH) &=& (MJ), \label{Eq.104} \\
\left(MIHH'\right) &=& (MJJ'), \label{Eq.105} \\
\left(MIHH'S_{1}\right) &=& \left(MJJ'K\right). \label{Eq.106}
\end{eqnarray}
En fait, après passage par la lentille, les rayons issus de $M$
ayant tous subi deux réfractions dans $(L)$, eu égard au théorème de
Malus -Dupin, $\left(\Sigma''\right)$, $\left(\Sigma'\right)$ et $\left(\Sigma\right)$ sont aussi des
surfaces d'ondes issues de $M$.
\\
La source $S$ est ponctuelle et équidistante des deux trous $S_{1}$
et $S_{2}$. Le milieu est homogène : $nSS_{1} = nSS_{2}$ ou encore :
\begin{eqnarray}\label{Eq.100}
\left(SS_{1}\right) = \left(SS_{2}\right)
\end{eqnarray}
$S_{1}$ et $S_{2}$ appartiennent à la même surface d'onde (qui est
sphérique car $S$ est une source ponctuelle qui émet des ondes
sphériques dans un milieu homogène) issue de $S$. \\
Calculons la différence de chemin optique (cette grandeur trouvera
un intérêt dans le chapitre O2 dédié aux interférences lumineuses)
$\delta = \left(SS_{2}KJ'JM\right) - \left(SS_{1}H'HIM\right)$. \\
On écrit : $\left(SS_{2}KJ'JM\right) = \left(SS_{2}\right)+\left(S_{2}K\right)+\left(KJ'JM\right)$ et
$\left(SS_{1}H'HIM\right) = \left(SS_{1}\right)+\left(S_{1}H'HIM\right)$. \\
Avec les équations (\ref{Eq.109}) et (\ref{Eq.100}), on en déduit
que : \\
\centerline{\mathcolorbox{gray!20}{\delta = \left(S_{2}K\right)=nS_{2}K}}.
\\
\underline{Variante possible :} \\
$\left(SS_{2}KJ'JM\right) = \left(SS_{2}\right)+\left(S_{2}K\right)+(KJ')+ (J'JM)$ et
$\left(SS_{1}H'HIM\right) = \left(SS_{1}\right)+\left(S_{1}H'\right)+\left(H'HIM\right)$. \\
Avec les équations (\ref{Eq.108}) et (\ref{Eq.100}), on en déduit
que : \\
$\delta = \left(S_{2}K\right)+(KJ')-\left(S_{1}H'\right)$. Or, $S_{1}$ et $K \in
\left(\Sigma''\right)$, $H'$ et $J'\in \left(\Sigma'\right)$, donc $\left(S_{1}H'\right) = (KJ')$.
\\
Finalement, après implification, on retrouve que : \\
\centerline{\mathcolorbox{gray!20}{\delta =
\left(S_{2}K\right)=nS_{2}K}}.
\\
\underline{Autre variante possible :} \\
$\left(SS_{2}KJ'JM\right) = \left(SS_{2}\right)+\left(S_{2}K\right)+(KJ'J)+ (JM)$ et \\
$\left(SS_{1}H'HIM\right) = \left(SS_{1}\right)+\left(S_{1}H'H\right)+(HIM)$. \\
Avec les équations (\ref{Eq.107}) et (\ref{Eq.100}), on en déduit
que : \\
$\delta = \left(S_{2}K\right)+(KJ'J)-\left(S_{1}H'H\right)$. Or, $S_{1}$ et $K \in
\left(\Sigma''\right)$, $H$ et $J\in \left(\Sigma\right)$, donc $\left(S_{1}H'H\right) = (KJ'J)$.
\\
Finalement, après implification, on retrouve que : \\
\centerline{\mathcolorbox{gray!20}{\delta = \left(S_{2}K\right)=nS_{2}K}}.
\end{document}
