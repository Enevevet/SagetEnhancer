\documentclass{article}
\usepackage{geometry}
\geometry{a4paper,margin=2cm}

\usepackage{wrapfig}

\everymath{\displaystyle}
\renewcommand{\epsilon}{\varepsilon}
%\renewcommand{\epsilon}{\mathchar"122}

\usepackage{esvect}
\usepackage{wrapfig}
\usepackage{mathrsfs}

\usepackage{physics}

% (?<!\\Bigg)\)\_
% <(.*?)>
% \_\{(?=em)(.*?)\}
% \\fbox{\$\left((.|\n\right)*?)\$}
% : ?\\\\ ?\n\$\left((.(?!(\\mathcolorbox\right))|\n)*?)\$ ?\.? ?\n?\\\\

\usepackage{xcolor}
\usepackage{soul}
\newcommand{\mathcolorbox}[2]{\fcolorbox{black}{#1}{$#2$}}

\DeclareSymbolFont{legacyletters}{OML}{cmm}{m}{it}
\let\j\relax
\DeclareMathSymbol{\j}{\mathord}{legacyletters}{"7C}


\newcommand{\oneast}{\bigskip\par{\large\centerline{*\medskip}}\par}


\usepackage[T1]{fontenc}
%% \usepackage[french]{babel}
\usepackage{epsfig}
\usepackage{graphicx}
\usepackage{amsmath}
%\setlength{\mathindent}{0cm}
\usepackage{amsfonts}
\usepackage{amssymb}
\usepackage{float}
\usepackage{esint}
\usepackage{enumitem}
\usepackage{frcursive}
%\usepackage{fourier}
%% \usepackage{amsrefs}
\reversemarginpar
\newcommand{\asterism}{%
\leavevmode\marginpar{\makebox[10em][c]{$10^{500}$\makebox[1em][c]{%
\makebox[0pt][c]{\raisebox{-0.3ex}{\smash{$\star\star$}}}%
\makebox[0pt][c]{\raisebox{0.8ex}{\smash{$\star$}}}%
}}}}

\setlength\parindent{0pt}

\let\oldiint\iint
\renewcommand{\iint}{\oldiint\limits}

\let\oldiiint\iiint
\renewcommand{\iiint}{\oldiiint\limits}

\let\oldoint\oint
\renewcommand{\oint}{\oldoint\limits}

\let\oldoiint\oiint
\renewcommand{\oiint}{\oldoiint\limits}


\renewcommand\overrightarrow{\vv}
\let\oldexp\exp
\renewcommand{\exp}[1]{\oldexp\left(#1\right)}

\newcommand{\ext}{\text{ext}}
\newcommand{\cste}{\text{cste}}

%\renewcommand{\div}{\mathrm{div}}
\let\div\relax
\DeclareMathOperator{\div}{\mathrm{div}}
\let\rot\relax
\DeclareMathOperator{\rot}{\overrightarrow{\mathrm{rot}}}
\let\grad\relax
\DeclareMathOperator{\grad}{\overrightarrow{\mathrm{grad}}}

\title{\huge{\textbf{Interférences lumineuses en lumière non polarisée - Division du front d'ondes (O2)}}}
\author{par Guillaume Saget, professeur de Sciences Physiques en MP, Lycée Champollion}
\date{}

\begin{document}
\maketitle


\begin{abstract}
Je propose ici le cours O2 qui vient apporter les réponses au
support de cours éponyme.
\end{abstract}


\section*{I. Interférences entre deux ondes lumineuses}
\subsection*{6) Superposition de deux ondes émises par deux sources ponctuelles cohérentes : formule d'Augustin Fresnel}
Je propose ici deux démonstrations.
\begin{itemize}
\item \underline{Première démonstration:} \\
Soient $a_{1}(M,t') = A_{1}(M)\cos\left(\omega
t'-k_{0}(S_{1}M\right)-\varphi_{\mathrm{s1}})$ et $a_{2}(M,t') =
A_{2}(M)\cos\left(\omega t'-k_{0}(S_{2}M\right)-\varphi_{\mathrm{s2}})$  les champs
vibratoires des deux sources en M. \\
On ne restreint en rien la généralité du problème à choisir une
nouvelle origine des dates telles que : $\omega t = \omega
t'-\varphi_{\mathrm{s1}}$. Il vient : $a_{1}(M,t)=A_{1}\cos\left(\omega t
-k_{0}(S_{1}M\right))$ et $a_{2}(M,t)=A_{2}\cos\left(\omega t
-k_{0}(S_{2}M\right)-\varphi_{0})$ avec $\varphi_{0} =
\varphi_{\mathrm{s2}}-\varphi_{\mathrm{s1}}$ le déphasage à l'origine entre les deux
sources, constant puisque les deux ondes sont cohérentes.
\\
Le champ vibratoire en M à la date $t'$ \textbf{des deux ondes
cohérentes est la somme des champs
vibratoires}, soit : \mathcolorbox{gray!20}{a(M,t)= a_{1}(M,t)+a_{2}(M,t)}. \\
Par définition de l'intensité lumineuse, en notant $\tau_{\mathrm{d}}$ le
temps de réponse du photorécepteur (typiquement $\tau_{\mathrm{d}} \sim
10^{-6}$ s pour une photodiode) $I_{1}(M)=2<a_{1}(M,t)>_{\mathrm{\tau_{\mathrm{d}}}}$.
\\
On a : $I_{1}(M) = 2A_{1}^{2}(M)<\cos^{2}\left(\omega t
-k_{0}(S_{1}M\right))>_{\mathrm{\tau_{\mathrm{d}}}}$. Décomposons l'intervalle $\tau_{\mathrm{d}}$ en
un nombre quasi-entier $N$ d'intervalles de durée identique $T$
(période temporelle des signaux ; $T\sim 10^{-15}$ s, $\tau_{\mathrm{d}} \sim
10^{-6}$ s pour une photodiode, donc $N \sim 10^{9}$) : $\tau_{\mathrm{d}}
\approx NT$. Il vient :
\begin{eqnarray}\label{Eq.1}
I_{1}(M) &=& 2A_{1}^{2}(M)<\cos^{2}\left(\omega t
-k_{0}(S_{1}M\right))>_{\mathrm{\tau_{\mathrm{d}}}} \notag \\
&=& 2A_{1}^{2}(M)\times
\frac{1}{\tau_{\mathrm{d}}}\int_{0}^{\tau_{\mathrm{d}}}\cos^{2}\left(\omega t
-k_{0}(S_{1}M\right))\mathrm{d}t \notag \\
&\approx& \frac{2A_{1}^{2}(M)}{NT}\times
\sum_{k=1}^{N}\int_{(k-1)T}^{kT}\cos^{2}\left(\omega t -k_{0}(S_{1}M\right))\mathrm{d}t
\notag
\\
\end{eqnarray}
Or, $\forall k \in \left[1,\,N\right]$, $\int_{(k-1)T}^{kT}\cos^{2}\left(\omega t
-k_{0}(S_{1}M\right))\mathrm{d}t = \int_{0}^{T}\cos^{2}\left(\omega t -k_{0}(S_{1}M\right))\mathrm{d}t
= \frac{T}{2}$ (il suffit de linéariser le
$\cos^{2}$ pour arriver à ce résultat). \\
Il s'ensuit que $<\cos^{2}\left(\omega t -k_{0}(S_{1}M\right)>_{\mathrm{\tau_{\mathrm{d}}}} =
\frac{1}{NT}\times N \times  \int_{0}^{T}\cos^{2}\left(\omega t
-k_{0}(S_{1}M\right))\mathrm{d}t = \frac{1}{T}\int_{0}^{T}\cos^{2}\left(\omega t
-k_{0}(S_{1}M\right))\mathrm{d}t =$\\ $<\cos^{2}\left(\omega t -k_{0}(S_{1}M\right)>_{T} =
\frac{1}{2}$, puis : \mathcolorbox{gray!20}{I_{1}(M) = A_{1}^{2}(M)}. \\
De même (le déphasage $\varphi_{0}$ est constant à l'échelle de
$\tau_{\mathrm{d}}$), donc le calcul est
similaire), \mathcolorbox{gray!20}{I_{2}(M) = A_{2}^{2}(M)}. \\
L'intensité du champ en M est : $I(M) = 2<a^{2}(M,t)>_{\mathrm{\tau_{\mathrm{d}}}}
\approx 2<a^{2}(M,t)>_{T} = 2<a_{1}^{2}(M,t)>_{T}+
2<a_{2}^{2}(M,t)>_{T}+ 4<a_{1}(M,t)a_{2}(M,t)>_{T}$. \\
Or ($\cos(a)\cos(b) = \frac{1}{2}\left[\cos(a+b)+\cos(a-b)\right]$): \\
$<a_{1}(M,t)a_{2}(M,t)>_{T} =$ \\
$\frac{1}{T}\times A_{1}(M)A_{2}(M)\int_{0}^{T}\cos\left(\omega t -
k_{0}(S_{1}M\right))\cos\left(\omega t - k_{0}(S_{2}M\right)-\varphi_{0})\mathrm{d}t = \frac{1}{2T}\int_{0}^{T}\cos\left(2\omega t
-k_{0}\left[(S_{1}M\right)+\left(S_{2}M\right)\right]-\varphi_{0})\mathrm{d}t$ $+$
$\frac{1}{2T}\int_{0}^{T}\cos\left(k_{0}\left[(S_{2}M\right)-\left(S_{1}M\right)\right]-\varphi_{0})\mathrm{d}t$.
\\
Or, $\frac{1}{T}\int_{0}^{T}\cos\left(2\omega t
-k_{0}\left[(S_{1}M\right)+\left(S_{2}M\right)\right]-\varphi_{0})\mathrm{d}t = <\cos\left(2\omega t
-k_{0}\left[(S_{1}M\right)+\left(S_{2}M\right)\right]-\varphi_{0})>_{T} = 0$ et
$\frac{1}{T}\int_{0}^{T}\cos\left(k_{0}\left[(S_{2}M\right)-\left(S_{1}M\right)\right]+\varphi_{0})\mathrm{d}t$
$=$ $\cos\left(k_{0}\left[(S_{2}M\right)-\left(S_{1}M\right)\right]+\varphi_{0})$ (ce dernier terme est indépendant du temps). \\
En posant $\delta =\left(S_{2}M\right)-\left(S_{1}M\right)$ la différence de marche
géométrique entre les deux sources en $M$, on en déduit que :
\\
$I(M) = I_{1}(M)+I_{2}(M)+2A_{1}(M)A_{2}(M)\cos\left(k_{0}\delta
+\varphi_{0}\right)$. \\
Enfin, on sait que : $A_{1}(M)= \sqrt{I_{1}(M)}$ et $A_{2}(M)=
\sqrt{I_{2}(M)}$, d'où, en posant $\Delta \varphi = k_{0}\delta
+\varphi_{0}$ le déphasage entre les deux ondes en $M$, on obtient
la mirifique formule de Fresnel associée aux deux sources
ponctuelles, cohérentes interférant au point $M$ : \\
\centerline{\mathcolorbox{gray!20}{I(M)=I_{1}(M)+I_{2}(M)+2\sqrt{I_{1}(M)I_{2}(M)}\cos\left(\Delta
\varphi\right)}}. \\
Le terme d'intensité corrélant les deux sources est : $I_{\mathrm{12}} =
2\sqrt{I_{1}(M)I_{2}(M)}\cos\left(\Delta \varphi\right)$.
\item \underline{Seconde démonstration:} \\
En notation complexe, les champs vibratoires sont :
$\underline{a}_{1}(M,t) = A_{1}e^{i\varphi_{1}}$ avec $\varphi_{1} =
\omega t -k_{0}\left(S_{1}M\right)$ et $\underline{a}_{2}(M,t) =
A_{2}e^{i\varphi_{2}}$ avec $\varphi_{2} = \omega t
-k_{0}\left(S_{2}M\right)-\varphi_{0}$. \\
Or, $I_{1}(M) = A_{1}^{2}(M) =
\underline{a}_{1}(M,t)\underline{a}_{1}^{*}(M,t)=|\underline{a}_{1}|^{2}(M,t)$
; de même, $I_{2}(M) = A_{2}^{2}(M) =
\underline{a}_{2}(M,t)\underline{a}_{2}^{*}(M,t)=|\underline{a}_{2}|^{2}(M,t)$.
\\
L'intensité $I(M)$ se calcule donc comme suit (Cf. cours O1) : $I(M)
= \underline{a}(M,t)\underline{a}^{*}(M,t) =
\left[\underline{a}_{1}(M,t)+\underline{a}_{2}(M,t)\right]\left[\underline{a}_{1}^{*}(M,t)+\underline{a}_{2}^{*}(M,t)\right]
= I_{1}(M)+I_{2}(M)+
\underline{a}_{1}(M,t)\underline{a}_{2}^{*}(M,t) +
\underline{a}_{1}^{*}(M,t)\underline{a}_{2}(M,t)$. \\
Or, $\underline{z}+\underline{z}^{*} = 2
\mathcal{R}_{e}\left(\underline{z}\right)$ avec ici $ \underline{z} =
\underline{a}_{1}(M,t)\underline{a}_{2}^{*}(M,t) =
A_{1}(M)A_{2}(M)e^{i\left(\varphi_{1}-\varphi_{2}\right)} = \sqrt{I_{1}(M)I_{2}(M)}e^{i\left(\varphi_{1}-\varphi_{2}\right)}$.
\\
Il s'ensuit que : $I(M) = I_{1}(M)+I_{2}(M)+$
\\ $2\sqrt{I_{1}(M)I_{2}(M)}\cos[\varphi_{1}-\varphi_{2}\right]$.
\\
En remplaçant les phases par leur expression, on retrouve la formule
de Fresnel à deux ondes cohérentes.
\end{itemize}

\subsection*{7) Critère expérimental d'obtention d'interférences lumineuses}
Je propose là encore deux démonstrations.
\begin{itemize}
\item \underline{Première démonstration:} Soient $a_{1}(M,t) = A_{1}(M)\cos\left(\omega
t-k_{0}(S_{1}M\right)-\varphi_{\mathrm{s1}})$ et $a_{2}(M,t) = A_{2}(M)\cos\left(\omega
t-k_{0}(S_{2}M\right)-\varphi_{\mathrm{s2}})$  les champs vibratoires des deux
sources en M. \\
Le temps caractéristique de variation des phases à l'origine
$\varphi_{\mathrm{s1}}$ et $\varphi_{\mathrm{s2}}$ est $\Delta t$ (temps de cohérence
temporelle de la source). \\
Le calcul repart de l'équation (\ref{Eq.1}) ; à l'échelle de $T \sim
10^{-15}$ s (pour des ondes lumineuses), les phases à l'origine sont
quasi-constante ($\Delta t \sim  10^{-11}$ s pour des lampes
spectrales est très grand devant $T$). \\
Redécoupons la durée $\tau_{\mathrm{d}}$ en $N$ intervalles $\tau_{k}$ de
durée identique $T$. A l'échelle d'un quelconque intervalle de temps
$\tau_{k}=\left[(k-1)T,\,kT\right]$, les phases à l'origine $\varphi_{\mathrm{s1}}$ et
$\varphi_{\mathrm{s2}}$ sont  quasi-constantes ($\Delta t \gg T$). \\
Ainsi, écrivons que $\forall k \in \left[1,\,N\right]$, $\varphi_{\mathrm{s1}} =
\varphi_{\mathrm{s1,k}} = \mathrm{cste}_{k}$ et $\forall k \in \left[1,\,N\right]$, $\varphi_{\mathrm{s2}}
= \varphi_{\mathrm{s2,k}} = \mathrm{cste}'_{k}$ avec ici pour tout $t$
$\varphi_{\mathrm{s1}}=\varphi_{\mathrm{s2}}$. \\
Le calcul de l'intensité $I_{1}$ se fait comme suit : \\
$I_{1}(M) = 2<a_{1}^{2}\left(M,t>_{\mathrm{\tau_{\mathrm{d}}}} \approx
\frac{2A_{1}^{2}}{NT}\sum_{k=1}^{N}\int_{(k-1\right)T}^{kT}\cos^{2}\left(\omega
t -k_{0}(S_{1}M\right)-\varphi_{\mathrm{s1,k}})\mathrm{d}t$. \\
Or, $\forall k \in \left[1,\,N\right]$,
$\frac{1}{T}\int_{(k-1)T}^{kT}\cos^{2}\left(\omega t
-k_{0}(S_{1}M\right)-\varphi_{\mathrm{s1,k}})\mathrm{d}t = <\cos^{2}\left(\omega t
-k_{0}(S_{1}M\right)-\varphi_{\mathrm{s1,k}})>_{\mathrm{\tau_{k}}} = \frac{1}{2}$. \\
On en déduit que $\forall k \in \left[1,\,N\right]$ : $<\cos^{2}\left(\omega t
-k_{0}(S_{1}M\right)-\varphi_{\mathrm{s1}})>_{\mathrm{\tau_{\mathrm{d}}}} = \frac{1}{N}\times N
\times \frac{1}{2}=\frac{1}{2} = <\cos^{2}\left(\omega t
-k_{0}(S_{1}M\right)-\varphi_{\mathrm{s1,k}})>_{\mathrm{\tau_{k}}}$. \\
Finalement, $I_{1}(M)=A_{1}^{2}(M)$. De même, $I_{2}(M)=A_{2}^{2}(M)$. \\
Calculons maintenant le terme de corrélation entre les deux sources : $<a_{1}(M,t)a_{2}(M,t)>_{\mathrm{\tau_{\mathrm{d}}}} \approx$ \\
$<a_{1}(M,t)a_{2}(M,t)>_{\mathrm{\tau_{k}}}$. \\
En tenant compte que $\varphi_{\mathrm{s1}}
=\varphi_{\mathrm{s2}}$, on écrit : \\
$<a_{1}(M,t)a_{2}(M,t)>_{\mathrm{\tau_{k}}} = \frac{1}{2}\times
A_{1}(M)A_{2}(M)<\cos\left(2\omega t -
k_{0}\left[(S_{1}M\right)+\left(S_{2}M\right)\right]-2\varphi_{\mathrm{s1,k}})>_{\mathrm{\tau_{k}}}$ $+$
$\frac{1}{2}\times
A_{1}(M)A_{2}(M)<\cos\left(k_{0}\left[(S_{2}M\right)-\left(S_{1}M\right)\right])>_{\mathrm{\tau_{k}}}$. \\
Du fait que $\varphi_{\mathrm{s1,k}}$ est quasi-constant à l'échelle de
$\tau_{k}$, $<\cos\left(2\omega t -
k_{0}\left[(S_{1}M\right)+\left(S_{2}M\right)\right]-2\varphi_{\mathrm{s1,k}})>_{\mathrm{\tau_{k}}} =0$ et
$<\cos\left(k_{0}\left[(S_{2}M\right)-\left(S_{1}M\right)\right])>_{\mathrm{\tau_{k}}} =
\cos\left(k_{0}\left[(S_{2}M\right)-\left(S_{1}M\right)\right])$. \\
Finalement, tous calculs
rassemblés, on obtient
de nouveau la formule d'interférence à deux ondes de Fresnel (ici $\varphi_{0}=\varphi_{\mathrm{s2}}-\varphi_{\mathrm{s1}}=0$) : \\
\centerline{\mathcolorbox{gray!20}{I(M)=I_{1}(M)+I_{2}(M)+2\sqrt{I_{1}(M)I_{2}(M)}\cos\left(\Delta
\varphi\right)}}.
\item \underline{Seconde démonstration:} \\
Les deux sources sont synchrones ; le déphasage entre ces sources
$\Delta \varphi = k_{0}\delta$ est constant. \\
Le critère théorique est rempli : les deux sources sont mutuellement
cohérentes.
\end{itemize}

\subsection*{11) Contraste}
L'intensité lumineuse est maximale pour $\Delta \varphi = 2m\pi$
avec $m \in \mathbb{Z}$ : $I_{\mathrm{max}} =
I_{1}+I_{2}+2\sqrt{I_{1}I_{2}}$. \\
L'intensité lumineuse est minimale pour $\Delta \varphi = (2m+1)\pi$
avec $m \in \mathbb{Z}$ : $I_{\mathrm{min}} =
I_{1}+I_{2}-2\sqrt{I_{1}I_{2}}$. \\
Le calcul du contraste donne : $C =
\frac{I_{\mathrm{max}}-I_{\mathrm{min}}}{I_{\mathrm{max}}+I_{\mathrm{min}}} =
\frac{2\sqrt{I_{1}I_{2}}}{I_{1}+I_{2}}$. \\
Le contraste est maximal\footnote{On peut étudier le signe de la
dérivée $\frac{\mathrm{d}C}{\mathrm{d}I_{1}}$ à $I_{2}$ fixée (ou $\frac{\mathrm{d}C}{\mathrm{d}I_{2}}$
à $I_{1}$ fixée). Tous calculs menés, $\frac{\mathrm{d}C}{\mathrm{d}I_{1}} =
\frac{I_{2}}{\left(I_{1}+I_{2}\right)\sqrt{I_{1}I_{2}}}\left[I_{2}-I_{1}\right]$. Elle
s'annule pour $I_{1}=I_{2}$. Pour $I_{1} < I_{2}$,
$\frac{\mathrm{d}C}{\mathrm{d}I_{1}}
> 0$ ; pour $I_{1} > I_{2}$, $\frac{\mathrm{d}C}{\mathrm{d}I_{1}}
< 0$ ; conséquemment, $I_{1}=I_{2}$ est un maximum de $C$.} pour
$I_{1}=I_{2} = I_{0}$. Dans ce cas, il vaut : \mathcolorbox{gray!20}{C=1}.
\\
Sachant que $2\sqrt{I_{1}I_{2}} = C\left[I_{1}+I_{2}\right]$, la formule de
Fresnel devient : $I(M) = I_{1}+I_{2}+C\left[I_{1}+I_{2}\right]\cos\Delta
\varphi = \left[I_{1}+I_{2}\right]\left[1+C\cos\Delta \varphi\right]$. \\
Pour $I_{1} = I_{2} = I_{0}$ ($C=1$), la formule de Fresnel prend la
forme suivante : \mathcolorbox{gray!20}{I(M) = 2I_{0}\left[1+\cos\left(\Delta \varphi\right)\right]}.

\subsection*{12) Franges d'interférences}
\subsection*{a) Définition}
Pour une frange brillante, $\Delta \varphi = 2m\pi = k_{0}\delta =
\frac{2\pi \delta}{\lambda_{0}} = 2p\pi$ avec $p$ l'ordre
d'interférences. On en déduit que pour une frange brillante, $p=m
\in \mathbb{Z}$ : l'ordre est entier. \\
Pour une frange sombre, $\Delta \varphi = (2m+1)\pi = k_{0}\delta =
\frac{2\pi \delta}{\lambda_{0}} = 2p\pi$ avec $p$ l'ordre
d'interférences. On en déduit que pour une frange sombre,
$p=m+\frac{1}{2}$ avec $m\in\mathbb{Z}$ : l'ordre est demi-entier.

\subsection*{b) Nature des franges}
Le lieu des points $M$ tel que $I(M) = \mathrm{cste}$ équivaut au lieu des
points $M$ tel que $\Delta \varphi(M) = \mathrm{cste}$ ou encore $\delta (M)
= \left(S_{2}M\right)-\left(S_{1}M\right)$ ou encore dans un milieu transparent (l'indice
est réel) homogène (l'indice ne dépend pas du point $M$)
$S_{1}M-S_{2}M = \mathrm{cste}'$ : il s'agit d'hyperboloïdes de révolution de
droite $\left(S_{1}S_{2}\right)$.
\begin{itemize}
\item Si l'écran est placé orthogonalement à cette droite, les franges
sont circulaires ; on dit aussi que ce sont des anneaux
d'interférences.
\item Si l'écran est placé "parallèlement" à cette
droite, les franges sont quasi-rectilignes.
\end{itemize}

\subsection*{15) Perte de cohérence temporelle}
\begin{itemize}
\item Tant que $|\delta(M) < \ell_{c}$, les trains d'ondes se
superposant en $M$ proviennent du même train d'ondes : $\varphi_{\mathrm{s1}}
= \varphi_{\mathrm{s2}}$ pour tout $t$. On est dans le cadre d'application de
la formule d'interférences de Fresnel à deux ondes cohérentes (les
sources sont ponctuelles).
\item Si $|\delta(M) > \ell_{c}$, le
déphasage à l'origine $\varphi_{0} = \varphi_{\mathrm{s2}}-\varphi_{\mathrm{s1}}$
entre les sources n'est plus constant au cours du temps et varie de
façon aléatoire à l'échelle du temps de réponse du détecteur optique
: il s'ensuit que le déphasage $\Delta \varphi = k_{0}\delta
+\varphi_{0}$ varie également aléatoirement ; la seconde condition
du critère théorique de cohérence n'est plus remplie : les deux
sources ne sont plus cohérentes : on dit que l'on a perte de
cohérence temporelle de la source primaire.
\end{itemize}
\underline{Ce qui suit est un peu technique et hors-programme} (seul
le résultat
est à connaître). \\
Plus précisément, reprenons le calcul de l'intensité lumineuse
menant à la formule de Fresnel. \\
Le terme qui pose souci est le terme de corrélation \\
$<a_{1}(M,t)a_{2}(M,t)>_{\mathrm{\tau_{\mathrm{d}}}} =
\frac{A_{1}A_{2}}{2}<\cos\left(2\omega
t-k_{0}\left[(S_{1}M\right)+\left(S_{2}M\right)\right]-\varphi_{\mathrm{s1}}-\varphi_{\mathrm{s2}})>_{\mathrm{\tau_{\mathrm{d}}}} +
\frac{A_{1}A_{2}}{2}<\cos\left(k_{0}\left[(S_{2}M\right)-\left(S_{1}M\right)\right]+\varphi_{\mathrm{s2}}-\varphi_{\mathrm{s1}})>_{\mathrm{\tau_{\mathrm{d}}}}$.
\\
Pour les mêmes raisons que lors de l'obtention de la formule de
Fresnel à partir du critère expérimental, (sur chaque intervalle
$\tau_{k} = \left[(k-1)T,kT\right]$, $\varphi_{\mathrm{s1}}+\varphi_{\mathrm{s2}}$ est constant)
$<\cos\left(2\omega
t-k_{0}\left[(S_{1}M\right)+\left(S_{2}M\right)\right]-\varphi_{\mathrm{s1}}-\varphi_{\mathrm{s2}})>_{\mathrm{\tau_{\mathrm{d}}}} =
0$. \\
En revanche, le calcul de
$<\cos\left(k_{0}\left[(S_{2}M\right)-\left(S_{1}M\right)\right]+\varphi_{\mathrm{s2}}-\varphi_{\mathrm{s1}})>_{\mathrm{\tau_{\mathrm{d}}}}$
doit être explicité. \\
On écrit : \\
$<\cos\left(k_{0}\delta
+\varphi_{\mathrm{s2}}-\varphi_{\mathrm{s1}}\right)>_{\mathrm{\tau_{\mathrm{d}}}}\approx
\frac{1}{NT}\sum_{k=1}^{N}\int_{(k-1)T}^{kT}\cos\left(k_{0}\delta
+\varphi_{\mathrm{s2,k}}-\varphi_{\mathrm{s1,k}}\right)\mathrm{d}t = \frac{1}{N}\sum_{k=1}^{N}<\cos\left(k_{0}\delta
+\varphi_{\mathrm{s2,k}}-\varphi_{\mathrm{s1,k}}\right)>_{\mathrm{\tau_{k}}}=\frac{1}{N}\sum_{k=1}^{N}\cos\left(k_{0}\delta
+\varphi_{\mathrm{s2,k}}-\varphi_{\mathrm{s1,k}}\right)$. \\
Ici, l'hypothèse cruciale est le caractère aléatoire des valeurs
prises par le déphasage
$\varphi_{\mathrm{0,k}}=\varphi_{\mathrm{s2,k}}-\varphi_{\mathrm{s1,k}}$ d'un intervalle
$\tau_{k}$ à $\tau_{\mathrm{k'}}$ ($k'\neq k$). \\
A l'échelle de $\tau_{\mathrm{d}}$,
$\cos\left(k_{0}\delta +\varphi_{\mathrm{s2}}-\varphi_{\mathrm{s1}}\right)$ prend de façon
équiprobable toutes les valeurs comprises entre $-1$ et $1$ ; aussi,
$\cos\left(k_{0}\delta +\varphi_{\mathrm{s2}}-\varphi_{\mathrm{s1}}\right) =
\frac{1}{N}\sum_{k=1}^{N}\cos\left(k_{0}\delta
+\varphi_{\mathrm{s2,k}}-\varphi_{\mathrm{s1,k}}\right)=0$. \\
Au final, lorsque $|\delta| > \ell_{c}$, $I_{\mathrm{12}} = 0$. \\
\textbf{En conséquence, lorsque les deux sources sont incohérentes
(ici par perte de cohérence temporelle), l'intensité en un point $M$
est la somme des intensités : \mathcolorbox{gray!20}{I(M)=I_{1}(M)+I_{2}(M)}}.

\section*{II. Interférences entre ondes planes cohérentes}
\subsection*{2) Eclairement}
Il faut savoir jongler entre les grandeurs éclairement et intensité
lumineuse. Dans notre programme, ces deux notions sont confondues.
\\
Considérons deux ondes planes cohérentes; de vecteurs d'onde
$\overrightarrow{k}_{1}$ et $\overrightarrow{k}_{2}$, issues de deux
sources ponctuelles à l'infini, de même éclairement $\mathcal{E}_{0}$. \\
Les champs vibratoires associés sont celles d'ondes planes :
$a_{1}(M,t) = A_{0}\cos\left(\omega t
-\overrightarrow{k}_{1}.\overrightarrow{OM}-\varphi_{0}\right)$ et
$a_{2}(M,t) = A_{0}\cos\left(\omega t
-\overrightarrow{k}_{2}.\overrightarrow{OM}-\varphi_{0}\right)$ avec $O$
une origine (ce point appartient conjointement au plan formé par les
deux vecteurs d'onde et au plan de l'écran qui est le plan $(Oxy)$)
et $\varphi_{0}$ la phase à l'origine en $O$ constante. \\
En notation complexe, les champs vibratoires sont :
$\underline{a}_{1}(M,t) = A_{0}e^{i\varphi_{1}}$ avec $\varphi_{1} =
\omega t -\overrightarrow{k}_{1}.\overrightarrow{OM}-\varphi_{0}$ et
$\underline{a}_{2}(M,t) = A_{0}e^{i\varphi_{2}}$ avec $\varphi_{2} =
\omega t
-\overrightarrow{k}_{2}.\overrightarrow{OM}-\varphi_{0}$. \\
Or, $\mathcal{E}_{1}(M) =
\underline{a}_{1}(M,t)\underline{a}_{1}^{*}(M,t)=|\underline{a}_{1}|^{2}(M,t)=A_{0}^{2}
$ ; de même, $\mathcal{E}_{2}(M) =
\underline{a}_{2}(M,t)\underline{a}_{2}^{*}(M,t)=|\underline{a}_{2}|^{2}(M,t)=
A_{0}^{2}$.
\\
L'éclairement $\mathcal{E}(M)$ se calcule par : $\mathcal{E}(M) =
\underline{a}(M,t)\underline{a}^{*}(M,t) =
\left[\underline{a}_{1}(M,t)+\underline{a}_{2}(M,t)\right]\left[\underline{a}_{1}^{*}(M,t)+\underline{a}_{2}^{*}(M,t)\right]
= 2\mathcal{E}_{0}+ \underline{a}_{1}(M,t)\underline{a}_{2}^{*}(M,t)
+ \underline{a}_{1}^{*}(M,t)\underline{a}_{2}(M,t)$, ou encore : \\
$\mathcal{E}(M) = 2\mathcal{E}_{0}+
\mathcal{E}_{0}e^{i\left(\varphi_{1}-\varphi_{2}\right)}+\mathcal{E}_{0}e^{-i\left(\varphi_{1}-\varphi_{2}\right)}$.
\\
Il s'ensuit que : $\mathcal{E}(M) =
2\mathcal{E}_{0}(M)\left(1+\cos[\varphi_{1}-\varphi_{2}\right]\right)$.
\\
On remplaçe les phases par leur expression et on retrouve la formule
de Fresnel à deux ondes cohérentes (planes ici) : \\
\centerline{\mathcolorbox{gray!20}{\mathcal{E}(M) = 2\mathcal{E}_{0}(M)\left(1+\cos[\Delta
\varphi\right]\right)}} avec $\Delta \varphi =
\left[\overrightarrow{k}_{1}-\overrightarrow{k}_{2}\right].\overrightarrow{OM}$
ou (en exploitant la parité de la fonction cosinus) : $\Delta
\varphi =
\left[\overrightarrow{k}_{2}-\overrightarrow{k}_{1}\right].\overrightarrow{OM}$.
\\
Par la suite, on conservera (arbitrairement) $\Delta \varphi =
\left[\overrightarrow{k}_{1}-\overrightarrow{k}_{2}\right].\overrightarrow{OM}$.
On posera donc $\overrightarrow{K} =
\overrightarrow{k}_{1}-\overrightarrow{k}_{2}$. \\
Explicitons maintenant les coordonnées des vecteurs d'onde dans la
base $(Oxyz)$ ($k_{0} = \frac{2\pi}{\lambda_{0}}$ est le nombre d'onde dans le vide). \\
On a : $\overrightarrow{k}_{1} =
k_{0}\left[\cos\left(\frac{\alpha}{2}\right)\overrightarrow{u}_{x}+\sin\left(\frac{\alpha}{2}\right)\overrightarrow{u}_{z}\right]$
et $\overrightarrow{k}_{2} =
k_{0}\left[\cos\left(\frac{\alpha}{2}\right)\overrightarrow{u}_{x}-\sin\left(\frac{\alpha}{2}\right)\overrightarrow{u}_{z}\right]$.
\\
On en déduit immédiatement le déphasage entre les ondes interférant
en un point $M$ par : $\Delta \varphi =
2k_{0}x\sin\left(\frac{\alpha}{2}\right)\overrightarrow{u}_{z} =
\frac{4\pi}{\lambda_{0}}x\sin\left(\frac{\alpha}{2}\right)\overrightarrow{u}_{z}$
et la différence de marche par : $\delta =\frac{\Delta
\varphi}{k_{0}}=2x\sin\left(\frac{\alpha}{2}\right)$. \\
La formule de Fresnel à deux ondes planes cohérentes (avec ici
$\mathcal{E}_{1} = \mathcal{E}_{2}$) est :
\begin{eqnarray}\label{Eq.2}
\mathcal{E} =
2\mathcal{E}_{0}\left[1+\cos\left(\frac{4\pi}{\lambda_{0}}x\sin(\frac{\alpha}{2}\right))\right].
\end{eqnarray}

\subsection*{3) Interfrange $i$}
Je propose deux démonstrations.
\begin{itemize}
\item \underline{Première démonstration :} L'interfrange est la période spatiale de l'éclairement ou de
l'intensité lumineuse ; ceci se traduit mathématiquement par :
\begin{eqnarray}\label{Eq.3}
\mathcal{E}(x)  =2\mathcal{E}_{0}\left[1+\cos\left(\frac{2\pi x}{i}\right)\right].
\end{eqnarray}
En comparant les équations (\ref{Eq.2}) et (\ref{Eq.3}), on obtient
directement : \mathcolorbox{gray!20}{i =
\frac{\lambda_{0}}{2\sin\left(\frac{\alpha}{2}\right)}}.
\item \underline{Seconde démonstration :} On utilise ici la
définition basique de l'interfrange. \\
Notons $x_{m}$ la position du centre de la $m^{\grave{e}me}$ frange
brillante ; on a : $\delta_{m} = m\lambda_{0} =
2x_{m}\sin\left(\frac{\alpha}{2}\right)$
avec $m \in \mathbb{Z}$. \\
De même, soit $x_{\mathrm{m+1}}$ la position du centre de la
$m+1^{\grave{e}me}$ frange brillante ; on a : $\delta_{\mathrm{m+1}} =
(m+1)\lambda_{0} = 2x_{\mathrm{m+1}}\sin\left(\frac{\alpha}{2}\right)$ avec $m \in
\mathbb{Z}$. \\
On en déduit que : $i = x_{\mathrm{m+1}}-x_{m} =
\frac{\delta_{\mathrm{m+1}}}{2\sin\left(\frac{\alpha}{2}\right)}-\frac{\delta_{m}}{2\sin\left(\frac{\alpha}{2}\right)}$
$=$ $\frac{\lambda_{0}}{2\sin\left(\frac{\alpha}{2}\right)}$. \\
Si $\alpha \ll1$, $\sin\left(\frac{\alpha}{2}\right) \approx \frac{\alpha}{2}$,
d'où : \mathcolorbox{gray!20}{i \approx \frac{\lambda_{0}}{\alpha}}.
\end{itemize}

\section*{III. Interférences en lumière polychromatique}
\subsection*{2) Proposition}
Une source polychromatique primaire ponctuelle  constituée de deux
radiations monochromatiques peut être vue comme la superposition de
deux sources primaires ponctuelles monochromatiques. Les trains
d'onde qu'elles émettent ne proviennent pas d'un même train d'onde
originel : les deux sources sont incohérentes entre elles. Les
intensités lumineuses s'ajoutent (une démonstration quantitative
sera proposée dans le TD O2).

\subsection*{3) Cas du doublet spectral}
A chaque radiation monochromatique de la source primaire ponctuelle,
on associe une source primaire, ponctuelle, monochromatique de
longueur d'onde
$\lambda_{1}$ ou $\lambda_{2}$. \\
En l'absence d'interférences, la source primaire, ponctuelle,
monochromatique primaire  à $\lambda_{1}$, d'intensité lumineuse
$I_{0}$ crée deux sources secondaires, ponctuelles, cohérentes car
monochromatiques (= synchrones) à $\lambda_{1}$, de même
intensité $I_{0}$. \\
Dès lors, la formule de Fresnel est licite et on écrit :
$I_{\mathrm{\lambda_{1}}}(M) = 2I_{0}\left[1+\cos\left(\frac{2\pi
\delta}{\lambda_{1}}\right)\right]$. \\
De même, en l'absence d'interférences, la source primaire,
ponctuelle, monochromatique à $\lambda_{2}$, d'intensité lumineuse
$I'_{0}$ (pour simplifier, on suppose que $I_{0}=I_{0}'$) crée deux
sources secondaires, ponctuelles, cohérentes car monochromatiques à
$\lambda_{2}$, de même
intensité $I_{0}$. \\
Ainsi, on a : $I_{\mathrm{\lambda_{2}}}(M) = 2I_{0}\left[1+\cos\left(\frac{2\pi
\delta}{\lambda_{2}}\right)\right]$. \\
Les deux sources primaires, ponctuelles, monochromatiques à
$\lambda_{1}$ et $\lambda_{2}$ étant incohérentes entre elles (car
non synchrones), les intensités qu'elles produisent en $M$ par le
jeu des sources secondaires, s'ajoutent : \\
$I(M) =
2I_{0}\left[1+\cos\left(\frac{2\pi
\delta}{\lambda_{1}}\right)\right]+2I_{0}\left[1+\cos\left(\frac{2\pi
\delta}{\lambda_{2}}\right)\right]$ ou encore ($\cos(p)+\cos(q) =
2\cos\left(\frac{p+q}{2}\right)\cos\left(\frac{p-q}{2}\right)$) :
\\
$I(M) =
2I_{0}\left[2+2\cos\left(\pi\delta[\frac{1}{\lambda_{1}}+\frac{1}{\lambda_{2}}\right]\right)\times
\cos\left(\pi\delta[\frac{1}{\lambda_{1}}-\frac{1}{\lambda_{2}}\right]\right)\right]$. \\
Par ailleurs, en notant $\lambda_{0} =
\frac{\lambda_{1}+\lambda_{2}}{2}$ la longueur d'onde moyenne et
$\Delta \lambda = \lambda_{2}-\lambda_{1}$, on a ($\lambda_{2} >
\lambda_{1}$), $\lambda_{1} = \lambda_{0}-\frac{\Delta \lambda}{2}$
et $\lambda_{2} = \lambda_{0}+\frac{\Delta \lambda}{2}$. \\
On en déduit que :
\begin{itemize}
\item $\frac{1}{\lambda_{1}}+\frac{1}{\lambda_{2}} =
\frac{\lambda_{2}\lambda_{1}}{\lambda_{1}+\lambda_{2}} \approx
\frac{2}{\lambda_{0}}$ ;
\item $\frac{1}{\lambda_{1}}-\frac{1}{\lambda_{2}} =
\frac{\lambda_{2}-\lambda_{1}}{\lambda_{1}\lambda_{2}} \approx
\frac{\Delta \lambda}{\lambda_{0}^{2}}$.
\end{itemize}
On en déduit que l'intensité lumineuse s'exprime par : \\
\centerline{\mathcolorbox{gray!20}{I(M) = 4I_{0}\left[1+\cos\left(\frac{2\pi \delta }{\lambda_{0}}\right)\times
\cos\left(\frac{\pi \delta \Delta \lambda}{\lambda_{0}^{2}}\right)\right]}}. \\
Cette expression se met sous la forme : \\
$I(M) =
4I_{0}\left[1+V\left(\delta\right)\cos\left(\frac{2\pi \delta }{\lambda_{0}}\right)\right]$. \\
$V\left(\delta\right)$ est la fonction de visibilité. Cette fonction de période
spatiale $\Lambda  = \frac{\lambda_{0}^{2}}{2\Delta \lambda} \gg
\lambda_{0}$ (cette fonction est de la forme $V\left(\delta\right) =
\cos\left(\frac{2\pi \delta }{\Lambda}\right)$) varie lentement à l'échelle de
$\lambda_{0}$ : elle module l'intensité lumineuse et constitue
"l'enveloppe" de l'intensité lumineuse. On observe le phénomène de
"battements" (Cf. Annexe n$^{\circ}$6 du support de cours O2).

\subsection*{4) Raie spectrale de largeur non nulle: modélisation du profil par un profil
rectangulaire}
La démonstration comporte plusieurs étapes de
raisonnement. Détaillons les.
\begin{itemize}
\item On décompose la source primaire, ponctuelle, polychromatique en
une infinité de sources primaires, ponctuelles
quasi-monochromatiques de fréquence $\nu$ et d'épaisseur spectrale
$\mathrm{d}\nu$.
\item L'intensité lumineuse en un point $M$ due à la source
ponctuelle quasi-monochromatique de fréquence $\in[\nu, \nu + \mathrm{d}\nu\right]$
est donnée par la formule de Fresnel à deux ondes (cette source
produit deux sources secondaires, ponctuelles, cohérentes car
synchrones, de même intensité $\mathrm{d}I_{0}$) : $\mathrm{d}I(M) =
2\mathrm{d}I_{0}\left[1+\cos\frac{2\pi\delta(M)}{\lambda}\right]=2\mathrm{d}I_{0}\left[1+\cos\frac{2\pi
\nu\delta(M)}{c}\right]$. \\
Soit $I_{0} = f_{0}\Delta \nu$ l'intensité lumineuse produite par la
source primaire ponctuelle dans le modèle du profil spectral
("taillé à
la serpe") rectangulaire. On écrit aussi $\mathrm{d}I_{0} = f_{0}\mathrm{d}\nu$. \\
On a : $\mathrm{d}I(M) = 2f_{0}\left[1+\cos\frac{2\pi
\nu\delta(M)}{c}\right]\mathrm{d}\nu$. \\
Posons $\tau(M) = \frac{\delta(M)}{c}$ (c'est un temps
caractéristique du phénomène d'interférences : il exprime le
décalage temporel entre les deux ondes issues de la source primaire
(et divisées par division d'amplitude ou par division du front
d'ondes) et ayant parcouru des chemins optiques
différents avant d'interférer en $M$) ; on en déduit que : \\
\centerline{\mathcolorbox{gray!20}{\mathrm{d}I(M) = 2f_{0}\left[1+\cos\left(2\pi\nu\tau(M\right))\right]\mathrm{d}\nu}}.
\item Les différentes radiations monochromatiques composant le profil
spectral de la source primaire sont incohérentes entre elles : les
intensités qu'elles produisent, via le jeu des sources secondaires,
au point $M$ de l'écran s'ajoutent ; il s'ensuit que :  $I(M) =
\int_{\nu_{0}-\frac{\Delta \nu}{2}}^{\nu_{0}+\frac{\Delta
\nu}{2}}\mathrm{d}I\left(M,\nu\right)$.
\item Le calcul détaillé est le suivant :
\\
$I(M) = 2f_{0}\int_{\nu_{0}-\frac{\Delta
\nu}{2}}^{\nu_{0}+\frac{\Delta \nu}{2}}\left[1+\cos\frac{2\pi
\nu\delta(M)}{c}\right]\mathrm{d}\nu = 2f_{0}\Delta \nu +
\int_{\nu_{0}-\frac{\Delta \nu}{2}}^{\nu_{0}+\frac{\Delta
\nu}{2}}cos\left(\frac{2\pi \nu\delta(M\right)}{c})\mathrm{d}\nu = 2I_{0}+
\frac{2I_{0}}{\Delta \nu}\int_{\nu_{0}-\frac{\Delta
\nu}{2}}^{\nu_{0}+\frac{\Delta \nu}{2}}cos\left(\frac{2\pi
\nu\delta(M\right)}{c})\mathrm{d}\nu$. \\
Le calcul se poursuit comme suit :
\\
$I(M) = 2I_{0}+\frac{2I_{0}}{\Delta \nu}\times\frac{c}{2\pi
\delta}\left[\sin\left(\frac{2\pi \nu \delta}{c}\right)\right]_{\mathrm{\nu_{0}}-\frac{\Delta
\nu}{2}}^{\nu_{0}+\frac{\Delta \nu}{2}} = 2I_{0}\biggl\left(1+\frac{c}{2\pi \delta \Delta
\nu}(\sin[\frac{2\pi\delta}{c}(\nu_{0}+\frac{\Delta\nu}{2}\right)\right]
-\sin\left(\frac{2\pi\delta}{c}\left[\nu_{0}-\frac{\Delta\nu}{2}\right]\right))\biggr)$.
\\
Or, $\sin a - \sin b = 2\sin\left(\frac{a-b}{2}\right)\cos\left(\frac{a+b}{2}\right)$. Il
s'ensuit que : \\
$I(M) = 2I_{0}\left[1+\frac{c}{\pi\delta \Delta \nu}\sin\left(\frac{\pi
\delta\Delta \nu}{c}\right)\cos\left(\frac{2\pi \delta \nu_{0}}{c}\right)\right]$. \\
On pose \mathcolorbox{gray!20}{V = \frac{c}{\pi\delta \Delta \nu}\sin\left(\frac{\pi
\delta\Delta \nu}{c}\right) = sinc\left(\frac{\pi \delta\Delta \nu}{c}\right)}, la
fonction de
visibilité. \\
Finalement, on obtient la formule de l'intensité lumineuse (ou de
l'éclairement) suivante :
\\
\centerline{\mathcolorbox{gray!20}{I(M) = 2I_{0}\left[1+V\cos\left(\frac{2\pi \delta(M\right)
\nu_{0}}{c})\right]}} ou encore : \\
\centerline{\mathcolorbox{gray!20}{I(M) =2I_{0}\left[1+V\cos\left(\frac{2\pi \delta(M\right)}{\lambda_{0}})\right]}}. \\
En introduisant $\tau(M) = \frac{\delta(M)}{c}$, il vient : $I(M) =
2I_{0}\left[1+V\left(\tau(M\right))\cos\left(2\pi\nu_{0}\tau(M\right))\right]$.
\item L'allure de l'intensité en fonction de $\tau$ est donnée sur la
figure \ref{Fig.3}. \\
La première annulation de la visibilité qui correspond au brouillage
des franges est : $V\left(\tau(M\right))=0$, soit $sinc\left(\pi\Delta \nu
\tau(M\right))=0$, puis : $\pi\Delta \nu \tau(M) = \pi$ (et NON 0 ! car un
équivalent de $sinc(x)$ au voisinage de $x=0$ est 1 !). On en déduit
qu'à la première annulation\footnote{On rappelle que
$\frac{1}{\Delta \nu} = \frac{1}{\nu_{1}-\nu_{2}}$ avec $\nu_{1} =
\frac{c}{\lambda_{1}}$ et $\lambda_{1} = \lambda_{0}-\frac{\Delta
\lambda}{2}$, $\nu_{2} = \frac{c}{\lambda_{2}}$ et $\lambda_{2} =
\lambda_{0}+\frac{\Delta \lambda}{2}$, d'où : $\frac{1}{\Delta \nu}
= \frac{1}{\frac{c}{\lambda_{0}-\frac{\Delta
\lambda}{2}}-\frac{c}{\lambda_{0}+\frac{\Delta \lambda}{2}}} = \frac{\left(\lambda_{0}-\frac{\Delta
\lambda}{2}\right)\left(\lambda_{0}+\frac{\Delta \lambda}{2}\right)}{c\Delta\lambda}
\approx \frac{\lambda_{0}^{2}}{c\Delta\lambda}$.}, \mathcolorbox{gray!20}{\tau(M) =
\frac{1}{\Delta \nu}=\frac{\lambda_{0}^{2}}{c\Delta
\lambda}=\frac{\ell_{c}}{c}}.
\\
Cette première annulation permet d'évaluer le temps de cohérence
temporelle de la source : $\tau(M) = \frac{1}{\Delta \nu}$, ainsi
que la longueur de cohérence temporelle $\ell_{c} =
\frac{\lambda_{0}^{2}}{\Delta\lambda}$.


\begin{wrapfigure}{r}{0.25\textwidth}
\epsfig{file=Fig3.PNG,height=4.5cm, width=7.5cm}
\caption\protect{Tracé de $\frac{I}{2I_{0}}$ en fonction de $\tau(M)$. On a
pris pour un confort visuel $\frac{\Delta \nu}{\nu_{0}}=10$. La
première annulation de la visibilité donne accès au temps de
cohérence temporel de la source $\frac{\ell_{c}}{c} =
\tau_{\mathrm{annulation}} = \Delta t$.}\label{Fig.3}
\end{wrapfigure}


\end{itemize}
On remarque que plus la largeur spectrale de la source est faible,
plus la longueur de cohérence temporelle de la source primaire
augmente. A la limite où $\Delta \lambda \rightarrow 0$, $\ell_{c}
\rightarrow +\infty$ : on retrouve le caractère éternel de la source
monochromatique qui n'a ni début , ni fin. La durée d'émission est
infinie !

\subsection*{7)  Critère semi-quantitatif de brouillage des franges (critère de cohérence temporelle)}
Appliquons le critère au cas de la source polychromatique de profil
spectral idéalisé rectangulaire. \\
Au centre du profil spectral, on peut écrire que : $\delta(M) =
p_{1}\lambda_{0}$ ; pour la radiation du profil à
$\lambda_{0}+\frac{\Delta \lambda}{2}$, on a : $\delta(M) =
p_{2}\left[\lambda_{0}+\frac{\Delta \lambda}{2}\right]$. On en déduit en $M$ la
différence de l'ordre d'interférences : $\Delta p = p_{1}-p_{2} =
\frac{\delta}{\lambda_{0}}-\frac{\delta}{\lambda_{0}+\frac{\Delta
\lambda}{2}} =
\delta[\frac{1}{\lambda_{0}}-\frac{1}{\lambda_{0}+\frac{\Delta
\lambda}{2}}\right] = \delta \times \frac{\frac{\Delta
\lambda}{2}}{\lambda_{0}\left[\lambda_{0}+\frac{\Delta \lambda}{2}\right]}
\approx \delta\frac{\Delta \lambda}{2\lambda_{0}^{2}}$. \\
D'après le critère, il y a brouillage des franges à la condition que
$\Delta p > \frac{1}{2}$, soit : $\delta\frac{\Delta
\lambda}{2\lambda_{0}^{2}} > \frac{1}{2}$ ou encore : $\delta
> \frac{\lambda_{0}^{2}}{\Delta \lambda} = \ell_{c}$. \\
On retrouve par une approche semi-qualitative le résultat du calcul
complet mené dans le paragraphe III.4.

\section*{IV. Interférences par division du front d'ondes}
\subsection*{1) Dispositif des trous d'Young}
\subsubsection*{b) Cas de la source primaire ponctuelle quasi-monochromatique}
Dans la base cartésienne $(Oxyz)$, les différents points \textit{ad
hoc} ont pour coordonnées : $S\left(0,0,-(\mathrm{d}+\mathrm{D}\right))$,
$S_{1}\left(\frac{a}{2},0,-\mathrm{D}\right)$, $S_{2}\left(-\frac{a}{2},0,-\mathrm{D}\right)$ et $M\left(x,y,0\right)$.
\\
Calculons la distance $S_{1}M$ : \\
$S_{1}M = \sqrt{\left(x-\frac{a}{2}\right)^{2}+y^{2}+\mathrm{D}^{2}}$. Or, la distance
$\mathrm{D}$ est très grande devant les autres distances du problème. Aussi,
réécrivons $S_{1}M$ sous la forme: \\
$S_{1}M =
\mathrm{D}\left[1+\frac{y^{2}}{\mathrm{D}^{2}}+\frac{\left(x-\frac{a}{2}\right)^{2}}{\mathrm{D}^{2}}\right]^{\frac{1}{2}}$.
\\
Or, les quantités $\frac{y}{\mathrm{D}}$ et $\frac{x-\frac{a}{2}}{\mathrm{D}}$ sont
des infiniments petits du premier ordre. Posons $u^{2} =
\frac{y^{2}}{\mathrm{D}^{2}}+\frac{\left(x-\frac{a}{2}\right)^{2}}{\mathrm{D}^{2}}$ un infiniment
petit du second ordre. On sait qu'un DL à "l'ordre 2" donne :
$\left(1+u^{2}\right)^{\alpha} \approx 1+\alpha u^{2}$. \\
Il s'ensuit que  : $S_{1}M \approx
\mathrm{D}\left[1+\frac{y^{2}}{2\mathrm{D}^{2}}+\frac{\left(x-\frac{a}{2}\right)^{2}}{2\mathrm{D}^{2}}\right]$.
\\
De même, $S_{2}M =
\mathrm{D}\left[1+\frac{y^{2}}{\mathrm{D}^{2}}+\frac{\left(x+\frac{a}{2}\right)^{2}}{\mathrm{D}^{2}}\right]^{\frac{1}{2}}$,
puis : $S_{2}M \approx
\mathrm{D}\left[1+\frac{y^{2}}{2\mathrm{D}^{2}}+\frac{\left(x+\frac{a}{2}\right)^{2}}{2\mathrm{D}^{2}}\right]$. \\
La différence de marche est : $\delta = \left(SS_{2}M\right) - \left(SS_{1}M\right) = \left(SS_{2}\right) + \left(S_{2}M\right)-\left[\left(SS_{1}\right)+\left(S_{1}M\right)\right]$.  \\
Or, $S$ est une source ponctuelle qui émet des ondes sphériques.
Pour toute date $t$, la surface d'onde issue de $S$ atteignant
$S_{1}$ atteint également $S_{2}$ : $S_{1}$ et $S_{2}$ appartiennent
à cette surface d'onde, donc : $\left(SS_{1}\right) = \left(SS_{2}\right)$. On travaille
par ailleurs dans l'air assimilé au vide, donc : $\left(SS_{1}\right) = SS_{1}$
et $\left(SS_{2}\right) = SS_{2}$. \\
Dit autrement, les sources secondaires sont équidistantes de $S$.
\\
Il ne reste plus que : $\delta = SS_{2}-SS_{1}$. A l'aide des
approximations précédentes, on a : \\
$\delta \approx
\mathrm{D}\left[1+\frac{y^{2}}{2\mathrm{D}^{2}}+\frac{\left(x+\frac{a}{2}\right)^{2}}{2\mathrm{D}^{2}}\right]$ $-$
$\mathrm{D}\left[1+\frac{y^{2}}{2\mathrm{D}^{2}}+\frac{\left(x-\frac{a}{2}\right)^{2}}{2\mathrm{D}^{2}}\right]$. \\
On développe ce calcul et on aboutit à : $\delta \approx
\frac{\left(x+\frac{a}{2}\right)^{2}}{2\mathrm{D}}-\frac{\left(x-\frac{a}{2}\right)^{2}}{2\mathrm{D}}$, ou
encore : \mathcolorbox{gray!20}{\delta(x) = \frac{ax}{\mathrm{D}}}. \\
\begin{itemize}
\item Calculons l'intensité en un point de l'écran. \\
Notons $I_{0}$ l'intensité lumineuse de chacune des sources
secondaires. En vertu de la formule de Fresnel, l'intensité en $M$
est : \mathcolorbox{gray!20}{I(M) = 2I_{0}\left[1+\cos\left(\frac{2\pi a x}{\lambda_{0}\mathrm{D}}\right)\right]}.
\item Déterminons la nature des franges. \\
Le lieu des points $M$ sur l'écran d'égale intensité est le lieu
d'égale différence de marche donc d'égal $x$ : les franges sont
rectilignes. \\
\underline{Variante :} L'écran étant placé parallèlement aux sources
secondaires, les franges sont rectilignes.
\item Calculons l'interfrange. L'interfrange est la période spatiale
de l'éclairement, soit : $I(x)  =2I_{0}\left[1+\cos\left(\frac{2\pi x }{i}\right)\right]$.
En comparant cette formule à la précédente, on obtient de suite :
\mathcolorbox{gray!20}{i=\frac{\lambda_{0}\mathrm{D}}{a}}. \\
Une autre méthode consiste à expliciter la définition "basique" de
l'interfrange. Pour ce faire, en notant, par exemple,  $x_{m}$ le
centre de la $m^{\grave{e}me}$ frange sombre, on a : $\delta_{m} =
\left(m+\frac{1}{2}\right)\lambda_{0} = \frac{ax_{m}}{\mathrm{D}}$, d'où : $x_{m} =
\left(m+\frac{1}{2}\right)\frac{\lambda_{0}\mathrm{D}}{a}$. De même, $x_{\mathrm{m+1}} =
\left(m+\frac{3}{2}\right)\frac{\lambda_{0}\mathrm{D}}{a}$, d'où :
\mathcolorbox{gray!20}{i=x_{\mathrm{m+1}}-x_{m} = \frac{\lambda_{0}\mathrm{D}}{a}}.
\end{itemize}

\subsection*{3)  Utilisation d'une source étendue et quasi-monochromatique: cohérence
spatiale} La démonstration comporte plusieurs étapes de
raisonnement. Détaillons les.
\begin{itemize}
\item On décompose la source primaire $S$, quasi-monochromatique (de longueur d'onde $\lambda_{0}$), étendue
en une infinité de sources primaires, quasi-monochromatiques,
ponctuelles $S'$ situées aux absisses $X$ ($-\frac{b}{2}\leq X \leq
+\frac{b}{2}$), d'épaisseur $\mathrm{d}X$, d'intensité élémentaire $\mathrm{d}I_{0}$.
\item Montrons que la différence de marche issue de $S'$ est : $\delta \approx
\frac{ax}{\mathrm{D}}+\frac{aX}{\mathrm{d}}$. Ici $x$ est fixé (le point $M$ de
l'écran est fixe). \\
Dans la base cartésienne $(Oxyz)$, les différents points \textit{ad
hoc} ont pour coordonnées : $S'\left(X,0,-(\mathrm{d}+\mathrm{D}\right))$,
$S_{1}\left(\frac{a}{2},0,-\mathrm{D}\right)$, $S_{2}\left(-\frac{a}{2},0,-\mathrm{D}\right)$ et $M\left(x,y,0\right)$.
\\
Calculons la distance $S'S_{1}$ : \\
$S'S_{1}= \sqrt{\left(\frac{a}{2}-X\right)^{2}+y^{2}+\mathrm{d}^{2}}$. Or, la distance
$\mathrm{d}$ est très grande devant les autres distances $b$, $a$, $y$.
Aussi, réécrivons $S'S_{1}$ sous la forme : $S'S_{1} =
\mathrm{d}\left[1+\frac{y^{2}}{\mathrm{d}^{2}}+\frac{\left(\frac{a}{2}-X\right)^{2}}{\mathrm{d}^{2}}\right]^{\frac{1}{2}}$.
\\
Un DL à "l'ordre 2" conduit à : \\
$S'S_{1} \approx
\mathrm{d}\left[1+\frac{y^{2}}{2\mathrm{d}^{2}}+\frac{\left(X-\frac{a}{2}\right)^{2}}{2\mathrm{d}^{2}}\right]$.
\\
De même, $S'S_{2} \approx
\mathrm{d}\left[1+\frac{y^{2}}{2\mathrm{d}^{2}}+\frac{\left(X+\frac{a}{2}\right)^{2}}{2\mathrm{d}^{2}}\right]$. \\
La différence de marche est : $\delta = \left(SS_{2}M\right) - \left(SS_{1}M\right) = \left(SS_{2}\right) + \left(S_{2}M\right)-\left[\left(SS_{1}\right)+\left(S_{1}M\right)\right]$.  \\
On travaille dans l'air ; comme précédemment, on obtient que :
$S_{2}M-S_{1}M \approx \frac{ax}{\mathrm{D}}$ la différence de marche en
aval des sources secondaires. \\
On en déduit que: \mathcolorbox{gray!20}{\delta = S'S_{2}-S'S_{1} + \frac{ax}{\mathrm{D}}
\approx \frac{aX}{\mathrm{d}}+\frac{ax}{\mathrm{D}}}.
\item L'intensité lumineuse en un point $M$ due à la source primaire,
ponctuelle, située dans la bande $\in[X, X + \mathrm{d}X\right]$,
quasi-monochromatique est $\mathrm{d}I_{0}$. \\
Notons $I_{0} = Ab$ l'intensité lumineuse produite par la source
primaire étendue de largeur $b$. \\
Avec l'hypothèse que la source primaire étendue émet un éclairement
uniforme, le modèle du profil spatial ("taillé à la serpe" gauloise
!) rectangulaire est licite. On écrit donc $\mathrm{d}I_{0} = A\mathrm{d}X$.
\item Chaque source primaire, ponctuelle, quasi-monochromatique créé deux sources
secondaires, ponctuelles, cohérentes de même intensité $\mathrm{d}I_{0}$.
\item L'intensité en un point $M$ est donnée par la formule de Fresnel à deux ondes : $\mathrm{d}I(x,X) =
2\mathrm{d}I_{0}\left[1+\cos\frac{2\pi\delta(x,X)}{\lambda}\right]=2\mathrm{d}I_{0}\left[1+\cos\frac{2\pi\delta(x,X)}{\lambda_{0}}\right]$. \\
On a encore : $\mathrm{d}I(x,X) =
2A\left[1+\cos\frac{2\pi\delta(x,X)}{\lambda_{0}}\right]\mathrm{d}X$.
\item Les différentes sources primaires, ponctuelles (quasi-monochromatiques)
composant la source primaire étendue (quasi-monochromatique) sont
incohérentes entre elles\footnote{Les différents trains d'ondes
qu'elles émettent sont sans relation de phase entre eux : à une date
donnée, les différentes phases à l'origine varient aléatoirement
d'un train d'onde à l'autre.} : les intensités qu'elles produisent
au point $M$ de l'écran s'ajoutent ; il s'ensuit que : $I(x,X) =
\int_{-\frac{b}{2}}^{+\frac{b}{2}}\mathrm{d}I(x,X)$.
\item Le calcul détaillé est le suivant :
\\
$I(x,X) = 2A\int_{-\frac{b}{2}}^{+\frac{b}{2}}\left[1+\cos\left(\frac{2\pi
ax}{\lambda_{0}\mathrm{D}}+\frac{2\pi aX}{\lambda_{0}\mathrm{d}}\right)\right]\mathrm{d}X$. \\
Le calcul se poursuit comme suit :
\\
$I(x,X) = 2I_{0}+\frac{2I_{0}}{b}\times
\int_{-\frac{b}{2}}^{+\frac{b}{2}}\cos\left(\frac{2\pi
ax}{\lambda_{0}\mathrm{D}}+\frac{2\pi aX}{\lambda_{0}\mathrm{d}}\right)\mathrm{d}X$ ou encore : \\
$I(x,X) = 2I_{0}\left[1+\frac{\lambda_{0}\mathrm{d}}{2\pi ab}\left[\sin\left(\frac{2\pi
ax}{\lambda_{0}\mathrm{D}}+\frac{2\pi
aX}{\lambda_{0}\mathrm{d}}\right)\right]_{\mathrm{-\frac{b}}{2}}^{+\frac{b}{2}} = 2I_{0}\left[1+\frac{\lambda_{0}\mathrm{d}}{2\pi ab}\times \left[\sin\left(\frac{2\pi
ax}{\lambda_{0}\mathrm{D}}+\frac{\pi ab}{\lambda_{0}\mathrm{d}}\right) - \sin\left(\frac{2\pi
ax}{\lambda_{0}\mathrm{D}}-\frac{\pi ab}{\lambda_{0}\mathrm{d}}\right)\right]\right]$.
\\
Or, $\sin a - \sin b = 2\sin\left(\frac{a-b}{2}\right)\cos\left(\frac{a+b}{2}\right)$. Il
s'ensuit que : \\
$I(x,X) = 2I_{0}\left[1+\frac{\lambda_{0}\mathrm{d}}{\pi ab}\times \sin\left(\frac{\pi
ab}{\lambda_{0}\mathrm{d}}\right)\cos\left(\frac{2\pi ax}{\lambda_{0}\mathrm{D}}\right)\right]$. \\
On pose \mathcolorbox{gray!20}{V(b) = \frac{\lambda_{0}\mathrm{d}}{\pi a b}\times
\sin\left(\frac{\pi ab}{\lambda_{0}\mathrm{d}}\right)=sinc\left(\frac{\pi
ab}{\lambda_{0}\mathrm{d}}\right)}, la fonction de
visibilité. \\
Finalement, on obtient la formule de l'intensité lumineuse (ou de
l'éclairement) suivante :
\\
\centerline{\mathcolorbox{gray!20}{I(x,X) = 2I_{0}\left[1+V(b)\cos\left(\frac{2\pi ax}{\lambda_{0}\mathrm{D}}\right)\right]}}. \\
Il y a brouillage des franges lorsque la visibilité s'annule, soit
$\frac{\pi ab}{\lambda_{0}\mathrm{d}}= m \pi$ avec $m$ un entier naturel NON
NUL, d'où : $b_{m} = m\frac{\lambda_{0}\mathrm{d}}{a}$. \\
Mathématiquement, notre modèle prédit le brouillage des franges
toutes les fois où la largeur de la source primaire étendue est un
multiple entier de la quantité $\frac{\lambda_{0}\mathrm{d}}{a}$. \\
Physiquement parlant, dans le modèle des trains d'onde, la perte de
cohérence (ici spatiale) de la source primaire se manifeste dès que
la première annulation est réalisée, ce qui est acquis pour la
largeur de la source primaire valant : \mathcolorbox{gray!20}{b_{1} =
\frac{\lambda_{0}\mathrm{d}}{a}}. \\
Cette longueur caractéristique du
brouillage des franges est appelée longueur de cohérence spatiale de
la source.
\end{itemize}

\subsection*{4)  Critère semi-quantitatif de brouillage des franges (critère de cohérence spatiale)}
Appliquons le critère au cas de la source étendue
quasi-monochromatique.
\begin{itemize}
\item Pour la source ponctuelle $S'=O''$ en $X=0$, on peut écrire que :
$\delta(X=0) = p_{1}\lambda_{0}=\frac{ax}{\mathrm{D}}$.
\item Pour la source ponctuelle $S'=A$ en $X=\frac{b}{2}$, on peut écrire
que : $\delta\left(X=+\frac{b}{2}\right) =
p_{2}\lambda_{0}=\frac{ab}{2\mathrm{d}}+\frac{ax}{\mathrm{D}}$.
\end{itemize}
On en déduit en $M$ la différence de l'ordre d'interférences :
$\Delta p = p_{2}-p_{1} =
\frac{ab}{2\lambda_{0}\mathrm{d}}$. \\
D'après le critère, il y a brouillage des franges à la condition que
$\Delta p > \frac{1}{2}$, soit : $ \frac{ab}{2\lambda_{0}\mathrm{d}} >
\frac{1}{2}$ ou encore : \mathcolorbox{gray!20}{b
> \frac{\lambda_{0}\mathrm{d}}{a} = b_{1}}. \\
On retrouve par une approche semi-qualitative le résultat du calcul
complet mené dans le paragraphe IV.4.
\end{document}
