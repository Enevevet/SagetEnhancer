\documentclass{article}
\usepackage{geometry}
\geometry{a4paper,margin=2cm}

\usepackage{wrapfig}

\everymath{\displaystyle}
\renewcommand{\epsilon}{\varepsilon}
%\renewcommand{\epsilon}{\mathchar"122}

\usepackage{esvect}
\usepackage{wrapfig}
\usepackage{mathrsfs}

\usepackage{physics}

% (?<!\\Bigg)\)\_
% <(.*?)>
% \_\{(?=em)(.*?)\}
% \\fbox{\$\left((.|\n\right)*?)\$}
% : ?\\\\ ?\n\$\left((.(?!(\\mathcolorbox\right))|\n)*?)\$ ?\.? ?\n?\\\\

\usepackage{xcolor}
\usepackage{soul}
\newcommand{\mathcolorbox}[2]{\fcolorbox{black}{#1}{$#2$}}

\DeclareSymbolFont{legacyletters}{OML}{cmm}{m}{it}
\let\j\relax
\DeclareMathSymbol{\j}{\mathord}{legacyletters}{"7C}


\newcommand{\oneast}{\bigskip\par{\large\centerline{*\medskip}}\par}


\usepackage[T1]{fontenc}
%% \usepackage[french]{babel}
\usepackage{epsfig}
\usepackage{graphicx}
\usepackage{amsmath}
%\setlength{\mathindent}{0cm}
\usepackage{amsfonts}
\usepackage{amssymb}
\usepackage{float}
\usepackage{esint}
\usepackage{enumitem}
\usepackage{frcursive}
%\usepackage{fourier}
%% \usepackage{amsrefs}
\reversemarginpar
\newcommand{\asterism}{%
\leavevmode\marginpar{\makebox[10em][c]{$10^{500}$\makebox[1em][c]{%
\makebox[0pt][c]{\raisebox{-0.3ex}{\smash{$\star\star$}}}%
\makebox[0pt][c]{\raisebox{0.8ex}{\smash{$\star$}}}%
}}}}

\setlength\parindent{0pt}

\let\oldiint\iint
\renewcommand{\iint}{\oldiint\limits}

\let\oldiiint\iiint
\renewcommand{\iiint}{\oldiiint\limits}

\let\oldoint\oint
\renewcommand{\oint}{\oldoint\limits}

\let\oldoiint\oiint
\renewcommand{\oiint}{\oldoiint\limits}


\renewcommand\overrightarrow{\vv}
\let\oldexp\exp
\renewcommand{\exp}[1]{\oldexp\left(#1\right)}

\newcommand{\ext}{\text{ext}}
\newcommand{\cste}{\text{cste}}

%\renewcommand{\div}{\mathrm{div}}
\let\div\relax
\DeclareMathOperator{\div}{\mathrm{div}}
\let\rot\relax
\DeclareMathOperator{\rot}{\overrightarrow{\mathrm{rot}}}
\let\grad\relax
\DeclareMathOperator{\grad}{\overrightarrow{\mathrm{grad}}}


\title{\huge{\textbf{Cours EM5 bis : Induction électromagnétique}}}
\author{par Guillaume Saget, professeur de Sciences Physiques en MP, Lycée Champollion}
\date{}
\begin{document}
\maketitle

\begin{abstract}
Je propose ici le cours EM5 bis dédié au phénomène d'induction (au
sens de Neumann et au sens de Lorentz). Dans tout le corps de texte,
$\wedge$ désigne le produit vectoriel.
\end{abstract}


\section*{I. Induction de Neumann}
\subsection*{2. Champ électromoteur et force électromotrice de Neumann}
L'équation de Maxwell permettant d'expliquer le phéno-mène
d'induction est l'équation de Maxwell-Faraday (M.F.): \\
\centerline{\mathcolorbox{gray!20}{\rot \overrightarrow{E}= -\frac{\partial
\overrightarrow{B}}{\partial t}}}. \\
Physiquement, cette relation exprime que les variations temporelles
du champ magnétique sont sources d'un champ électrique dit induit.
\\
Comme $\overrightarrow{B}(M,t)$ dérive d'un potentiel vecteur,
l'équation (M.F.) devient : $\rot \overrightarrow{E}=
-\frac{\partial \rot \overrightarrow{A}}{\partial
t}$. \\
En intervertissant les opérateurs d'espace et de temps et en
exploitant la linéarité de l'opérateur rotationnel, il vient :
$\rot \left(\overrightarrow{E}+\frac{\partial
\overrightarrow{A}}{\partial t}\right) = \overrightarrow{0}$. \\
Il existe un champ scalaire appelé potentiel scalaire généralisé
$V(M,t)$ tel que : $\overrightarrow{E}+\frac{\partial
\overrightarrow{A}}{\partial t} = -\grad V$, ou
encore (en rajoutant les fioritures
d'usage) : \\
\centerline{\mathcolorbox{gray!20}{\overrightarrow{E}(M,t) = -\frac{\partial
\overrightarrow{A}}{\partial t}(M,t) -\grad V(M,t)}}. \\
On appelle champ électromoteur de Neumann la contribution du champ
électrique à circulation non conservative, i.e. :
\mathcolorbox{gray!20}{\overrightarrow{E}_{m}(M,t) = -\frac{\partial
\overrightarrow{A}}{\partial t}(M,t)}.


\begin{wrapfigure}{r}{0.25\textwidth}
\epsfig{file=Fig1.PNG,height=3cm, width=3.3cm}
\caption\protect{Orientation d'usage d'un contour $\Gamma$.}\label{Fig.1}
\end{wrapfigure}



Soit un circuit filiforme, fixe dont le contour $\left(\Gamma\right)$
a
été préalablement orienté (Cf. figure \ref{Fig.1}). \\
On appelle fem (induite) la circulation du champ électrique le long
du contour
$\Gamma$ : \\
\centerline{\mathcolorbox{gray!20}{e(t) = \oint_{M \in
\Gamma}\overrightarrow{E}(M,t).\overrightarrow{\mathrm{d}\ell}(M) = \oint_{M
\in \Gamma}\overrightarrow{E}_{m}(M,t).\overrightarrow{\mathrm{d}\ell}(M)}}.

\subsection*{3. Loi de Faraday}
Soit un circuit filiforme, fixe, indéformable dont le
contour $\left(\Gamma\right)$ a été préalablement orienté sur lequel s'appuie
une surface ouverte (Cf. \ref{Fig.1}). Considérons un intégrande
$\overrightarrow{\mathrm{d}S}(P)$ de cette surface orienté selon la règle
de la main droite. \\
Multiplions scalairement $\rot \overrightarrow{E}$
par $\overrightarrow{\mathrm{d}S}(P)$ puis intégrons sur toute la surface
ouverte $S$ ; en vertu de l'équation de Maxwell-Faraday, on a :
\begin{eqnarray}\label{Eq.1}
\iint_{P\in S}
\rot \overrightarrow{E}.\overrightarrow{\mathrm{d}S}(P) =
-\iint_{P\in S} \frac{\partial \overrightarrow{B}(P,t)}{\partial
t}.\overrightarrow{\mathrm{d}S}(P).
\end{eqnarray}
En vertu du théorème de Stokes-Ampère, l'équation (\ref{Eq.1})
s'écrit encore :
\begin{eqnarray}\label{Eq.2}
\oint_{M\in \Gamma}\overrightarrow{E}(M,t).\overrightarrow{\mathrm{d}\ell}(M)
= -\iint_{P\in S} \frac{\partial \overrightarrow{B}(P,t)}{\partial
t}.\overrightarrow{\mathrm{d}S}(P).
\end{eqnarray}
Or, l'élément de circuit est fixe, indéformable, dans le référentiel
d'étude : on peut intervertir l'opérateur $\frac{\partial}{\partial
t}$ et l'intégrale double. Il vient :
\begin{eqnarray}\label{Eq.3}
\oint_{M\in \Gamma}\overrightarrow{E}(M,t).\overrightarrow{\mathrm{d}\ell}(M)
= - \frac{\mathrm{d}}{\mathrm{d}t}\iint_{P \in
S}\overrightarrow{B}(P,t).\overrightarrow{\mathrm{d}S}(P).
\end{eqnarray}
Or, $\phi(t) = \iint_{P\in
S}\overrightarrow{B}(P,t).\overrightarrow{\mathrm{d}S}(P)$ est le flux du
champ magnétique à travers $(S)$ (rappelons que ce flux est
indépendant de la forme de la surface ouverte s'appuyant sur
$\Gamma$). On en déduit l'expression de la loi de Faraday : \\
\centerline{\mathcolorbox{gray!20}{e(t) = -\frac{\mathrm{d}\phi}{\mathrm{d}t}(t)}}.
\subsection*{5. Induction de Neumann dans un circuit filiforme}
\subsubsection*{b) Exemple d'application : spire circulaire plongée dans un
champ magnétique uniforme variant sinusoïdalement dans le temps}
\underline{Questions :}
\begin{enumerate}
\item Cf. figure \ref{Fig.2}.



\begin{wrapfigure}{r}{0.25\textwidth}
\epsfig{file=Fig2.PNG,height=3cm, width=5.3cm}
\caption\protect{A droite circuit électrique équivalent (vue du dessus). La
fem et l'intensité du courant induit sont orientés selon le sens
positif choisi. $R$ est la résistance électrique de la
spire.}\label{Fig.2}
\end{wrapfigure}




\item Le flux du champ magnétique est (la surface ouverte choisi est le disque de rayon $r$ délimité par $\Gamma$ ; d'après le sens positif
choisi, $\overrightarrow{\mathrm{d}S}=\mathrm{d}S\overrightarrow{u}_{z}$ ; le champ
magnétique est uniforme MAIS dépendant du temps): $\phi(t) =
\iint_{P\in S} \overrightarrow{B}.\overrightarrow{\mathrm{d}S}(P) = B_{0}\cos
\omega t \pi r^{2}$. \\
Conformément à la loi de Faraday, $e(t) = -\frac{\mathrm{d}\Phi}{\mathrm{d}t} =
B_{0}\omega \sin\omega t \pi r^{2}$, puis $i(t) = \frac{e}{R} =
\frac{B_{0}\omega \sin\omega t \pi r^{2}}{R}$.
\end{enumerate}

\subsection*{6. Auto-induction}
\subsubsection*{c) Coefficients de Neumann}
Considérons deux circuits filiformes $\mathcal{C}_{1}$ et
$\mathcal{C}_{2}$, fixes dans le référentiel d'étude dont les
contours ont préalablement été orientés (Cf. figure \ref{Fig.3}).


\begin{wrapfigure}{r}{0.25\textwidth}
\epsfig{file=Fig3.PNG,height=3cm, width=5.3cm}
\caption\protect{Orientation relative des deux circuits.}\label{Fig.3}
\end{wrapfigure}



Considérons deux surfaces ouvertes $S_{1}$ et $S_{2}$ s'appuyant sur
chacun des circuits (les intégrandes de surface sont orientés à
partir des orientations des
circuits via la sempiternelle règle de la main droite). \\
Ces circuits sont parcourus par des courants d'intensité $i_{1}(t)$
et $i_{2}(t)$.
Chacun des circuits créé un champ magnétique $\overrightarrow{B}_{i}$ avec $i=1,2$. \\
On introduit ici un coefficient caractéristique qui ne dépend que la
géométrie du circuit appelé coefficient d'inductance propre ou d'auto-induction $L_{i}$  positif défini pour chacun des circuits par : \\
\centerline{\mathcolorbox{gray!20}{\Phi_{\mathrm{1\rightarrow 1}}(t) = L_{1}i_{1}(t)}} et
\mathcolorbox{gray!20}{\Phi_{\mathrm{2\rightarrow 2}}(t) = L_{2}i_{2}(t)}. \\
$\Phi_{\mathrm{1\rightarrow 1}}(t)$ désigne le flux du champ magnétique
$\overrightarrow{B}_{1}$ à travers le circuit $\mathcal{C}_{1}$. \\
$\Phi_{\mathrm{2\rightarrow 2}}(t)$ désigne le flux du champ magnétique
$\overrightarrow{B}_{2}$ à travers le circuit $\mathcal{C}_{2}$. \\
Notons que ces définitions sont encore licites si les intensités des
courants sont constantes\footnote{En d'autres termes, on peut les
définir même si le phénomène d'induction n'existe pas ; si les
intensités sont constantes et les circuits fixes l'un par rapport à
l'autre, il n'existe pas d'induction (au sens de Neumann,
l'induction naît des variations du flux magnétique liées directement
à celles du champ magnétique. Au sens de Lorentz, l'induction naît
des variations du flux magnétique induites par le déplacement
relatif des circuits couplés).}. \\
On définit encore les coefficients d'inductance mutuelle
caractéristiques des interactions entre ces deux circuits (chacun
des champs magnétiques propres contribuent au flux du champ
magnétique de l'autre circuit) par : \\
\centerline{\mathcolorbox{gray!20}{\Phi_{\mathrm{1\rightarrow 2}}(t) = M_{\mathrm{12}}i_{1}(t)}} et
\mathcolorbox{gray!20}{\Phi_{\mathrm{2\rightarrow 1}}(t) = M_{\mathrm{21}}i_{2}(t)}. \\
$\Phi_{\mathrm{1\rightarrow 2}}(t)$ désigne le flux du champ magnétique
$\overrightarrow{B}_{1}$ à travers le circuit $\mathcal{C}_{2}$. \\
$\Phi_{\mathrm{2\rightarrow 1}}(t)$ désigne le flux du champ magnétique
$\overrightarrow{B}_{2}$ à travers le circuit $\mathcal{C}_{1}$. \\
Notons que ces définitions sont encore licites si les intensités des
courants sont constantes. \\
Contrairement aux coefficients $L_{1}$ et $L_{2}$, les coefficients
$M_{\mathrm{12}}$ et $M_{\mathrm{21}}$ dépendent de l'orientation relative des
circuits : \textbf{ces coefficients sont algébriques !} \\
Notons enfin qu'on peut montrer\footnote{J'ai renoncé à la
démonstration en 2014, i.e. au changement de programme...} (à l'aide
des potentiels vecteurs) que les coefficients $M_{\mathrm{12}}$ et $M_{\mathrm{21}}$
sont égaux. Par la suite,
on posera : \mathcolorbox{gray!20}{M_{\mathrm{12}} =M_{\mathrm{21}} =M}. \\
Les coefficients $L_{1}$, $L_{2}$ et $M$ sont appelés coefficients
de Neumann; leur unité SI est le henry (symbole : H).

\subsubsection*{d) Exercice d'application}
\underline{Prérequis :} le flux est additif : si la
distribution de courant est un circuit formé par $N$ spires alors le
flux à travers le circuit est $N$ fois le flux à travers une spire :
$\Phi(t) = N\Phi_{\mathrm{1\,spire}}$.

\begin{itemize}
\item On assimile le long solénoïde (noté solénoïde "1") à un
solénoïde de longueur infinie pour le calcul du champ magnétique
intérieur qui vaut donc (Cf. cours EM3) :
\mathcolorbox{gray!20}{\overrightarrow{B}_{1} = \mu_{0}\frac{ N
I}{\ell}\overrightarrow{u}_{z}}. \\
En orientant positivement chaque contour selon l'intensité du
courant, chaque intégrande de surface ouverte délimitée par un
contour est orienté via la règle de la main droite comme étant :
$\overrightarrow{\mathrm{d}S} =\mathrm{d}S\overrightarrow{u}_{z}$. \\
Le calcul du flux de $\overrightarrow{B}_{1}$ à travers une spire
est  ($\overrightarrow{B}_{1}$ est uniforme et $S$ est la surface
d'une spire) : $\Phi_{\mathrm{1\longrightarrow 1 spire}} =
\mu_{0}\frac{NSI}{\ell}$. Or, $\Phi_{1}(t) = N\Phi_{\mathrm{1\,spire}}$, donc
le flux du champ magnétique créé par le solénoïde à travers lui-même
est : $\Phi_{1} = \mu_{0}\frac{N^{2}SI}{\ell}$.\\
Par définition de l'inductance propre, on écrit : $\Phi_{1}=
\mu_{0}\frac{N^{2}SI}{\ell} = L_{1}I$, donc : \mathcolorbox{gray!20}{L_{1} =
\mu_{0}\frac{N^{2}S}{\ell}}, grandeur qui ne dépend que de la
géométrie du solénoïde. \\
A.N.: \underline{$L_{1} = 10,0$ mH}.
\item Orientons positivement chaque contour du petit solé-noïde (nommé solénoïde 2) selon l'intensité du courant $I_{2}$,
chaque intégrande de surface ouverte délimitée par un contour est
orienté via la règle de la main droite comme étant :
$\overrightarrow{\mathrm{d}S}_{2} =\mathrm{d}S_{2}\overrightarrow{u}_{\mathrm{z2}}$. \\
Pour obtenir le coefficient d'inductance mutuelle, il faut jouer sur
la propriété que $M_{\mathrm{12}} = M_{\mathrm{12}}$. \\
En effet, pour le petit solénoïde, difficile d'utiliser le modèle du
solénoïde de longueur infinie...En outre le calcul du flux du champ
créé par ce dernier à travers le grand solénoïde paraît délicat. En
revanche, il est simple de calcul le flux du champ
$\overrightarrow{B}_{1}$ à travers le petit solénoïde. \\
Explicitons ce dernier calcul. \\
Le calcul du flux de $\overrightarrow{B}_{1}$ à travers une spire du
petit solénoïde est  ($\overrightarrow{B}_{1}$ est uniforme et
$S_{2}$ est la surface d'une spire) : $\Phi_{\mathrm{1\longrightarrow 1
spire\, \mathrm{d}e\,2}} = \mu_{0}\frac{NI}{\ell}S_{2}\cos\theta$. Or,
$\Phi_{\mathrm{1\longrightarrow 2}}(t) = N_{2}\Phi_{\mathrm{1\longrightarrow 1
spire\, \mathrm{d}e\,2}}$, donc le flux du champ magnétique créé par le grand
solénoïde à travers le petit solénoïde est : $\Phi_{\mathrm{1\longrightarrow 2}}(t) = \mu_{0}\frac{NN_{2}S_{2}I}{\ell}\cos\theta$.\\
Par définition du coefficient d'inductance mutuelle, on écrit :
$\Phi_{\mathrm{1\longrightarrow 2}}(t)
=\mu_{0}\frac{NN_{2}S_{2}I}{\ell}\cos\theta = MI$, donc : \mathcolorbox{gray!20}{M =
\mu_{0}\frac{NN_{2}S_{2}}{\ell}\cos\theta}, grandeur qui ne dépend
que de la géométrie des solénoïdes et de leur position
relative. \\
En effet, si $\theta \in \left[0,\frac{\pi}{2}\right] \cup
\left[\frac{3\pi}{2},2\pi\right]$, $M\geq 0$ ($\cos\theta \geq 0$) ; si $\theta
\in \left[\frac{\pi}{2},\frac{3\pi}{2}\right]$, $M\leq 0$ ($\cos\theta \leq
0$).
\end{itemize}

\subsubsection*{f) Utilisation des coefficients de Neumann}
Les coefficients de Neumann sont, par exemple, utilisés en
électrocinétique lorsqu'on traite de circuits électriques couplés,
fixes dans le référentiel du laboratoire (Cf. paragraphe suivant).

\subsection*{7. Energie magnétique}
\subsubsection*{b) Energie magnétique de deux circuits couplés}
Les variations temporelles du flux du champ magnétique
$\Phi_{i}(t)$ (avec $i=1,2$) à travers la surface $S_{i}$ (donc à
travers le circuit lui-même à l'origine du champ magnétique
$\overrightarrow{B}_{i}$) sont sources d'une fem induite dite
d'auto-induction. \\
On applique la loi de Faraday au niveau de chacune des deux bobines
sièges d'un phénomène d'induction (couplage avec le champ de l'autre
bobine) et d'auto-induction (induction à travers le circuit
inducteur lui-même). On a : $e_{1} = -\frac{\mathrm{d}\Phi_{1}}{\mathrm{d}t}$ avec
$\Phi_{1} = \Phi_{\mathrm{1 \rightarrow 1}}+ \Phi_{\mathrm{2 \rightarrow 1}}$ avec
$\Phi_{\mathrm{1 \rightarrow 1}} = L_{1}i_{1}$ et $\Phi_{\mathrm{2 \rightarrow 1}} =
Mi_{2}$ ; $e_{2} = -\frac{\mathrm{d}\Phi_{2}}{\mathrm{d}t}$ avec $\Phi_{2} = \Phi_{\mathrm{2
\rightarrow 2}}+ \Phi_{\mathrm{1 \rightarrow 2}}$ avec $\Phi_{\mathrm{2 \rightarrow 2}}
= L_{2}i_{2}$ et $\Phi_{\mathrm{1 \rightarrow 2}} = Mi_{1}$.
\\
On a besoin du caractère rigide des bobines (sinon $L_{1}$, $L_{2}$
et $M$ dépendraient du temps) pour écrire : $e_{1} =
-L_{1}\frac{\mathrm{d}i_{1}}{\mathrm{d}t}-M\frac{\mathrm{d}i_{2}}{\mathrm{d}t}$ et $e_{2} =
-L_{2}\frac{\mathrm{d}i_{2}}{\mathrm{d}t}-M\frac{\mathrm{d}i_{1}}{\mathrm{d}t}$. \\



\begin{wrapfigure}{r}{0.25\textwidth}
\epsfig{file=Fig6.PNG,height=3.0cm, width=8.3cm}
\caption\protect{Schéma électrique équivalent à la figure 2 du cours. Les
bobines sont remplacées par les fem induites.}\label{Fig.6}
\end{wrapfigure}



La loi des mailles donne (Cf. figure \ref{Fig.6}) : $E_{1} =
-e_{1}+r_{1}i_{1} =
L_{1}\frac{\mathrm{d}i_{1}}{\mathrm{d}t}+M\frac{\mathrm{d}i_{2}}{\mathrm{d}t}+r_{1}i_{1}$ et $E_{2} =
-e_{2}+r_{2}i_{2} =
L_{2}\frac{\mathrm{d}i_{2}}{\mathrm{d}t}+M\frac{\mathrm{d}i_{1}}{\mathrm{d}t}+r_{2}i_{2}$. Faisons la
somme de $E_{1}i_{1}+E_{2}i_{2}$ ; on a : \\
$E_{1}i_{1}+E_{2}i_{2} =
L_{1}i_{1}\frac{\mathrm{d}i_{1}}{\mathrm{d}t}+Mi_{1}\frac{\mathrm{d}i_{2}}{\mathrm{d}t}+r_{1}i_{1}^{2}+
L_{2}i_{2}\frac{\mathrm{d}i_{2}}{\mathrm{d}t}+Mi_{2}\frac{\mathrm{d}i_{1}}{\mathrm{d}t}+r_{2}i_{2}^{2}$,
ou encore : \\
$E_{1}i_{1}+E_{2}i_{2} =
\frac{\mathrm{d}}{\mathrm{d}t}\left[\frac{1}{2}L_{1}i_{1}^{2}+\frac{1}{2}L_{2}i_{2}^{2}
+Mi_{1}i_{2}\right]+ r_{1}i_{1}^{2}+r_{2}i_{2}^{2}$. \\
On pose (à une constante inessentielle près), l'énergie potentielle
des deux circuits couplés : \\
\centerline{\mathcolorbox{gray!20}{E_{p} = \frac{1}{2}L_{1}i_{1}^{2}+\frac{1}{2}L_{2}i_{2}^{2} +
Mi_{1}i_{2}}}.
\\
Au final, $E_{1}i_{1}+E_{2}i_{2} = \frac{\mathrm{d}E_{p}}{\mathrm{d}t}+r_{1}i_{1}^{2}
+r_{2}i_{2}^{2}$. \\
L'interprétation est la suivante : la puissance fournie par les
générateurs (on travaille en convention générateur) est égale à la
puissance magnétique emmagasinée par les deux circuits couplés et à
la puissance reçue (on travaille avec ces dipôles en convention
récepteur) par les conducteurs ohmiques de résistances $r_{1}$ et
$r_{2}$ (et dissipée par effet Joule). \\
\underline{Remarque :} $E_{p}$ est une grandeur
positive\footnote{Comme le polynôme du second degré est positif, son
discriminant est négatif ou nul. En fait $E_{p}$ est l'énergie
magnétique introduite lors du bilan de Poynting (Cf. EM5) ; aussi,
on a avec les notations de EM5 : $E_{p} = E_{m} = \iiint_{M\in
\mathrm{Espace}}w_{m}(M,t)\mathrm{d}V(M)$ avec $w_{m} = \frac{B^{2}}{2\mu_{0}}$, donc
$E_{p} = \iiint_{M \in \left(\mathcal{D}\right)} \frac{B^{2}}{2\mu_{0}}\mathrm{d}V(M)
\geq 0$.}. On exploite cette propriété en considérant que $E_{p}$
est, par exemple, fonction de la variable $i_{1}$ (à $i_{2}$ fixé ;
on pourrait aussi dire que $E_{p}$ est une fonction de la variable
$i_{2}$ à $i_{1}$ fixé, $E_{p}$ présentant une "symétrie" sur les
indices "1" et "2"). On écrit donc ($a = \frac{1}{2}L_{1}$,
$b=Mi_{2}$ et $c=\frac{1}{2}L_{2}i_{2}^{2}$) $E_{p} =
\frac{1}{2}L_{1}i_{1}^{2}+\left[Mi_{2}\right]i_{1}+ \frac{1}{2}L_{2}i_{2}^{2}$.
\\
Le discriminant est : $\Delta = M^{2}i_{2}^{2}-L_{1}L_{2}i_{2}^{2}
\leq 0$. Comme $i_{2} \neq 0$, il vient : $M^{2} \leq L_{1}L_{2}$,
d'où : \mathcolorbox{gray!20}{|M| \leq \sqrt{L_{1}L_{2}}}. \\
On pose $k = \frac{|M|}{\sqrt{L_{1}L_{2}}}$ avec $0 \leq k \leq 1$.
\\
Si $k\ll 1$, on parle de couplage faible ; en revanche, si $k \approx
1$, on dit que le couplage entre les deux circuits est fort. \\
\underline{Astuce de calcul:} S'il n'existe qu'un seul circuit
(disons le circuit 1), on peut calculer $L_{1}$ à partir de
l'égalité suivante : $E_{p} = \iiint_{M\in
\left(\mathcal{D}_{1}\right)}\frac{B_{1}^{2}}{2\mu_{0}}(M,t)\mathrm{d}V(M)=\frac{1}{2}L_{1}i_{1}^{2}(t)$.

\section*{II. Induction de Lorentz}
\subsection*{2. Changement de référentiel galiléen en électromagnétisme}
\subsubsection*{a) Position du problème : " un couac " de la mécanique
newtonienne}
Soit (R) le référentiel du laboratoire galiléen et (R') le
référentiel lié au circuit. (R') est en translation rectiligne
uniforme par rapport à
(R) avec la vitesse d'entraînement $\overrightarrow{v}_{e}=\overrightarrow{\mathrm{cste}}$. \\
Considérons une charge au repos dans (R').\\
Quelque soit le point de vue de l'observateur (lié à (R) ou à (R')),
il existe un champ magnétique et électrique créés par la charge. \\
Or, en mécanique classique, on fait le raisonnement suivant:
\begin{itemize}
\item Dans (R), la charge $q$ créé un champ électrique et magnétique
(car elle est en mouvement donc est source d'un champ magnétique).
\item Dans (R'), la charge étant au repos, elle créé un champ
électrostatique $\overrightarrow{E}'$ mais le champ magnétique est
nul : $\overrightarrow{B'}  = \overrightarrow{0}$.
\end{itemize}

\subsubsection*{c) Vérification dans la limite de la transformation
galiléenne (mécanique newtonienne) pour l'ARQS magnétique}
La démonstration (magnifique !) se fait dans le cadre de
la relativité restreinte (la démonstration sera mise sur le site
dans un nouvel onglet
"Additifs" ; on se place dans le domaine faiblement relativiste). \\
On se propose de vérifier les expressions proposées dans l'ARQS
magnétique (qui est la limite licite dans le cadre de l'induction
électromagnétique). On postule l'invariance de la charge et de la
force par
changement de référentiel. \\
On écrit :
\begin{eqnarray}\label{Eq.10}
\overrightarrow{F}=q\overrightarrow{E}+q\overrightarrow{v}\wedge \overrightarrow{B}
=
\overrightarrow{F'}=q\overrightarrow{E'}+q\overrightarrow{v'}\wedge \overrightarrow{B'}.
\end{eqnarray}
Le théorème de composition newtonienne des vitesses entre (R) et
(R') donne (Cf. M1 avec ici $\overrightarrow{v} =
\overrightarrow{v}_{\mathrm{/R}}(M)$, $\overrightarrow{v'} =
\overrightarrow{v}_{\mathrm{/R'}}(M)$, $\overrightarrow{v}_{e} =
\overrightarrow{v}_{\mathrm{/R}}(O')$) : $\overrightarrow{v} =
\overrightarrow{v'}+\overrightarrow{v}_{e}$. \\
Ici $\overrightarrow{v'} = \overrightarrow{0}$. \\
Dans l'ARQS magnétique : $\overrightarrow{B} = \overrightarrow{B'}$
; il vient : $q\overrightarrow{E}+q\overrightarrow{v}_{e} \wedge \overrightarrow{B} = q\overrightarrow{E'}$. \\
Dans l'ARQS magnétique, on retiendra donc que : \\
\centerline{\mathcolorbox{gray!20}{\overrightarrow{E'}=\overrightarrow{E}+\overrightarrow{v}_{e}
\wedge \overrightarrow{B}}}.

\subsection*{3. Champ électromoteur de Lorentz et force électromotrice
de Lorentz} \textbf{ a)  Définitions :} \\
On définit (au sens de Lorentz) le champ électromoteur par
: \mathcolorbox{gray!20}{\overrightarrow{E}_{m}=\overrightarrow{v}_{e}\wedge \overrightarrow{B}}.
\\
La définition de la fem induite est inchangée par rapport à celle introduite au sens de Neumann : \\
\centerline{\mathcolorbox{gray!20}{e=\oint_{M\in
\Gamma}\overrightarrow{E}_{m}.\overrightarrow{\mathrm{d}\ell}(M)}}.

\subsection*{4. Loi de Faraday}
La loi de Faraday demeure licite dans le cadre de
l'induction de Lorentz à la condition que le circuit (constitue le
référentiel relatif (R') nommé "référentiel circuitocentrique") soit
rigide (=indéformable). Aussi, on écrit : \\
\centerline{\mathcolorbox{gray!20}{e=\oint_{M \in
\Gamma}\overrightarrow{E}_{m}.\overrightarrow{\mathrm{d}\ell}(M) =
-\frac{\mathrm{d}\Phi}{\mathrm{d}t}}}.

\subsection*{5)  Rappels sur la force de Laplace}
Soit un circuit filiforme parcouru par un courant
d'intensité $i$ (stationnaire ou dépendant du temps), plongé dans un
champ
magnétique $\overrightarrow{B}(M,t)$. \\
Le circuit subit des actions mécaniques dont la résultante (dite des
actions de Laplace) ou plus simplistement "force de Laplace" a pour
expression : \\
\centerline{\mathcolorbox{gray!20}{\overrightarrow{F}_{L} = \oint_{M \in
\Gamma}i\overrightarrow{\mathrm{d}\ell}(M) \wedge \overrightarrow{B}(M,t)}}.
\\
\textbf{Dans cette expression, le déplacement élémentaire le long du
circuit est de même sens que l'intensité du courant.}

\subsection*{7)  Un exemple classique : les rails de Laplace (principe des
générateurs)} A $t = 0$, un opérateur communique une
vitesse $\overrightarrow{v}_{0}=v_{0}\overrightarrow{u}_{x}$ à la
barre (métallique) qui est ensuite lâchée. On néglige les
frottements solides exercés par le rail sur le barreau ainsi que les
forces de frottement fluide dûes à
l'air. \\
La barre est plongée dans un champ magnétique stationnaire et
uniforme (donc champ constant) orienté selon un axe vertical
ascendant : $\overrightarrow{B}  =B\overrightarrow{u}_{z}$.
\begin{itemize}
\item \underline{Approche qualitative :} Du fait que le barreau est
en déplacement relatif, on rentre dans le cadre de l'induction de
Lorentz. Le déplacement de ce dernier produit une variation du flux
du champ magnétique à travers le circuit formé par les rails et le
barreau, donc est le siège d'une fem induite, laquelle est source
d'un courant induit $i$ : on réalise de façon "primitive" la
conversion d'un travail mécanique apporté par un opérateur en un
travail électrique fourni par un générateur : c'est le principe de
la transduction électromécanique (conversion ici d'une grandeur
mécanique en une grandeur électrique).



\begin{wrapfigure}{r}{0.25\textwidth}
\epsfig{file=Fig4.PNG,height=3.5cm, width=8.3cm}
\caption\protect{Expérience des rails de Laplace et schéma électrique
équivalent à droite. La fem induite est localisée au niveau de la
barre.}\label{Fig.4}
\end{wrapfigure}



\item \underline{Approche quantitative :} Le calcul de la force de
Laplace s'exerçant sur le circuit de contour $\Gamma$ se ramène à
celle s'exerçant sur le seul barreau : \\
$\overrightarrow{F} = \oint_{M\in
\Gamma}i(t)\overrightarrow{\mathrm{d}\ell}(M)\wedge \overrightarrow{B} =
i\int_{A}^{C}\left[\mathrm{d}y\overrightarrow{u}_{y}\right] \times
\left[B\overrightarrow{u}_{z}\right] = i\ell B \overrightarrow{u}_{x}$. \\
Calculons la fem induite. \textit{A priori} le circuit se déformant,
on ne peut pas appliquer la loi de Faraday pour calculer la fem
induite. \\
On la calcule directement par circulation du champ électromoteur de
Lorentz\footnote{Il y a ici un petit "hic" ... Lequel ?} :
$\overrightarrow{E}_{m} = \overrightarrow{v}\wedge \overrightarrow{B} = -vB\overrightarrow{u}_{y}$ (avec
$\overrightarrow{v}=\overrightarrow{v}_{e}$ la vitesse du barreau
qui est encore la vitesse d'entraînement du référentiel relatif,
référentiel lié au barreau qui est en mouvement de translation par
rapport à R). On écrit : \\
$e(t) = \oint_{M \Gamma}
\overrightarrow{E}_{m}(t).\overrightarrow{\mathrm{d}\ell}(M)=
\int_{A}^{C}\left[-vB\overrightarrow{u}_{y}\right].\left[\mathrm{d}y\overrightarrow{u}_{y}\right]$
. \\
Il vient : \mathcolorbox{gray!20}{e(t)=-v(t)B\ell}. \\
Tentons maintenant de calculer la fem induite\footnote{"Passage en
force" !} par la loi de Faraday : entre $t$ et $t+\mathrm{d}t$ la surface du
circuit a varié de $\mathrm{d}S(t) = \ell \mathrm{d}x(t) = \ell v(t)\mathrm{d}t$. Le flux a
varié de ($\overrightarrow{\mathrm{d}S}$ est dirigé comme
$\overrightarrow{B}$) $\mathrm{d}\Phi =
\overrightarrow{B}.\overrightarrow{\mathrm{d}S} = vB\ell \mathrm{d}t$. Il vient :
\mathcolorbox{gray!20}{e(t) =
-v(t)B\ell}. \\
Diantre ! Pourquoi la loi de Faraday s'applique-t-elle ici ?\\
En fait, le circuit est formé par les rails indéformables et ne sont
là que pour refermer le circuit électriquement ; de surcroît,
l'élément en mouvement relatif est le barreau qui est indéformable :
la loi de Faraday s'applique donc (voir TD avec l'exemple de la roue
de Barlow
pour avoir un cas de non-application de la loi de Faraday). \\
On réalise ensuite un circuit électrique équivalent (Cf. figure
\ref{Fig.4}) sur lequel on applique la loi des mailles qui
trivialement donne : $e(t) = Ri$ avec $R$ la résistance électrique
du circuit et $i$ l'intensité du courant induit (orienté comme $e$ ;
c'est une grandeur algébrique).
Il vient : \mathcolorbox{gray!20}{i(t) = -\frac{vB\ell}{R}}. \\
La force de Laplace prend la forme : \\
\centerline{\mathcolorbox{gray!20}{\overrightarrow{F} =i\ell B\overrightarrow{u}_{x} =
-\frac{\ell^{2}B^{2}}{R}v\overrightarrow{u}_{x} = -\alpha
\overrightarrow{v}}} \\
avec $\alpha = \frac{\ell^{2}B^{2}}{R}$ : la
force de Laplace est équivalente à une force de frottement fluide.
\\
Enfin, on applique à la barre en mouvement de translation (tous les
points ont la même vitesse) le théorème de la résultante cinétique
ou théorène du centre d'inertie
($\overrightarrow{P}+\overrightarrow{R}_{N} = \overrightarrow{0}$)
projeté selon l'axe (Ox) ($m$ est la masse de la barre) :
$m\frac{\mathrm{d}v}{\mathrm{d}t} = i(t)\ell B = -\frac{\ell^{2}B^{2}}{R}v(t)$. \\
On pose $\tau = \frac{mR}{\ell^{2}B^{2}}$ une grandeur homogène à un
temps, caractéristique des variations temporelles de la vitesse $v$
de la barre. Il vient : \mathcolorbox{gray!20}{\frac{\mathrm{d}v}{\mathrm{d}t}+\frac{v}{\tau} = 0}.
\\
A l'aide de la CI, on calcule la loi d'évolution de la vitesse de la
barre : \mathcolorbox{gray!20}{v(t) = v_{0}e^{-\frac{t}{\tau}}}.
\end{itemize}

\subsection*{8. Bilan de conversion électromécanique}
On vérifie le bilan de puissance électromécanique à partir de
l'exemple des rails de Laplace. \\
La puissance fournie par la fem induite (on travaille en convention
générateur) est : $\mathcal{P}_{\mathrm{fem}} =
e(t)i(t) = \frac{e^{2}}{R} = \frac{v^{2}\ell^{2}B^{2}}{R}$. \\
La puissance fournie par les actions de Laplace est :
$\mathcal{P}_{\mathrm{Laplace}} = \overrightarrow{F}_{L}.\overrightarrow{v} =
i\ell B v = \frac{e}{R}B\ell v = -\frac{v^{2}\ell^{2} B^{2}}{R}$. \\
On a bien : $\mathcal{P}_{\mathrm{fem}}+\mathcal{P}_{\mathrm{Laplace}} = 0$.

\subsection*{10. Principe des moteurs électriques}
\subsubsection*{a) Moteurs linéaires (machines à courant continu) à partir des rails de Laplace}
On garde les mêmes
orientations qu'au paragraphe 8. \\
Le calcul de la fem induite et de la force de Laplace demeurent
inchangées : $e = -B\ell v$ et $\overrightarrow{F}_{L} = iB\ell
\overrightarrow{u}_{x}$ mais $i$ n'est plus\footnote{Pourquoi?} \textit{stricto sensu} l'intensité du courant induit seul. \\
On fait un schéma électrique équivalent (Cf. figure \ref{Fig.5}).



\begin{wrapfigure}{r}{0.25\textwidth}
\epsfig{file=Fig5.PNG,height=3.2cm, width=8.3cm}
\caption\protect{Expérience des rails de Laplace et schéma électrique
équivalent à droite. La fem induite est localisée au niveau de la
barre.}\label{Fig.5}
\end{wrapfigure}



Par application de la loi des mailles, on a: $E = -e+Ri$ (on
désignait jadis par fcem\footnote{Pour force contre-électromotrice
... Peut-être la dénomination a-t-elle survécu en SI ?} la quantité
$e' = -e$). Il
vient : $i = \frac{E+e}{R} = \frac{E-B\ell v}{R}$. \\
L'application du TCI à la barre projeté selon (Ox) donne l'expression (d'une laideur sans nom) suivante : \\
$m\frac{\mathrm{d}v}{\mathrm{d}t} = i\ell B = \frac{B \ell E}{R} -
\frac{B^{2}\ell^{2}}{R}v$. \\
On divise par $m$, on introduit le
temps caractéristique $\tau = \frac{mR}{B^{2}\ell^{2}}$ et on
aboutit à
l'équation : $\frac{\mathrm{d}v}{\mathrm{d}t}+\frac{v}{\tau} = \frac{B\ell}{mR}E$. \\
La résolution de cette équation différentielle linéaire du premier
ordre à coefficients constants donne : $v(t) = \tau\frac{B\ell}{mR}E
+Ae^{-\frac{t}{\tau}}$. On pose $v_{\mathrm{lim}} = \frac{E}{B\ell}$. \\
Avec la CI, $v(t=0) = 0$, il vient : \mathcolorbox{gray!20}{v(t) =
v_{\mathrm{lim}}\left[1-e^{-\frac{t}{\tau}}\right]}. \\
Lorsque $t \rightarrow +\infty$, $v \rightarrow v_{\mathrm{lim}}$ ; une fois
le régime permanent atteint, le barreau est en mouvement de
translation rectiligne uniforme. \\
\underline{Un peu de physique avec "les mains" :} \\
Initialement, le générateur produit un courant d'intensité $I_{0} =
\frac{E}{R}$ dans le circuit. Le barreau initialement au repos se
met en mouvement sous l'effet de la force de Laplace "initiale"
d'expression : $\overrightarrow{F}_{L} (t=0) =
I_{0}B \ell\overrightarrow{u}_{x}$. \\
On est bien en présence d'une transduction électromécanique mais
c'est "l'effet inverse" de celui mettant en évidence les rails de
Laplace fonctionnant comme un générateur : ici on convertit un
travail électrique apporté par un générateur en un travail mécanique
qui met en mouvement un barreau : c'est le principe d'un moteur.\\
Reprenons notre raisonnement ... Mais le déplacement de la barre
produit des variations du flux du champ magnétique à travers le
circuit initiant l'apparition d'une fem induite et d'un courant
induit d'intensité : $i_{\mathrm{in\mathrm{d}uit}} =
\frac{e}{R} = -\frac{vB\ell}{R}$. \\
D'après la loi de modération de Lenz, $i_{\mathrm{in\mathrm{d}uit}}$ tend à s'opposer
aux causes qui lui ont donné naissance : $\mathrm{d}\Phi = B\ell v \mathrm{d}t >0$,
donc $e(t) < 0$, donc $i_{\mathrm{in\mathrm{d}uit}} < 0$ : $i_{\mathrm{in\mathrm{d}uit}}$ s'oppose ici à
$I_{0}$. Ceci perdure jusqu'à une vitesse acquise $v_{\mathrm{lim}}$ par le
barreau pour laquelle $i_{\mathrm{in\mathrm{d}uit}}$ compense $I_{0}$ : on a alors
$I_{0}-\frac{v_{\mathrm{lim}}B\ell}{R}=0$, soit : $v_{\mathrm{lim}} = \frac{RI_{0}}{B
\ell} = \frac{E}{B \ell}$.\\

\subsubsection*{b) Machines rotatives à courant continu}
On véifie ici le bilan de transduction électromécanique ; on calcule
$\mathcal{P}_{\mathrm{fem}} = e i = -\Phi_{0}\omega i$ et la puissance
fournie par les actions de Laplace\footnote{Rappels de MPSI et/ou de
SI : pour calculer le travail d'une force on écrit: $\delta W
=\overrightarrow{F}.\overrightarrow{\mathrm{d}\ell}$ ; pour des actions qui
sont réductibles à un couple $\overrightarrow{\Gamma}$ (Cf. mon
cours M3 de l'ancien programme ... snif!) s'exerçant sur un système
matériel en mouvement à la vitesse angulaire
$\overrightarrow{\omega}$ autour d'un axe de rotation, on écrit :
$\delta W = \mathcal{P}\mathrm{d}t =
\overrightarrow{\Gamma}.\overrightarrow{\omega}\mathrm{d}t$.}
$\mathcal{P}_{\mathrm{Laplace}} =
\overrightarrow{\Gamma}.\overrightarrow{\omega} = \Gamma \omega =
\Phi_{0}\omega i$. \\
Au final, $\mathcal{P}_{\mathrm{fem}}+\mathcal{P}_{\mathrm{Laplace}}=0$.

\subsection*{III. Cadre d'étude plus général}
\subsection*{1. Position du problème et expression de la f.e.m. induite
dans le cas général}
Dans le cas général, on définit la fem à partir du travail de la
charge (subissant la force de Lorentz) le long du contour $\Gamma$ :
$e(t) = \frac{\delta W\left(\overrightarrow{F}\right)}{q}$, soit
($\overrightarrow{v}_{e} =
\overrightarrow{v}_{\mathrm{/R}}(M)=\overrightarrow{v}$ la vitesse de la
charge $q$ dans (R) ; la charge est au repos dans le référentiel
relatif (R') : $\overrightarrow{v}'=\overrightarrow{0}$):
\\
\centerline{\mathcolorbox{gray!20}{e(t) = \oint_{M\in
\Gamma}\overrightarrow{E}.\overrightarrow{\mathrm{d}\ell}(M)+\oint_{M\in
\Gamma}\left[\overrightarrow{v}_{e}\wedge \overrightarrow{B}\right].\overrightarrow{\mathrm{d}\ell}(M)}}. \\
A l'aide du théorème de Stokes (en utilisant les conventions
d'orientation habituelles, $S$ surface ouverte s'appuyant sur le
contour $\Gamma$), on a : $e(t) = \iint_{P\in
S}\rot \overrightarrow{E}.\overrightarrow{\mathrm{d}S}(P)+\oint_{M\in
\Gamma}\left[\overrightarrow{v}_{e}\wedge \overrightarrow{B}\right].\overrightarrow{\mathrm{d}\ell}(M)$. \\
Enfin, à l'aide de l'équation de Maxwell-Faraday, on obtient
l'expression la plus générale de la fem induite :
\begin{eqnarray}\label{Eq.60}
e(t) = -\iint_{P\in S}\frac{\partial\overrightarrow{B}}{\partial
t}(P,t).\overrightarrow{\mathrm{d}S}(P) +\oint_{M\in
\Gamma}\left[\overrightarrow{v}_{e}\wedge \overrightarrow{B}\right].\overrightarrow{\mathrm{d}\ell}(M).
\end{eqnarray}

\subsection*{2. Cas limites des inductions de Neumann et Lorentz comme
cas particuliers}\begin{itemize}
\item Si $\overrightarrow{v}_{e} = \overrightarrow{0}$ : (R') est au
repos par rapport à (R), on est dans le cadre de l'induction de
Neumann ; l'équation (\ref{Eq.60}) devient : $e(t) = -\iint_{P\in
S}\frac{\partial\overrightarrow{B}}{\partial
t}(P,t).\overrightarrow{\mathrm{d}S}(P)= -\frac{\mathrm{d}}{\mathrm{d}t}\Phi(t)$ avec $\Phi(t)
= \iint_{P\in S}\overrightarrow{B}(P,t).\overrightarrow{\mathrm{d}S}(P)$. La
loi de Faraday est licite.
\item Si $\frac{\partial \overrightarrow{B}}{\partial t} =
\overrightarrow{0}$, on est dans le cadre de l'induction de Lorentz;
l'équation (\ref{Eq.60}) devient : $e(t) = \oint_{M\in
\Gamma}\left[\overrightarrow{v}_{e}\wedge \overrightarrow{B}\right].\overrightarrow{\mathrm{d}\ell}(M)$.
\end{itemize}

\subsection*{3. Loi de Faraday dans les circuits de constitution rigide en translation rectiligne uniforme par rapport (R)}
On peut montrer\footnote{Démonstration sur le site dans les additifs
?} (Cf. annexe 6 du cours EM5 bis pour le schéma) que si le circuit
$\Gamma$ est rigide durant son déplacement, on a : $e(t) =
-\iint_{P\in S}\frac{\partial\overrightarrow{B}}{\partial
t}(P,t).\overrightarrow{\mathrm{d}S}(P) +\oint_{M\in
\Gamma}\left[\overrightarrow{v}_{e}\wedge \overrightarrow{B}\right].\overrightarrow{\mathrm{d}\ell}(M) =
-\frac{\mathrm{d}}{\mathrm{d}t}\iint_{P\in
S}\overrightarrow{B}(P,t).\overrightarrow{\mathrm{d}S}(P) = -\frac{\mathrm{d}\Phi}{\mathrm{d}t}(t)$ : la loi de Faraday demeure licite.

\subsection*{4. Exit la loi de Faraday dans les circuits de constitution
variable ...}
Si le circuit en déplacement n'est pas rigide, il faut conserver
l'équation (\ref{Eq.60}) pour calculer la fem induite.
\end{document}
