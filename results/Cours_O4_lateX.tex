\documentclass{article}
\usepackage{geometry}
\geometry{a4paper,margin=2cm}

\usepackage{wrapfig}

\everymath{\displaystyle}
\renewcommand{\epsilon}{\varepsilon}
%\renewcommand{\epsilon}{\mathchar"122}

\usepackage{esvect}
\usepackage{wrapfig}
\usepackage{mathrsfs}

\usepackage{physics}

% (?<!\\Bigg)\)\_
% <(.*?)>
% \_\{(?=em)(.*?)\}
% \\fbox{\$\left((.|\n\right)*?)\$}
% : ?\\\\ ?\n\$\left((.(?!(\\mathcolorbox\right))|\n)*?)\$ ?\.? ?\n?\\\\

\usepackage{xcolor}
\usepackage{soul}
\newcommand{\mathcolorbox}[2]{\fcolorbox{black}{#1}{$#2$}}

\DeclareSymbolFont{legacyletters}{OML}{cmm}{m}{it}
\let\j\relax
\DeclareMathSymbol{\j}{\mathord}{legacyletters}{"7C}


\newcommand{\oneast}{\bigskip\par{\large\centerline{*\medskip}}\par}


\usepackage[T1]{fontenc}
%% \usepackage[french]{babel}
\usepackage{epsfig}
\usepackage{graphicx}
\usepackage{amsmath}
%\setlength{\mathindent}{0cm}
\usepackage{amsfonts}
\usepackage{amssymb}
\usepackage{float}
\usepackage{esint}
\usepackage{enumitem}
\usepackage{frcursive}
%\usepackage{fourier}
%% \usepackage{amsrefs}
\reversemarginpar
\newcommand{\asterism}{%
\leavevmode\marginpar{\makebox[10em][c]{$10^{500}$\makebox[1em][c]{%
\makebox[0pt][c]{\raisebox{-0.3ex}{\smash{$\star\star$}}}%
\makebox[0pt][c]{\raisebox{0.8ex}{\smash{$\star$}}}%
}}}}

\setlength\parindent{0pt}

\let\oldiint\iint
\renewcommand{\iint}{\oldiint\limits}

\let\oldiiint\iiint
\renewcommand{\iiint}{\oldiiint\limits}

\let\oldoint\oint
\renewcommand{\oint}{\oldoint\limits}

\let\oldoiint\oiint
\renewcommand{\oiint}{\oldoiint\limits}


\renewcommand\overrightarrow{\vv}
\let\oldexp\exp
\renewcommand{\exp}[1]{\oldexp\left(#1\right)}

\newcommand{\ext}{\text{ext}}
\newcommand{\cste}{\text{cste}}

%\renewcommand{\div}{\mathrm{div}}
\let\div\relax
\DeclareMathOperator{\div}{\mathrm{div}}
\let\rot\relax
\DeclareMathOperator{\rot}{\overrightarrow{\mathrm{rot}}}
\let\grad\relax
\DeclareMathOperator{\grad}{\overrightarrow{\mathrm{grad}}}

\title{\huge{\textbf{Interférences à $N$ ondes (O4)}}}
\author{par Guillaume Saget, professeur de Sciences Physiques en MP, Lycée Champollion}
\date{}

\begin{document}
\maketitle


\begin{abstract}
Je propose ici le cours O4 qui vient apporter les réponses au
support de cours éponyme.
\end{abstract}


\section*{I.  Le réseau plan en transmission : Généralités}
\subsection*{3.  Nombre de traits par unité de longueur}
Exemple de calcul du nombre de traits par millimètre. Pour
un réseau plan dont $a$ = $ 20 \,\mu$m $=$ $2,0\times 10^{-5}$ m, $n
\approx \frac{1}{n} = 2,0\times 10^{4}$ traits/m ou encore $n =
2,0\times 10^{2}$ traits/mm.

\section*{II. Etude théorique}
\subsection*{2.  Recherche des maxima d'interférences (interférences à N ondes)}
Calculons la différence de marche entre deux rayons
diffractés consécutifs diffractés dans la direction $\theta_{p}'$. \\
La source ponctuelle est quasi-monochomatique dans la direction
incidente $\theta_{i}$, les rayons émergents diffractent tous
parallèlement dans la direction $\theta'_{p}$. Tous les émergents se
coupent en un point $M_{\mathrm{\infty}}$ à l'infini. \\
Calculons la différence de marche entre les émergents $"i+1"$ et
$"i"$ : $\delta_{\mathrm{i+1/i}} =
\left(S_{\mathrm{\infty}}O_{\mathrm{i+1}}KM_{\mathrm{\infty}}\right)-\left(S_{\mathrm{\infty}} H O_{i}M_{\mathrm{\infty}}\right)$. \\
Attention ! Ici les distances sont algébriques. \\
On écrit : $\left(S_{\mathrm{\infty}}O_{\mathrm{i+1}}KM_{\mathrm{\infty}}\right) =
\left(S_{\mathrm{\infty}}O_{\mathrm{i+1}}\right)+\left(O_{\mathrm{i+1}}K\right)+\left(KM_{\mathrm{\infty}}\right)$. \\
Or, on travaille dans l'air assimilé au vide, donc : \\
$\left(S_{\mathrm{\infty}}O_{\mathrm{i+1}}KM_{\mathrm{\infty}}\right) =
\overline{S_{\mathrm{\infty}}O_{\mathrm{i+1}}}+\overline{O_{\mathrm{i+1}}K}      +
\overline{KM_{\mathrm{\infty}}}$.
\\
De même, $\left(S_{\mathrm{\infty}} H O_{i}M_{\mathrm{\infty}}\right) =
\overline{S_{\mathrm{\infty}}H}+\overline{HO_{i}}+\overline{O_{i}M_{\mathrm{\infty}}}$.
\\
$S_{\mathrm{\infty}}$ est une source ponctuelle à l'infini ; loin de
$S_{\mathrm{\infty}}$ les surfaces d'ondes sont des plans d'ondes ; ainsi
$\left(\Sigma\right)$ est l'un de ces plans ; comme $H$ et $O_{\mathrm{i+1}} \in
\left(\Sigma\right)$, $\overline{S_{\mathrm{\infty}}O_{\mathrm{i+1}}} = \overline{S_{\mathrm{\infty}}H}$.
\\
De même, en vertu du principe du retour inverse de la lumière,
$M_{\mathrm{\infty}}$ est une source ponctuelle. Par le même raisonnement, on
montre que $\left(\Sigma'\right)$ est un plan d'ondes issu de $M_{\mathrm{\infty}}$.
Comme $K$ et $O_{i} \in \left(\Sigma'\right)$, $\overline{O_{i}M_{\mathrm{\infty}}} =
\overline{KM_{\mathrm{\infty}}}$.
\\
La différence de marche devient : $\delta_{\mathrm{i+1/i}} =
\overline{S_{\mathrm{\infty}}O_{\mathrm{i+1}}}+\overline{O_{\mathrm{i+1}}K}+\overline{KM_{\mathrm{\infty}}}
-
\left[\overline{S_{\mathrm{\infty}}H}+\overline{HO_{i}}+\overline{O_{i}M_{\mathrm{\infty}}}\right]$
$=$ $\overline{O_{\mathrm{i+1}}K}-\overline{HO_{i}}$. \\
Or, dans le triangle $O_{i}O_{\mathrm{i+1}}K$, $\sin \theta' =
\frac{\overline{O_{\mathrm{i+1}}K}}{a}$, d'où : $\overline{O_{\mathrm{i+1}}K} = a\sin
\theta'$. \\
De même, dans le triangle $O_{i}O_{\mathrm{i+1}}H$, $\sin \theta =
\frac{\overline{HO_{i}}}{a}$, d'où : $\overline{HO_{i}} = a\sin
\theta$. \\
On en déduit que : $\delta_{\mathrm{i+1/i}} = a\left[\sin\theta'-\sin\theta\right]$. \\
On constate que $\forall i \in \left[1,N-1\right]$, $\delta_{\mathrm{i+1/i}}$ est
indépendante de $"i"$ : elle est la même entre deux rayons
consécutifs. On la notera par la suite $\delta$. \\
On cherche à exprimer la condition d'interférences constructives à
$N$ ondes dans la direction diffractée repérée par l'angle
$\theta'$. \\
On admet ici (la démonstration sera faite au paragraphe $II.4.$) que
la condition d'interférences est constructive entre deux rayons
consécutifs "i" et "i+1" si la différence de marche est un multiple
entier\footnote{Le résultat est analogue au phénomène
d'interférences constructives à 2 ondes.} $p$ (appelé ordre du
réseau) de la longueur d'onde $\lambda_{0}$ (on travaille dans l'air
assimilé au vide), soit : $\delta_{\mathrm{i+1/i}} = p\lambda_{0}$ avec $p
\in \mathbb{Z}$. \\
La condition d'interférences constructives à $N$ ondes est :
$\forall i \in \left[1,N-1\right]$, $\delta_{\mathrm{i+1/i}} = p\lambda_{0}$, ou encore
puisque $\delta_{\mathrm{i+1/i}}$ est indépendante de $"i"$ : $\delta =
p\lambda_{0}$. \\
On en déduit alors la relation fondamentale des réseaux plans en
transmission qui exprime la condition d'interférences constructives
à $N$ ondes dans la direction diffractée $\theta'$ :
$a\left[\sin\theta'-\sin\theta\right]
= p\lambda_{0}$. \\
Or, l'angle $\theta'$ dépend de l'ordre $p$ ; aussi, on préfère
remplacer $\theta'$ par $\theta'_{p}$. \\
On retiendra donc la formule suivante : \\
\centerline{\mathcolorbox{gray!20}{a\left[\sin\theta'_{p}-\sin\theta\right] = p\lambda_{0}}}.

\subsection*{3.  Minimum de déviation}
Dans l'ordre $p$, la déviation est définie par : $\mathrm{D}_{p} =
\theta'_{p}-\theta$. \\
Cherchons la condition pour laquelle la déviation est minimale. Du
fait de la relation fondamentale des réseaux plans, il existe une
relation entre $\theta'_{p}$ et $\theta$ : ces deux variables ne
sont pas indépendantes : $\theta'_{p} = f\left(\theta\right)$. \\
Calculons la dérivée de $\mathrm{D}$ par rapport à la variable $\theta$ :
$\mathrm{D}_{p}'
= \frac{\mathrm{d}\mathrm{D}_{p}}{\mathrm{d}\theta} = \frac{\mathrm{d}\theta'_{p}}{\mathrm{d}\theta} - 1$. \\
Différentions la relation fondamentale des réseaux plans (à $a$,
$\lambda_{0}$ fixées) dans l'ordre $p$ :
$a\left[\cos\theta'_{p}\mathrm{d}\theta'_{p} - \cos\theta \mathrm{d}\theta\right] = 0$, d'où :
$\frac{\mathrm{d}\theta'_{p}}{\mathrm{d}\theta} = \frac{\cos\theta}{\cos\theta'_{p}}$
\\
Il vient : $\frac{\mathrm{d}\mathrm{D}_{p}}{\mathrm{d}\theta} =
\frac{\cos\theta}{\cos\theta'_{p}} - 1$. On écrit ensuite la
condition d'extrémalité de $\mathrm{D}_{p}$, à savoir :
$\frac{\mathrm{d}\mathrm{D}_{p}}{\mathrm{d}\theta} = 0$, ce qui conduit à :
$\frac{\cos\theta}{\cos\theta'_{p}} = 1$, ou encore :
\begin{eqnarray}\label{EQ.1}
\cos\theta = \cos \theta'_{p}
\end{eqnarray}
Dans l'ordre $p$, la déviation est extrémale à la condition que les
angles $\theta$ et $\theta'_{p}$ satisfassent à la condition
(\ref{EQ.1}). \\
Les solutions sont : $\theta'_{p} = \theta$ ou $\theta'_{p} =
-\theta$. \\
La première solution conduit à une trivialité pour tout ordre $p$, à
savoir la nullité de la déviation extrémale $\mathrm{D}_{p}$ ; cette solution
est à écarter car elle est inessentielle en spectroscopie (on
cherche à déterminer une déviation minimale non nulle afin d'avoir
des résultats exploitables). \\
La solution à conserver est : \mathcolorbox{gray!20}{\theta'_{p} = -\theta}. \\
Pour ne pas confondre l'angle $\theta$ correspondant à la déviation
extrémale avec un angle $\theta$ quelconque, écrivons que l'angle
$\theta$ correspondant à cette déviation est : $\theta_{0}$ et
$\theta'_{\mathrm{p,0}}$ pour l'angle associé au rayon diffracté. \\
On admet que lorsque $\theta = \theta_{0}$ ($\theta'_{p} =
\theta'_{\mathrm{p,0}}$), la déviation est en fait minimale. On la notera :
$\mathrm{D}_{\mathrm{m,p}}$.
\\
Dans l'ordre $p$, à la déviation minimale ($\mathrm{D}_{\mathrm{m,p}} =
\theta'_{\mathrm{p,0}}-\theta_{0} = 2\theta'_{\mathrm{p,0}}$) , la relation
fondamentale des réseaux plans conduit à :
$a\left[\sin\theta'_{\mathrm{p,0}}-\sin\theta_{0}\right] = p\lambda_{0}$ ou encore :
$2a\sin\theta'_{\mathrm{p,0}} = p\lambda_{0}$. \\
Or, $\theta'_{\mathrm{p,0}} = \frac{\mathrm{D}_{\mathrm{m,p}}}{2}$. \\
Finalement, \mathcolorbox{gray!20}{\sin\left(\frac{\mathrm{D}_{\mathrm{m,p}}}{2}\right) =
\frac{p\lambda_{0}}{2a}} : la détermination de la déviation
minimale dans l'ordre $p$ (Cf. TP sur le goniomètre à réseau pour la
réalisation expérimentale) permet d'accéder à la longueur d'onde
$\lambda_{0}$.

\subsection*{4.  Recherche de la fonction éclairement liée au
phénomène d'interférences} Dans le modèle scalaire de la
lumière, notons $A_{0}$ l'amplitude de la source ponctuelle
$S_{\mathrm{\infty}}$. On adopte un modèle simplifié où l'amplitude de toutes
les ondes vaut $A_{0}$.
\begin{itemize}
\item Le champ vibratoire de la source ponctuelle à la date $t$ est :
$\underline{a}\left(S_{\mathrm{\infty}},t\right) = A_{0}e^{i\omega t}$.
\item A la date $t$, le champ vibratoire de l'onde diffractée dans la
direction $\theta'_{p}$ ayant passé par $O_{1}$ en $M_{\mathrm{\infty}}$ est
: $\underline{a}_{1}\left(M_{\mathrm{\infty}},t\right) = A_{0}e^{i\left[\omega
t-k_{0}\left(S_{\mathrm{\infty}}O_{1}M_{\mathrm{\infty}}\right)\right]}$,
\item celui de l'onde diffractée dans la direction $\theta'_{p}$ ayant
passé par $O_{2}$ en $M_{\mathrm{\infty}}$ est :
$\underline{a}_{2}\left(M_{\mathrm{\infty}},t\right) = A_{0}e^{i\left[\omega
t-k_{0}\left(S_{\mathrm{\infty}}O_{2}M_{\mathrm{\infty}}\right)\right]}$,
\item celui de l'onde diffractée dans la direction $\theta'_{p}$ ayant
passé par $O_{3}$ en $M_{\mathrm{\infty}}$ est :
$\underline{a}_{2}\left(M_{\mathrm{\infty}},t\right) = A_{0}e^{i\left[\omega
t-k_{0}\left(S_{\mathrm{\infty}}O_{3}M_{\mathrm{\infty}}\right)\right]}$, ...
\item celui de l'onde diffractée dans la direction $\theta'_{p}$ ayant
passé par $O_{N}$ en $M_{\mathrm{\infty}}$ est :
$\underline{a}_{N}\left(M_{\mathrm{\infty}},t\right) = A_{0}e^{i\left[\omega
t-k_{0}\left(S_{\mathrm{\infty}}O_{N}M_{\mathrm{\infty}}\right)\right]}$.
\end{itemize}
D'après le principe de Huygens-Fresnel, les $N$ motifs atteints par
la lumière issue de $S_{\mathrm{\infty}}$ se comportent comme des sources
secondaires, synchrones et cohérentes entre elles : les champs
vibratoires s'ajoutent en $M_{\mathrm{\infty}}$ (on peut se passer du
principe de Huygens-Fresnel en considérant que les $N$ motifs sont
des sources secondaires construites par division du front d'onde de
l'onde primaire issue de $S_{\mathrm{\infty}}$). \\
On a alors : $\underline{a}\left(M_{\mathrm{\infty}},t\right) =
\sum_{j=1}^{N}\underline{a}_{j}\left(M_{\mathrm{\infty}},t\right)$. \\
En mettant le terme $A_{0}e^{i\left[\omega t
-k_{0}\left(S_{\mathrm{\infty}}O_{1}M_{\mathrm{\infty}}\right)\right]}$ en facteur, on obtient : \\
$\underline{a}\left(M_{\mathrm{\infty}},t\right) = A_{0}e^{i\left[\omega t
-k_{0}\left(S_{\mathrm{\infty}}O_{1}M_{\mathrm{\infty}}\right)\right]}\left[1 +
e^{-ik_{0}\left[\left(S_{\mathrm{\infty}}O_{2}M_{\mathrm{\infty}}\right)-\left(S_{\mathrm{\infty}}O_{1}M_{\mathrm{\infty}}\right)\right]}$
$+$
$e^{-ik_{0}\left[\left(S_{\mathrm{\infty}}O_{3}M_{\mathrm{\infty}}\right)-\left(S_{\mathrm{\infty}}O_{1}M_{\mathrm{\infty}}\right)\right]}$
$+$ $...$
$+e^{-ik_{0}\left[\left(S_{\mathrm{\infty}}O_{N}M_{\mathrm{\infty}}\right)-\left(S_{\mathrm{\infty}}O_{1}M_{\mathrm{\infty}}\right)\right]}\right]$.
\\
Or, $\left(S_{\mathrm{\infty}}O_{2}M_{\mathrm{\infty}}\right)-\left(S_{\mathrm{\infty}}O_{1}M_{\mathrm{\infty}}\right) =
\delta$, $\left(S_{\mathrm{\infty}}O_{3}M_{\mathrm{\infty}}\right)-\left(S_{\mathrm{\infty}}O_{1}M_{\mathrm{\infty}}\right)  =
\left(S_{\mathrm{\infty}}O_{3}M_{\mathrm{\infty}}\right)-\left(S_{\mathrm{\infty}}O_{2}M_{\mathrm{\infty}}\right)+\left(S_{\mathrm{\infty}}O_{2}M_{\mathrm{\infty}}\right)-\left(S_{\mathrm{\infty}}O_{1}M_{\mathrm{\infty}}\right)
= 2\delta$, ...,
$\left(S_{\mathrm{\infty}}O_{N}M_{\mathrm{\infty}}\right)-\left(S_{\mathrm{\infty}}O_{1}M_{\mathrm{\infty}}\right)  =
(N-1)\delta$. \\
On en déduit que : $\underline{a}\left(M_{\mathrm{\infty}},t\right) = A_{0}e^{i\left[\omega t
-k_{0}\left(S_{\mathrm{\infty}}O_{1}M_{\mathrm{\infty}}\right)\right]}\left[1+e^{-ik_{0}\delta}+
e^{-2ik_{0}\delta}+...+e^{-i(N-1)k_{0}\delta}\right]$. \\
On est en présence d'une suite géométrique de raison $q =
e^{-ik_{0}\delta} = e^{-i\varphi}$ ; d'où :
$\underline{a}\left(M_{\mathrm{\infty}},t\right) = A_{0}e^{i\left[\omega t
-k_{0}\left(S_{\mathrm{\infty}}O_{1}M_{\mathrm{\infty}}\right)\right]}\left[1+q+q^{2}+...q^{N-1}\right]$. \\
D'où : $\underline{a}\left(M_{\mathrm{\infty}},t\right) = A_{0}e^{i\left[\omega t
-k_{0}\left(S_{\mathrm{\infty}}O_{1}M_{\mathrm{\infty}}\right)\right]}\frac{1-q^{N}}{1-q}$, ou encore :
$\underline{a}\left(M_{\mathrm{\infty}},t\right) = A_{0}e^{i\left[\omega t
-k_{0}\left(S_{\mathrm{\infty}}O_{1}M_{\mathrm{\infty}}\right)\right]}\frac{1-e^{-iN\varphi}}{1-e^{-i\varphi}}$.
\\
On factorise au numérateur par $e^{-i\frac{N\varphi}{2}}$ : \\
$\underline{a}\left(M_{\mathrm{\infty}},t\right) = A_{0}e^{i\left[\omega t
-k_{0}\left(S_{\mathrm{\infty}}O_{1}M_{\mathrm{\infty}}\right)\right]}\times e^{-i\frac{N\varphi}{2}}
\frac{e^{i\frac{N\varphi}{2}}-e^{-i\frac{N\varphi}{2}}}{1-e^{-i\varphi}}$.
\\
On factorise au dénominateur par $e^{-i\frac{\varphi}{2}}$ : \\
$\underline{a}\left(M_{\mathrm{\infty}},t\right) = A_{0}e^{i\left[\omega t
-k_{0}\left(S_{\mathrm{\infty}}O_{1}M_{\mathrm{\infty}}\right)\right]}\times
e^{-i\frac{(N-1)\varphi}{2}}
\frac{e^{i\frac{N\varphi}{2}}-e^{-i\frac{N\varphi}{2}}}{e^{i\frac{\varphi}{2}}-e^{-i\frac{\varphi}{2}}}$.
\\
En introduisant les fonctions sinus \textit{ad hoc} : \\
\centerline{\mathcolorbox{gray!20}{\underline{a}\left(M_{\mathrm{\infty}},t\right) = A_{0}e^{i\left[\omega t
-k_{0}\left(S_{\mathrm{\infty}}O_{1}M_{\mathrm{\infty}}\right)\right]}\times
e^{-i\frac{(N-1)\varphi}{2}}
\frac{\sin\left(\frac{N\varphi}{2}\right)}{\sin\left(\frac{\varphi}{2}\right)}}}. \\
L'éclairement en $M_{\mathrm{\infty}}$ est défini par :
$\mathcal{E}\left(M_{\mathrm{\infty}}\right) =
\underline{a}\left(M_{\mathrm{\infty}},t\right)\underline{a}^{\ast}\left(M_{\mathrm{\infty}},t\right)$
($\mathcal{E}_{0}=A_{0}^{2}$ l'éclairement de la source ponctuelle =
éclairement d'un des motifs). \\
Finalement, \mathcolorbox{gray!20}{\mathcal{E}\left(M_{\mathrm{\infty}}\right) =
\mathcal{E}_{\mathrm{int}}\left(M_{\mathrm{\infty}}\right) =
\mathcal{E}_{0}\biggl\left(\frac{\sin(\frac{N\varphi}{2}\right)}{\sin\left(\frac{\varphi}{2}\right)}\biggr)^{2}}.
\\
Les maxima de la fonction éclairement s'obtiennent en cherchant les
zéros du dénominateur, à savoir : $\sin^{2}\left(\frac{\varphi}{2}\right) = 0$,
d'où : $\frac{\varphi}{2} = m\pi$ avec $p \in \mathbb{Z}$, ou encore
: $\varphi = k_{0}\delta = \frac{2\pi}{\lambda_{0}}\delta = 2p\pi$.
On en déduit que : $\delta = a\left[\sin\theta'_{p}-\theta\right] =
p\lambda_{0}$. \\
Physiquement, les maxima de la fonction éclairement correspondent à
la condition d'interférences constructives dans la direction
diffractée $\theta'_{p}$. \\
\underline{Remarque : } Lorsque $\varphi = 2p\pi$, il n'y a pas
divergence de l'éclairement ; en effet, un équivalent au voisinage
de $0$ (modulo $2\pi$) est : $\mathcal{E}\left(M_{\mathrm{\infty}}\right) \sim
\mathcal{E}_{0}\biggl\left(\frac{\frac{N\varphi}{2}}{\frac{\varphi}{2}}\biggr\right)^{2}
= N^{2}\mathcal{E}_{0}$.

\subsection*{5.  Pouvoir de résolution (acronyme P.R.) d'un réseau
plan}
\subsubsection*{c.  Calcul du P.R. noté $R$}
\begin{itemize}
\item Raisonnons dans l'ordre $p$, pour un même angle d'incidence $\theta$ pour les radiations à $\lambda$ et à
$\lambda+\Delta \lambda$. Nous allons utiliser le critère de Rayleigh en raisonnant en termes de variations angulaires (plus simple à expliciter ici). \\
Pour la radiation à $\lambda$, la phase est : $\varphi =
\frac{2\pi}{\lambda}a\left[\sin\theta'_{p}-\sin\theta\right]$.
\\
En faisant varier $\theta'_{p}$ de façon élémentaire toutes choses
égales par ailleurs, la phase varie également de façon
infinitésimale. \\
La variation élémentaire de phase s'écrit alors : $\mathrm{d} \varphi =
\frac{2\pi}{\lambda}a \mathrm{d}\left[\sin\theta'_{p}\right] = \frac{2\pi}{\lambda}a \cos\theta'_{p}\mathrm{d}\theta'_{p}$. \\
En termes de variations finies, cette relation devient : $\Delta
\varphi = \frac{2\pi}{\lambda}a \Delta[\sin\theta'_{p}\right]$. Ce
déphasage est égal à la demi-largeur angulaire du pic d'éclairement
de cette radiation à la condition que\footnote{Les positions des
minima sont données à partir de $\sin \left(\frac{N\varphi}{2}\right) = 0$,
soit $\varphi = \frac{2m\pi}{N}$ avec $m \in \left[1,...,N-1\right]$ pour
l'ordre $p=0$, $m \in \left[pN+1,...,(p+1)N-1\right]$ pour l'ordre $p$. $m$ ne
peut pas être nul, ni égal à $N$, ..., à $pN$, ...  car on
retomberait alors sur la recherche des maxima principaux de la
fonction éclairement. La demi-largeur angulaire du pic d'éclairement
dans l'ordre $p$ vaut : $\Delta \varphi _{\mathrm{1/2}} =
\frac{2\pi[pN+1\right]}{N} - 2p\pi = \frac{2\pi}{N}$.} : $\Delta \varphi =
\Delta \varphi_{\mathrm{1/2}} = \frac{2\pi}{N}$. Il vient :
$\frac{2\pi}{\lambda}a \Delta[\sin\theta'_{p}\right] = \frac{2\pi}{N}$, ou
encore :
$\Delta[\sin\theta'_{p}\right] = \frac{\lambda}{aN}$. \\
Cette relation correspond à la demi-largeur angulaire du pic
d'éclairement de la radiation de longueur d'onde $\lambda$ à l'ordre
$p$. Aussi, on la note $\Delta[\sin\theta'_{p}\right]_{\mathrm{1/2}}$.
\item Par ailleurs, le maximum d'éclairement dans l'ordre $p$ correspond à la relation
fondamentale des réseaux plans : \\
pour la radiation à $\lambda$, on écrit : $a\left[\sin\theta'_{p}-\sin\theta\right] = p\lambda$. \\
pour la radiation à $\lambda+\Delta\lambda$, on écrit : $a\left[\sin\theta''_{p}-\sin\theta\right] = p\left[\lambda+\Delta \lambda\right]$. \\
On en déduit par différence de ces deux relations (on utilise les
valeurs absolues de façon à travailler avec des grandeurs positives,
les variations finies étant des grandeurs positives par
construction) : $a|\sin\theta''_{p}-\sin\theta'_{p}|=a\Delta[
\sin\theta''_{p}\right] = |p|\Delta \lambda$ ou encore : $\Delta
\left[\sin\theta''_{p}\right] = \frac{|p|\Delta \lambda}{a}$.
\item D'après le critère de Rayleigh, les deux radiations sont
séparables si : $\Delta
\left[\sin\theta''_{p}\right]=\frac{|p|\Delta\lambda}{a} \geq
\Delta[\sin\theta'_{p}\right]_{\mathrm{1/2}}= \frac{\lambda}{aN}$.
\item A la limite de résolution, on a : $\Delta \left[\sin\theta''_{p}\right]=\frac{|p|\Delta\lambda}{a}
= \Delta[\sin\theta'_{p}\right]_{\mathrm{1/2}}= \frac{\lambda}{aN}$, soit :
\mathcolorbox{gray!20}{R = \frac{\lambda}{\Delta \lambda} = |p|N}.
\end{itemize}
La résolution sera d'autant meilleure qu'on travaillera avec un
réseau comportant un plus grand nombre de traits.
\\
A.N. Pour $\lambda = 600$ nm, pour un réseau ayant typiquement un
pouvoir de résolution de $6,0\times 10^{3}$, dans l'ordre $p=1$, la
plus petite variation de longueur d'onde séparable est :
\underline{$\Delta \lambda = \frac{600}{6,0\times 10^{3}} = 0,1$
nm}.

\subsection*{6.  Dispersion angulaire d'un réseau
plan} Pour un ordre $p$ donné (à $\theta$ fixé),
différentions la relation fondamentale des réseaux plans en
transmission : $a\left[\sin\theta'_{p}-\sin\theta\right]=p\lambda_{0}$ :
$a\cos\theta'_{p}\mathrm{d}\theta'_{p}=p\mathrm{d}\lambda_{0}$. On en déduit que :
$\mathrm{D}_{\mathrm{ang}} = |\frac{\mathrm{d}\theta'_{p}}{\mathrm{d}\lambda_{0}}| =
\frac{|p|}{a|\cos\theta'_{p}|}$. \\
On constate que la dispersion est d'autant plus grande que l'ordre
du réseau est élevé et le pas du réseau faible.












































































































































\end{document}
