\documentclass{article}
\usepackage{geometry}
\geometry{a4paper,margin=2cm}

\usepackage{wrapfig}

\everymath{\displaystyle}
\renewcommand{\epsilon}{\varepsilon}
%\renewcommand{\epsilon}{\mathchar"122}

\usepackage{esvect}
\usepackage{wrapfig}
\usepackage{mathrsfs}

\usepackage{physics}

% (?<!\\Bigg)\)\_
% <(.*?)>
% \_\{(?=em)(.*?)\}
% \\fbox{\$\left((.|\n\right)*?)\$}
% : ?\\\\ ?\n\$\left((.(?!(\\mathcolorbox\right))|\n)*?)\$ ?\.? ?\n?\\\\

\usepackage{xcolor}
\usepackage{soul}
\newcommand{\mathcolorbox}[2]{\fcolorbox{black}{#1}{$#2$}}

\DeclareSymbolFont{legacyletters}{OML}{cmm}{m}{it}
\let\j\relax
\DeclareMathSymbol{\j}{\mathord}{legacyletters}{"7C}


\newcommand{\oneast}{\bigskip\par{\large\centerline{*\medskip}}\par}


\usepackage[T1]{fontenc}
%% \usepackage[french]{babel}
\usepackage{epsfig}
\usepackage{graphicx}
\usepackage{amsmath}
%\setlength{\mathindent}{0cm}
\usepackage{amsfonts}
\usepackage{amssymb}
\usepackage{float}
\usepackage{esint}
\usepackage{enumitem}
\usepackage{frcursive}
%\usepackage{fourier}
%% \usepackage{amsrefs}
\reversemarginpar
\newcommand{\asterism}{%
\leavevmode\marginpar{\makebox[10em][c]{$10^{500}$\makebox[1em][c]{%
\makebox[0pt][c]{\raisebox{-0.3ex}{\smash{$\star\star$}}}%
\makebox[0pt][c]{\raisebox{0.8ex}{\smash{$\star$}}}%
}}}}

\setlength\parindent{0pt}

\let\oldiint\iint
\renewcommand{\iint}{\oldiint\limits}

\let\oldiiint\iiint
\renewcommand{\iiint}{\oldiiint\limits}

\let\oldoint\oint
\renewcommand{\oint}{\oldoint\limits}

\let\oldoiint\oiint
\renewcommand{\oiint}{\oldoiint\limits}


\renewcommand\overrightarrow{\vv}
\let\oldexp\exp
\renewcommand{\exp}[1]{\oldexp\left(#1\right)}

\newcommand{\ext}{\text{ext}}
\newcommand{\cste}{\text{cste}}

%\renewcommand{\div}{\mathrm{div}}
\let\div\relax
\DeclareMathOperator{\div}{\mathrm{div}}
\let\rot\relax
\DeclareMathOperator{\rot}{\overrightarrow{\mathrm{rot}}}
\let\grad\relax
\DeclareMathOperator{\grad}{\overrightarrow{\mathrm{grad}}}

\title{\huge{\textbf{Compléments : Interféromètres de Fabry-Pérot et de Mach-Zehnder (O5)}}}
\author{par Guillaume Saget, professeur de Sciences Physiques en MP, Lycée Champollion}
\date{}

\begin{document}
\maketitle


\begin{abstract}
Je propose un complément dédié à deux systèmes interférométriques
diviseurs d'amplitude : le mirifique interféromètre de Fabry-Pérot
(mettant en jeu des interférences à $N$ ondes) et l'interféromètre
de Mach-Zehnder (mettant en jeu des interférences à deux ondes).
\end{abstract}


\section*{I. L'interféromètre de Fabry-Pérot}
\subsection*{1) Préliminaire}
\subsection*{a) Lame à faces parallèles d'épaisseur $e$ constante}
Travaillons sur les différentes interfaces :
\begin{itemize}
\item \underline{Interface 1/2 (interface air/verre) :} \\
Coefficient de réflexion en amplitude (sous incidence normale) :
$r_{1} =\frac{n_{1}-n_{2}}{n_{1}+n_{2}}$. \\
Coefficient de transmission sous incidence normale : $t_{1}
=\frac{2n_{1}}{n_{1}+n_{2}}$. Coefficient de réflexion en énergie :
\mathcolorbox{gray!20}{R_{1} = \left(\frac{n_{1}-n_{2}}{n_{1}+n_{2}}\right)^{2}}.
\\
Coefficient de transmission en énergie ($t_{1}$ réel ici):
\mathcolorbox{gray!20}{T_{1} = t_{1}^{2}\frac{n_{2}}{n_{1}} =
\frac{4n_{1}n_{2}}{\left(n_{1}+n_{2}\right)^{2}}}. \\
On vérifie que : $R_{1}+T_{1} = 1$ (loi de conservation de
l'énergie).
\item \underline{Interface 2/3 : (interface verre/air)} ; ici $n_{3} = n_{1}$ ; coefficient de réflexion en amplitude (sous incidence normale) :
$r_{2} =\frac{n_{2}-n_{1}}{n_{1}+n_{2}}$. \\
Coefficient de transmission sous incidence normale : $t_{2}
=\frac{2n_{2}}{n_{1}+n_{2}}$. Coefficient de réflexion en énergie :
\mathcolorbox{gray!20}{R_{2} = \left(\frac{n_{2}-n_{1}}{n_{1}+n_{2}}\right)^{2}}.
\\
Coefficient de transmission en énergie ($t_{2}$ réel ici):
\mathcolorbox{gray!20}{T_{2} = t_{2}^{2}\frac{n_{3}}{n_{2}} =
\frac{4n_{1}n_{2}}{\left(n_{1}+n_{2}\right)^{2}}}. \\
On vérifie que : $R_{2}+T_{2} = 1$ (loi de conservation de
l'énergie).
\end{itemize}
On vérifie bien que $R_{1} = R_{2} = R$ et $T_{1} = T_{2} = T$. \\
A.N. : Pour $n_{1} = 1,00$ (air), $n_{2} = n = 1,5$ (verre),
\underline{$R = 0,040$} ; \underline{$T = 0,96$}. \\
En l'absence de traitement des surfaces en verre, seul le phénomène
de transmission des rayons lumineux est assuré. Pour augmenter le
coefficient $R$, il faut réaliser un traitement de surface.
\\
\underline{Remarque :} On peut aussi traiter le cas de l'interface
2/1 (interface verre/air) : cette situation se rencontre lorsqu'un
rayon dans la lame de verre se réfléchit ou est transmis au point
$J$ par exemple. Le coefficient de réflexion en amplitude (sous
incidence normale) est : $r_{2}'
=\frac{n_{2}-n_{1}}{n_{1}+n_{2}}=r_{2}$ ; coefficient de
transmission sous incidence normale : $t_{2}'
=\frac{2n_{2}}{n_{1}+n_{2}}=t_{2}$. \\
Donc inutile de différentier les différentes situations : interfaces
1/2, 2/1, 2/3 et 3/2. On gardera uniquement les dénominations
interfaces air/verre et verre/air comme indiqué dans le cours avec
les coefficients \textit{ad hoc}.

\subsection*{c) Calcul de la différence de marche entre deux rayons transmis consécutifs}
Le calcul de la différence de marche géométrique a été réalisé dans
le TD O1. \\
Tous calculs menés, on obtient ($n_{2} = n$) : \mathcolorbox{gray!20}{\delta =
2ne\cos r}.

\subsection*{d)  Calcul de l'éclairement : formule d'Airy}
Notons $\varphi = k_{0}\delta$ ($k_{0}$ le nombre d'onde dans le
vide) le déphasage entre deux rayons transmis consécutivement par la
lame. \\
Déterminons l'expression du champ vibratoire transmis à l'infini par
la lame de verre :
\begin{itemize}
\item L'amplitude du premier transmis par la lame a pour amplitude
$A_{0}t_{1}t_{2}$ (le rayon traverse tour à tour l'interface
air/verre puis verre/air) et pour phase : $\varphi_{1} = \omega t
-k_{0}\left(S_{\mathrm{\infty}}OIM_{\mathrm{\infty}}\right)-\varphi_{s}$ : le champ vibratoire
transmis en $M_{\mathrm{\infty}}$ est : $a_{1}\left(M_{\mathrm{\infty}},t\right) =
A_{0}t_{1}t_{2}e^{i\varphi_{1}}$.
\item L'amplitude du second
transmis par la lame a pour amplitude $A_{0}t_{1}t_{2}r_{2}^{2}$.
\\
En effet, le rayon traverse d'abord l'interface air/verre
(coefficient de transmission $t_{1}$), puis est réfléchi en $I$ à
l'interface verre/air (coefficient de réflexion $r_{2}$), est
ensuite réfléchi en $J$ à l'interface verre/air (coefficient de
réflexion $r_{2}$) (2 réflexions = $2\times (2-1)$) et enfin est
transmis au niveau du point K dans l'air (interface verre/air,
coefficient de transmission $t_{2}$).
\\
La phase associée est $\varphi_{2} = \omega t
-k_{0}\left(S_{\mathrm{\infty}}OIJKM_{\mathrm{\infty}}\right)-\varphi_{s}$ : le champ vibratoire
transmis en $M_{\mathrm{\infty}}$ est : \\
$a_{2}\left(M_{\mathrm{\infty}},t\right) =
A_{0}t_{1}t_{2}r_{2}^{2}e^{i\varphi_{2}}$. \\
Le troisième rayon transmis en $M_{\mathrm{\infty}}$ a subi au total, une
transmission (à $t_{1}$ via l'interface air/verre), une transmission
(à $t_{2}$ sur l'interface verre/air), 4 = $2\times (3-1)$
réflexions sur les interfaces verre/air (de coefficient de réflexion
$r_{2}$) : Le champ vibratoire est de la forme : \\
$a_{3}\left(M_{\mathrm{\infty}},t\right) = A_{0}t_{1}t_{2}r_{2}^{4}e^{i\varphi_{3}}$
avec \\
$\varphi_{3} = \omega t -
k_{0}\left[S_{\mathrm{\infty}}OIJKLNM_{\mathrm{\infty}}\right]-\varphi_{s}$.
\item Le n$^{\grave{e}me}$ rayon transmis en $M_{\mathrm{\infty}}$ a subi au total, une
transmission (à $t_{1}$ via l'interface air/verre), une transmission
(à $t_{2}$ sur l'interface verre/air), $2(n-1)$ réflexions sur les
interfaces verre/air (de coefficient de réflexion $r_{2}$). \\
Le
champ vibratoire est de la forme : $a_{n}\left(M_{\mathrm{\infty}},t\right) =
A_{0}t_{1}t_{2}r_{2}^{2(n-1)}e^{i\varphi_{n}}$.
\end{itemize}
Les différentes ondes produites à l'infini proviennent d'une même
source : ces champs sont mutuellement cohérents (interférences par
division d'amplitude). Les champs vibratoires s'ajoutent; on écrit
ainsi :
\begin{eqnarray}undefined
a\left(M_{\mathrm{infty}},t\right)&=& \sum_{i=0}^{\infty}a_{i}\left(M_{\mathrm{\infty}},t\right), \notag \\
&=&
A_{0}t_{1}t_{2}e^{i\varphi_{1}}\left[1+r_{2}^{2}e^{i\left(\varphi_{2}-\varphi_{1}\right)}+
r_{2}^{4}e^{i\left(\varphi_{3}-\varphi_{1}\right)}+ ... \notag \\
&+& r_{2}^{2(n-1)}e^{i\left(\varphi_{n}-\varphi_{1}\right)}+ ...\right]
\label{Eq.111}
\end{eqnarray}
Or, $t_{1}t_{2} = \frac{4n_{1}n_{2}}{\left(n_{1}+n_{2}\right)^{2}} =T$ ;
$r_{2}^{2} = R$. \\
$\varphi = \varphi_{2} - \varphi_{1} =
k_{0}\left[\left(S_{\mathrm{\infty}}O\right)+(OI)+\left(IJKM_{\mathrm{\infty}}\right)\right]-k_{0}\left[\left(S_{\mathrm{\infty}}O\right)+(OI)+(IH)+\left(HM_{\mathrm{\infty}}\right)\right]$.
\\ Or, $(IJ)=(JK)$ ; par application du principe du retour inverse et
du théorème de Malus, on a $\left(KM_{\mathrm{\infty}}\right) = \left(HM_{\mathrm{\infty}}\right)$, d'où :
$\varphi_{2} - \varphi_{1} = k_{0}\left[2(IJ)-(IH)\right] = k_{0}\delta$ avec
$\delta = 2ne\cos(r)$ (Cf. TD O1 pour le calcul).
\\
De même, $\varphi_{3}-\varphi_{1} = 2\varphi = 2k_{0}\delta$ ; de
façon plus générale, $\varphi_{n}-\varphi_{1} = (n-1)\varphi$.\\
On en déduit que :
\begin{eqnarray*}
a(M_{\infty},t)=A_{0}Te^{i\varphi_{1}}[1+Re^{i\varphi}+R^{2}e^{2i\varphi}+...+R^{n-1}e^{i(n-1)\varphi}+...].
\end{eqnarray*}
On est une présence d'une série géométrique convergente de raison
$q=Re^{i\varphi}$ ($|q| = \left[R| < 1$). Il s'ensuit que :
\begin{eqnarray}\label{eq.444}
a\left(M_{\mathrm{\infty}},t\right)= \frac{A_{0}Te^{i\varphi_{1}}}{1-Re^{i\varphi}}.
\end{eqnarray}
L'éclairement (ou l'intensité lumineuse) en $M_{\mathrm{\infty}}$ est
($\mathcal{E}_{0} = A_{0}^{2}$ éclairement de la source ponctuelle):
\\
$\mathcal{E}\left(M_{\mathrm{\infty}}\right) =
\frac{\mathcal{E}_{0}T^{2}}{\left(1-Re^{i\varphi}\right)\left(1-Re^{-i\varphi}\right)}$.
ou encore sachant que $T^{2} = (1-R)^{2}$ : \\
$\mathcal{E}\left(M_{\mathrm{\infty}}\right) =
\frac{\mathcal{E}_{0}(1-R)^{2}}{\left(1+R^{2}-2R\cos\varphi\right)}$. \\
Or, $\cos\varphi = 1-2\sin^{2}\frac{\varphi}{2}$. On écrit donc :
$\mathcal{E}\left(M_{\mathrm{\infty}}\right) =
\frac{\mathcal{E}_{0}(1-R)^{2}}{\left(1-2R+R^{2}+4R\sin^{2}\frac{\varphi}{2}\right)}$. \\
Il ne reste plus qu'à diviser cette expression par $(1-R)^{2}$ et
poser $M = \frac{4R}{(1-R)^{2}}$ pour obtenir la formule d'Airy de
l'éclairement\footnote{On proscrit la formule d'interférences à deux
ondes de notre ami Fresnel !} : \mathcolorbox{gray!20}{\mathcal{E}\left(M_{\mathrm{\infty}}\right) =
\frac{\mathcal{E}_{0}}{1+M\sin^{2}\frac{\varphi}{2}}}.

\subsection*{7.  Calcul du contraste C}
Par définition, le contraste est : $C =
\frac{\mathcal{E}_{\mathrm{max}}-\mathcal{E}_{\mathrm{min}}}{\mathcal{E}_{\mathrm{max}}+\mathcal{E}_{\mathrm{min}}}$.
\\
Ici, $\mathcal{E}_{\mathrm{max}} = \mathcal{E}_{0}$ et $\mathcal{E}_{\mathrm{min}} =
\frac{\mathcal{E}_{0}}{1+M}$. Il vient : $C = \frac{M}{M+2}$. Or, $M
= \frac{4R_{3}}{\left(1-R_{3}\right)^{2}}$, d'où $C =
\frac{4R_{3}}{4R_{3}+2\left(1-2R_{3}+R_{3}^{2}\right)}$, puis le résultat :
\mathcolorbox{gray!20}{C = \frac{2R_{3}}{1+R_{3}^{2}}}.

\subsection*{8. Introduction de la finesse F}
On cherche les phases $\varphi$ telles que : $\mathcal{E}\left(\varphi\right) =
\frac{\mathcal{E}_{\mathrm{max}}}{2}= \frac{\mathcal{E}_{0}}{2}$, soit :
$1+M\sin^{2}\left(\frac{\varphi}{2}\right)=2$ ou encore :
$\sin^{2}\frac{\varphi}{2} = \frac{1}{M}$, puis
$\sin\frac{\varphi_{\mathrm{\pm}}}{2} = \pm\frac{1}{\sqrt{M}}$. Or,
expérimentalement, $M
\gg1$ de sorte que $\sin\frac{\varphi}{2}$ est "petit ; on fait
l'approximation suivante (DL à l'ordre 1) : $\sin\frac{\varphi}{2}
\approx \frac{\varphi}{2}$. Il vient : $\varphi_{\mathrm{\pm}} = \pm
\frac{2}{\sqrt{M}}$, puis : $\Delta \varphi_{\mathrm{1/2}} =
\varphi_{+}-\varphi_{-} = \frac{4}{\sqrt{M}}$ ; or, par définition
de la finesse, $F = \frac{2\pi}{\Delta \varphi_{\mathrm{1/2}}} =
\frac{\pi}{2}\sqrt{M}$ ; or $M = \frac{4R_{3}}{\left(1-R_{3}\right)^{2}}$. \\
On en déduit que : \mathcolorbox{gray!20}{\Delta \varphi_{\mathrm{1/2}} = \frac{2\left(1-R_{3}\right)}{\sqrt{R_{3}}}} et \mathcolorbox{gray!20}{F = \frac{\pi \sqrt{R_{3}}}{1-R_{3}}} ($0\leq R_{3} < 1$). \\
La finesse d'un Fabry-Pérot est directement liée à la qualité de
réflexion des faces intérieures de la cavité. Idéalement, si $R_{3}
\rightarrow 1$, $F \rightarrow +\infty$. Typiquement, la finesse
d'un interféromètre de Fabry-Pérot optique est de l'ordre de
$10^{2}$. \\
\underline{Remarque :} calcul de la finesse de l'interféromètre de
Michelson. \\
On utilise la formule de l'éclairement de Fresnel :
$\mathcal{E} = 2\mathcal{E}_{0}\left[1 + \cos \varphi\right]$. On cherche les
déphasages $\varphi$ tels que : $\mathcal{E}\left(\varphi\right) =
\frac{\mathcal{E}_{\mathrm{max}}}{2} = 2\mathcal{E}_{0}$. Les déphasages qui
correspondent sont de la forme : $\varphi_{m} =
(2m+1)\frac{\pi}{2}$. On en déduit : $\Delta \varphi_{\mathrm{1/2}} =
\varphi_{\mathrm{m+1}}-\varphi_{m} = \pi$, d'où une finesse valant : $F =
\frac{2\pi}{\Delta \varphi_{\mathrm{1/2}}} = 2$ !

\subsection*{9.  L'intervalle spectral libre (ISL)}
On cherche les maxima de l'éclairement. Ces derniers correspondent à
$\sin \frac{\varphi}{2} =0$, soit $\varphi_{m} = 2m\pi$ $m \in
\mathbb{Z}$. Notons $\nu_{m}$ les fréquences correspondant à ces
phases $\varphi_{m}$. Or, $\varphi = k_{0}\delta =
\frac{\omega}{c}\delta = \frac{2\pi \nu}{c}\delta$ ; donc, on a :
$\frac{2\pi \delta \nu_{m}}{c} = 2m\pi$. Il vient: $\nu_{m}=
\frac{mc}{\delta}$. Enfin,
comme on travaille sous incidence quasi-normale, $\delta \approx 2e$. \\
On en déduit que les fréquences \textit{ad hoc} sont: $\nu_{m} =
\frac{mc}{2e}$, d'où l'ISL : \mathcolorbox{gray!20}{\nu_{\mathrm{m+1}}-\nu_{m} =
\frac{c}{2e}}. \\
L'ISL est une grandeur physique couramment utilisée dans les
"cavités Fabry-Pérot".

\subsection*{10.  Pouvoir de résolution PR du Fabry-Pérot}
D'après le critère de Rayleigh (en supposant $\lambda_{2} >
\lambda_{1}$), le doublet est résolu si\footnote{Le doublet est
résolu ("on observe les deux radiations") si l'écart en longueur
d'onde entre les centres des deux radiations $\Delta \lambda$ est
supérieure au plus petit "$\Delta \lambda$" mesurable par
l'interféromètre, noté $\Delta \lambda_{\mathrm{min}}$, soit $\Delta \lambda
> \Delta \lambda_{\mathrm{min}}$. Ceci dit, ce raisonnement qu'on utilise à
plusieurs reprises dans ce cours est un peu simpliste ! En effet, il
existe (Cf. cours O1 à O3), la largeur "naturelle" $\Delta
\lambda_{\mathrm{nat}}$ des radiations (Cf. Chapitre O2 avec l'exemple de la
raie "large"); l'effet Doppler et les collisions dans une source
spectrale sont sources d'élargissement des raies. Typiquement, dans
les lampes spectrales $\Delta \lambda_{\mathrm{nat}} \sim 0,1$ nm. Le doublet
est effectivement résolu si $\Delta \lambda > \Delta \lambda_{\mathrm{nat}} >
\Delta \lambda_{\mathrm{lim}}$.} $\Delta \varphi \geq \Delta \varphi_{\mathrm{1/2}}$
avec $\Delta \varphi = \frac{2\pi \delta}{\lambda_{1}} - \frac{2\pi
\delta}{\lambda_{2}} \geq \Delta \varphi_{\mathrm{1/2}} = \frac{2\pi}{F}$.
Posons $\lambda_{0} = \frac{\lambda_{1}+\lambda_{2}}{2}$ la longueur
d'onde moyenne du doublet (Cf. Partie "Optique Ondulatoire") et
$\Delta \lambda = \lambda_{2}-\lambda_{1}$ l'écart en longueur
d'onde entre les
centres des deux radiations formant le doublet. \\
Il vient : $2\pi\delta \frac{\Delta \lambda}{\lambda_{0}^{2}} \geq
\frac{2\pi}{F}$ ou encore en écrivant que $\delta \approx
p\lambda_{0}$ avec $p$ ordre moyen d'interférence associé à la
longueur d'onde moyenne du doublet, $2\pi p \frac{\Delta
\lambda}{\lambda_{0}} \geq \frac{2\pi}{F}$. On en déduit que : $pF
\geq \frac{\lambda_{0}}{\Delta \lambda}$. \\
Faisons diminuer (par la pensée) l'écart $\Delta \lambda$ jusqu'à la
limite de résolution du doublet. Le plus petit écart détectable est
$\Delta \lambda_{\mathrm{min}}$ tel que : \mathcolorbox{gray!20}{pF =
\frac{\lambda_{0}}{\Delta \lambda_{\mathrm{min}}}}. \\
La quantité (sans unité) $PR = pF = \frac{\lambda_{0}}{\Delta
\lambda_{\mathrm{min}}}$ est appelée pouvoir de résolution de
l'interféromètre.


\subsection*{12.  Filtre interférentiels}
\underline{Exemple d'un filtre interférentiel} (en lame en
cryolithe, d'indice $n$ = 1,365 et d'épaisseur $e'$ = 1,00 $\mu$m). \\
Les longueurs d'onde passantes sont : 2,730 $\mu$m (m = 1) ; 1,365
$\mu$m (m = 2) ; 0,6825 $\mu$m (m = 3) ; 0,455 $\mu$m (m = 4) ;
0,341 $\mu$m (m = 5), ... Il y a donc deux longueurs d'onde
passantes dans le
visible. \\
Pour n'en laisser qu'une, il faut donc modifier l'épaisseur : avec
$e'$ = 0,20 $\mu$mm, il ne reste qu'une longueur d'onde passante
dans le visible: $\lambda_{0}$ = 546 nm (m = 1). La bande passante
$\Delta \lambda_{\mathrm{lim}}$ du filtre autour de la longueur d'onde
$\lambda_{0}$ = 546 nm (radiation verte de la lampe spectrale à
vapeur de mercure) est donnée par la largeur à mi-hauteur du pic de
transmission de la cavité Fabry-Pérot. Le calcul est similaire à
celui du pouvoir de résolution (PR) de l'interféromètre. On rappelle
que : $\Delta \lambda_{\mathrm{lim}} = \frac{\lambda_{0}}{pF}$. Ici $p$ est
renommé $m$. Typiquement pour un coefficient de réflexion en énergie
$R_{3}$
= = 0,99, on a une finesse F = 312. \\
Pour le filtre interférentiel, on a m = 1 et $\lambda_{0}$ = 546 nm.
On obtient : $\Delta \lambda_{\mathrm{lim}}$ = 1,75 nm.

\section*{II. L'interféromètre de Mach-Zehnder}
\subsection*{4) Calculs}
\subsection*{a) Calcul tout azimut}
\begin{itemize}
\item Calcul de l'amplitude du champ vibratoire en sortie de l'interféromètre sur la voie
(R) \\
-   Selon le chemin optique $\left(OA_{2}O'\right)=L_{2}$ : \\
Il y a réflexion sur la séparatrice $Sp_{1}$, puis sur le miroir
"parfait" $\left(M_{2}\right)$, puis transmission à travers la séparatrice
$Sp_{2}$. "L'amplitude complexe" du champ vibratoire en sortie est :
$\underline{A}_{2} =
A_{0}\underline{r}\underline{t}\underline{r_{s}} = -
A_{0}\underline{r}\underline{t}$. \\
-   Selon le chemin optique $\left(OA_{1}O'\right)=L_{1}$ : \\
Il y a transmission par la séparatrice $Sp_{1}$, puis réflexion sur
le miroir "parfait" $\left(M_{1}\right)$, puis réflexion sur la séparatrice
$Sp_{2}$. "L'amplitude complexe" du champ vibratoire en sortie est :
$\underline{A}_{1} =
A_{0}\underline{r}\underline{t}\underline{r_{s}} = -
A_{0}\underline{r}\underline{t}$.
\item Calcul de l'amplitude du champ vibratoire en sortie de
l'interféromètre sur la voie (R') \\
-   Selon le chemin optique $\left(OA_{2}O'\right)=L_{2}$ : \\
Il y a réflexion sur la séparatrice $Sp_{1}$, puis sur le miroir
"parfait" $\left(M_{2}\right)$, puis réflexion sur la séparatrice $Sp_{2}$.
"L'amplitude complexe" du champ vibratoire en sortie est :
$\underline{A}_{3} = A_{0}\underline{r}^{2}\underline{r_{s}} =
-A_{0}\underline{r}^{2}$. \\
-   Selon le chemin optique $\left(OA_{1}O'\right)$ : \\
Il y a transmission par la séparatrice $Sp_{1}$, puis réflexion sur
le miroir "parfait" $\left(M_{1}\right)$, puis transmission par la séparatrice
$Sp_{2}$. "L'amplitude complexe" du champ vibratoire en sortie est :
$\underline{A}_{4} = A_{0}\underline{t}^{2}\underline{r_{s}} = -
A_{0}\underline{t}^{2}$.
\item Calcul de la différence de marche $\delta$ entre les rayons interférant en
sortie. \\
La différence de marche entre deux rayons issus du point $O$ de la
séparatrice $Sp_{1}$ (diviseur d'amplitude) est selon la voie (R) ou
(R') : $\delta = L_{2}-L_{1} = \left(OA_{2}O'M_{\mathrm{\infty}}\right) -
\left(OA_{1}O'M_{\mathrm{\infty}}\right)$ = $\left[\mathrm{d}_{2}+\mathrm{d}_{1}-e
+ne+O'M_{\mathrm{\infty}}\right]-\left[\mathrm{d}_{1}+\mathrm{d}_{2}+O'M_{\mathrm{\infty}}\right]=(n-1)e$. \\
Le déphasage entre les rayons interférant à l'infini selon la voie
(R) ou (R') est : \mathcolorbox{gray!20}{\varphi = \frac{2\pi}{\lambda_{0}}(n-1)e}.
\item Calcul de l'éclairement en (R) : \\
On est présence d'un montage par division d'amplitude. Les champs
vibratoires en sortie selon la voie (R) s'ajoutent (car les rayons
sont issus d'un même incident ; il y a cohérence temporelle) : \\
$\underline{a_{R}}\left(M_{\mathrm{\infty}}\right) =
\underline{A}_{1}e^{i\varphi_{1}}+\underline{A}_{2}e^{i\varphi_{2}}$
avec ici $\underline{A}_{1} = \underline{A}_{2} =
-i\frac{A_{0}}{2}$, $\varphi_{1} = k_{0}L_{1}  -\omega t
-\varphi_{s}$ et $\varphi_{2} = k_{0}L_{2}  -\omega t
-\varphi_{s}$. \\
L'éclairement en sortie selon cette voie est : \\
$\mathcal{E}\left(M_{\mathrm{\infty}}\right) =
\left[\underline{A}_{1}e^{i\varphi_{1}}+\underline{A}_{2}e^{i\varphi_{2}}\right]\left[\underline{A}_{1}^{*}e^{-i\varphi_{1}}+\underline{A}_{2}^{*}e^{-i\varphi_{2}}\right]$
=
$|\underline{A}_{1}|^{2}+|\underline{A}_{2}|^{2}+\frac{A_{0}^{2}}{4}\left[e^{i\left(\varphi_{2}-\varphi_{1}\right)}+e^{-i\left(\varphi_{2}-\varphi_{1}\right)}\right]$.
\\
On pose $\varphi = \varphi_{2} -\varphi_{1} = k_{0}\left[L_{2}-L_{1}\right] =
\frac{2\pi}{\lambda_{0}}(n-1)e$. \\
Il vient ($|\underline{A}_{1}|^{2} = |\underline{A}_{2}|^{2} =
\frac{A_{0}^{2}}{4}$): $\mathcal{E}\left(M_{\mathrm{\infty}}\right) =
\frac{A_{0}^{2}}{2}\left[1+\cos\varphi\right]$, d'où ($\mathcal{E}_{0} =
A_{0}^{2}$ et $\cos\varphi = 2\cos^{2}\frac{\varphi}{2} -1$) : \\
\centerline{\mathcolorbox{gray!20}{\mathcal{E}\left(M_{\mathrm{\infty}}\right) =
\mathcal{E}_{0}\cos^{2}\frac{\varphi}{2}}}.
\item Calcul de l'éclairement en (R') : \\
On écrit : $\underline{a_{\mathrm{R'}}}\left(M_{\mathrm{\infty}}\right) =
\underline{A}_{3}e^{i\varphi_{2}}+\underline{A}_{4}e^{i\varphi_{1}}$
avec ici $\underline{A}_{3} =
-A_{0}\left[\frac{i}{\sqrt{2}}\right]^{2}=\frac{A_{0}}{2}$ et
$\underline{A}_{4} =
-A_{0}\left[\frac{1}{\sqrt{2}}\right]^{2}=-\frac{A_{0}}{2}$, $\varphi_{1} =
k_{0}L_{1} -\omega t -\varphi_{s}$ et $\varphi_{2} = k_{0}L_{2}
-\omega t
-\varphi_{s}$. \\
L'éclairement en sortie selon cette voie est ($\underline{A}_{3}\underline{A}_{4}^{*} = \underline{A}_{3}^{*}\underline{A}_{4}=-\frac{A_{0}^{2}}{4}$): \\
$\mathcal{E}'\left(M_{\mathrm{\infty}}\right) =
\left[\underline{A}_{3}e^{i\varphi_{2}}+\underline{A}_{4}e^{i\varphi_{1}}\right]\left[\underline{A}_{3}^{*}e^{-i\varphi_{2}}+\underline{A}_{4}^{*}e^{-i\varphi_{1}}\right]$
=
$|\underline{A}_{3}|^{2}+|\underline{A}_{4}|^{2}-\frac{A_{0}^{2}}{4}\left[e^{i\left(\varphi_{2}-\varphi_{1}\right)}+e^{-i\left(\varphi_{2}-\varphi_{1}\right)}\right]$.
\\
On pose $\varphi = \varphi_{2} -\varphi_{1} = k_{0}\left[L_{2}-L_{1}\right] =
\frac{2\pi}{\lambda_{0}}(n-1)e$. \\
Il vient ($|\underline{A}_{3}|^{2} = |\underline{A}_{4}|^{2} =
\frac{A_{0}^{2}}{4}$): $\mathcal{E}'\left(M_{\mathrm{\infty}}\right) =
\frac{A_{0}^{2}}{2}\left[1-\cos\varphi\right]$, d'où ($\mathcal{E}_{0} =
A_{0}^{2}$ et $\cos\varphi = 1-2\sin^{2}\frac{\varphi}{2}$) : \\
\centerline{\mathcolorbox{gray!20}{\mathcal{E}'\left(M_{\mathrm{\infty}}\right) =
\mathcal{E}_{0}\sin^{2}\frac{\varphi}{2}}}.
\end{itemize}
On vérifie que $\mathcal{E}_{0} =\mathcal{E}+\mathcal{E}'$. Ce
résultat est ... normal ! Les miroirs $\left(M_{1}\right)$ et $\left(M_{2}\right)$ sont
parfaits, la séparatrice également
($|\underline{r}|^{2}+|\underline{t}|^{2}=1$). C'est une conséquence
de la conservation de l'énergie.

























\end{document}



\\
\subsubsection*{b) Mise en équation}



\begin{wrapfigure}{r}{0.25\textwidth}
\epsfig{file=Fig1.PNG,height=4cm, width=6.6cm}
\caption\protect{Système élémentaire formé par un élément élémentaire de
corde de longueur $\mathrm{d}\ell$. Sur la figure, les vecteurs sont en
caractère gras.}\label{Fig.1}
\end{wrapfigure}


Considérons un élément de corde ($\Sigma$) de longueur élémentaire
$\mathrm{d}\ell = \sqrt{\mathrm{d}x^{2}+\mathrm{d}y^{2}}$, de masse $\mathrm{d}m = \mu(x,t)\mathrm{d}\ell$
($\mu(x,t)$ est la masse linéique de la corde en $x$ à $t$ (Cf.
figure \ref{Fig.1})).
\begin{itemize}
\item \underline{Question n$^{\circ}1$ :} Pour de "petites" déformations transversales de la corde (hypothèse n$^{\circ}$1), ce qui
correspond à de petits angles, on écrit (Cf. figure n$^{\circ}$3 du cours):\\
$\tan\alpha(x,t) \approx \alpha(x,t) = \frac{y\left(x+\mathrm{d}x,t\right)-y(x,t)}{\mathrm{d}x} =
\frac{\partial y}{\partial x}(x,t)$.
\item \underline{Question n$^{\circ}2$ :} A une date $t$, $\mathrm{d}\ell = \sqrt{\mathrm{d}x^{2}+\mathrm{d}y^{2}} =
\mathrm{d}x\left(1+(\frac{\partial y}{\partial x}\right)^{2})^{\frac{1}{2}}$ ($\mathrm{d}y =
\frac{\partial y}{\partial x}\mathrm{d}x+\frac{\partial y}{\partial t}\mathrm{d}t =
\frac{\partial y}{\partial x}\mathrm{d}x$ à $t$ fixé). Or, $\left(\frac{\partial
y}{\partial x}\right)^{2}$ $\ll$ $1$ est un infiniment petit du second
ordre (hypothèse n$^{\circ}$1). A l'ordre le plus bas, $\mathrm{d}\ell \approx \mathrm{d}x$ ce qui est compatible avec l'hypothèse n$^{\circ}3$. \\
$\forall t$, $\forall x$, écrivons que la masse $\mathrm{d}m(x,t)$ de
$\left(\Sigma\right)$ est égale à sa masse au repos (en l'absence de
perturbation appliquée). On a : $\mathrm{d}m(x,t) = \mu(x,t)\mathrm{d}\ell \approx
\mu(x,t)\mathrm{d}x = \mu_{0}\mathrm{d}x$. Il vient qu'à l'ordre d'approximation
utilisé, la masse linéique de la corde est constante : $\forall x$,
$\forall t$, $\mu(x,t) = \mu_{0}$.
\item \underline{Question n$^{\circ}3$ :} A une date $t$, grâce à l'hypothèse n$^{\circ}$4, l'élément de corde ($\Sigma$)
subit en $x$ la tension $-\overrightarrow{T}(x,t)$ et en $x+\mathrm{d}x$ la
tension $\overrightarrow{T}\left(x+\mathrm{d}x,t\right)$. \\
On suppose en outre que la corde est non plombée de sorte que son
poids élémentaire est négligeable devant la résultante des forces de
tension s'exerçant sur ($\Sigma$)
(hypothèse n$^{\circ}$3). \\
Appliquons le TCI à $\left(\Sigma\right)$ de masse $\mathrm{d}m = \mu_{0}\mathrm{d}x$.
L'accélération d'un point de la corde en $x$ à $t$ est :
$\overrightarrow{a} =
\frac{\partial^{2}y}{\partial t^{2}}(x,t)\overrightarrow{u}_{y}$. \\
L'accélération du centre de masse $G$ de $\left(\Sigma\right)$ est situé à la
position $x+\frac{\mathrm{d}x}{2}$ : $\overrightarrow{a}(G) =
\frac{\partial^{2}y}{\partial
t^{2}}\left(x+\frac{\mathrm{d}x}{2},t\right)\overrightarrow{u}_{y} \approx
\frac{\partial^{2}y}{\partial t^{2}}(x,t)\overrightarrow{u}_{y}$.
\\
On a : \\
$\mathrm{d}m\overrightarrow{a}=-\overrightarrow{T}(x,t)+\overrightarrow{T}\left(x+\mathrm{d}x,t\right)$.
\\
D'après l'hypothèse n$^{\circ}$1, Il n'y a pas de mouvement selon
l'axe (Ox) : la projection du TCI
donne : \\
$0 = -T_{x}(x,t)+T_{x}\left(x+\mathrm{d}x,t\right)$. \\
Or, $T_{x}\left(x+\mathrm{d}x,t\right) \approx T(x,t)
+\frac{\partial T_{x}}{\partial x}\mathrm{d}x$. Il vient : $\frac{\partial
T_{x}}{\partial x} = 0$ Donc $T_{x}$ est indépendant de $x$. \\
Or, $\forall x$, $\forall t$, $T_{x} = T\cos\alpha (x,t) \approx T$.
\\
$T$ est la norme de la tension de la corde : elle est donc constante
à l'ordre d'approximation considéré.
\item \underline{Question n$^{\circ}4$ :} Selon l'axe (Oy), le TCI donne :
\begin{eqnarray}\label{Eq.1}
\mu_{0}\mathrm{d}x\frac{\partial^{2}y}{\partial t^{2}}(x,t) =
-T_{y}(x,t)+T_{y}\left(x+\mathrm{d}x,t\right).
\end{eqnarray}
Or, $T_{y}(x,t) = T\sin\left(\alpha(x,t\right)) \approx T\alpha(x,t)$ et
$T_{y}\left(x+\mathrm{d}x,t\right) = T\sin\left(\alpha(x+\mathrm{d}x,t\right))\approx T\alpha\left(x+\mathrm{d}x,t\right)$. On
en déduit que (\ref{Eq.1}) devient :
\begin{eqnarray}\label{Eq.2}
\mu_{0}\mathrm{d}x\frac{\partial^{2}y}{\partial t^{2}}(x,t) =
T\left[\alpha\left(x+\mathrm{d}x,t\right)-\alpha(x,t)\right],
\end{eqnarray}
puis en effectuant un DL à l'ordre 1 de $\alpha\left(x+\mathrm{d}x,t\right) \approx
\alpha(x,t)+\frac{\partial \alpha}{\partial x}(x,t)\mathrm{d}x$, on obtient :
\begin{eqnarray}\label{Eq.3}
\mu_{0}\mathrm{d}x\frac{\partial^{2}y}{\partial t^{2}}(x,t) = T\frac{\partial
\alpha}{\partial x}(x,t)\mathrm{d}x.
\end{eqnarray}
Enfin, du fait que $\alpha(x,t) \approx \frac{\partial y}{\partial
x}(x,t)$, on aboutit à l'équation suivante :
\begin{eqnarray}\label{Eq.4}
\mu_{0}\frac{\partial^{2}y}{\partial t^{2}}(x,t) =
T\frac{\partial^{2}y}{\partial x^{2}}(x,t).
\end{eqnarray}
On pose \mathcolorbox{gray!20}{v = \sqrt{\frac{T}{\mu_{0}}}} une grandeur homogène
à une vitesse appelée célérité des ondes mécaniques le long de la
corde. On aboutit à une équation de propagation satisfaite par un
quelconque ébranlement de la corde connue sous le nom d'équation des
cordes vibrantes ou équation de propagation de Jean le Rond
d'Alembert des ondes mécaniques :\\
\centerline{\mathcolorbox{gray!20}{\frac{\partial^{2}y}{\partial x^{2}}(x,t)
-\frac{1}{v^{2}}\frac{\partial^{2}y}{\partial t^{2}}(x,t)=0}}.
\item AN: \underline{$v \sim \sqrt{\frac{3\times 10^{3}}{6\times
10^{-3}}} = \frac{10^{3}}{\sqrt{2}} \approx
\frac{10^{3}}{\frac{3}{2}} \approx 7\times 10^{2}$ m.s$^{-1}$}. \\
\end{itemize}

\subsubsection*{c) Solutions}
Les solutions de l'équation de d'Alembert (Cf. cours EM6 pour la
démonstration) est la somme de 2 ondes planes progressives
unidimensionnelles\footnote{Les surfaces d'ondes sont planes ; il y
a invariance de tout ébranlement vibratoire $y(x,t)$ par translation
selon les axes $Oy$ et $Oz$.} se propageant selon l'axe
(Ox) en sens opposés\footnote{ou la forme équivalente : $y(x,t) = f(x-vt)+g(x+vt)$.} : \\
\centerline{\mathcolorbox{gray!20}{y(x,t) = f\left(t-\frac{x}{v}\right)+g\left(t+\frac{x}{v}\right)}}.

\subsection*{5. Considérations énergétiques}
\subsubsection*{a) Energie cinétique d'un élément de corde et énergie cinétique linéique}
On considère l'élément de corde $\left(\Sigma\right)$. \\
La vitesse associée\footnote{A ne pas confondre avec la célérité $v$
qui est la vitesse de propagation de l'onde selon la corde, i.e.
selon $(Ox)$.} à un ébranlement vibratoire (déplacement
transversale) est définie par :
\mathcolorbox{gray!20}{\tilde{v}(x,t) = \frac{\partial y}{\partial t}(x,t)}. \\
L'énergie cinétique de ce système élémentaire de longueur $\mathrm{d}\ell
\approx \mathrm{d}x$ et de masse $\mathrm{d}m = \mu_{0}\mathrm{d}x$ est : $\mathrm{d}E_{c} =
\frac{1}{2}\mathrm{d}m\left(\frac{\partial y}{\partial t}\right)^{2}$, ou encore :
\mathcolorbox{gray!20}{\mathrm{d}E_{c} = \frac{1}{2}\mu_{0}\mathrm{d}x\left(\frac{\partial y}{\partial
t}\right)^{2}}. \\
On définit l'énergie cinétique linéique (unité SI : le J.m$^{-1}$)
par : \mathcolorbox{gray!20}{e_{c} = \frac{\mathrm{d}E_{c}}{\mathrm{d}x} =
\frac{1}{2}\mu_{0}\left(\frac{\partial y}{\partial t}\right)^{2}}.

\subsubsection*{b) Energie potentielle de déformation d'un élément de corde et énergie potentielle linéique}
On considère l'élément de corde $\left(\Sigma\right)$.
\begin{itemize}
\item \underline{Question n$^{\circ}1$ :} Soit un système soumis à
la résultante des forces conservatives $\overrightarrow{F}$. On
définit l'énergie potentielle associée par (Cf. M1):
$\overrightarrow{F} = -\grad E_{p}$ ce qui équivaut à
: $\delta W\left(\overrightarrow{F}\right) =
\overrightarrow{F}.\overrightarrow{\mathrm{d}\ell} = -\mathrm{d}E_{p}$. \\
Soit un opérateur exerçant une force $\overrightarrow{F}_{\mathrm{op}}$.
L'idée est d'introduire maintenant une définition d'une forme
d'énergie potentiellement restituable sous forme d'énergie cinétique
(d'où la dénomination d'énergie potentielle !) par $\mathrm{d}E_{p} = \delta
W_{\mathrm{op}} = \overrightarrow{F}_{\mathrm{op}}.\overrightarrow{\mathrm{d}\ell}$. \\
Comment s'y prendre ? Tentative de réponse pour y voir clair :
appliquons le théorème de l'énergie cinétique au système entre les
dates $t$ et $t+\mathrm{d}t$ : $\mathrm{d}E_{c} = \delta W\left(\overrightarrow{F}\right) +\delta
W\left(\overrightarrow{F}_{\mathrm{op}}\right)$ = $\mathrm{d}E_{c} = -\mathrm{d}E_{p}+\delta
W\left(\overrightarrow{F}_{\mathrm{op}}\right)$. Si l'on souhaite obtenir la définition
opérationnelle de l'énergie potentielle, il faut donc que $\mathrm{d}E_{c} =
0$ : en d'autres termes, le travail de l'opérateur doit être
quasi-statique\footnote{Entre $t$ et $t+\mathrm{d}t$, le système demeure à
l'équilibre mécanique : $\overrightarrow{F}+\overrightarrow{F}_{\mathrm{op}}
= \overrightarrow{0}$}; on en déduit que : \mathcolorbox{gray!20}{\mathrm{d}E_{p} = \delta
W_{\mathrm{op}}}. \\
La variation d'énergie potentielle $\Delta E_{p}$ entre deux dates
finies $t_{1}$ et $t_{2}$ est égale au travail d'un opérateur,
succession de transformations quasi-statiques entre ces dates, ce
qu'on écrit laconiquement : $\Delta E_{p} =
W\left(\overrightarrow{F}_{\mathrm{op}}\right)$.
\item \underline{Question n$^{\circ}2$ :} L'introduction de
l'énergie potentielle de déformation d'un élément $\left(\Sigma\right)$ de la
corde est grandement simplifiée si l'on considère une définition
opérationnelle : l'énergie potentielle (élémentaire) de déformation
de $\left(\Sigma\right)$ est égale au travail élémentaire d'un opérateur qui
fait passer la longueur de la corde de $\mathrm{d}x$ (longueur de repos) à
$\mathrm{d}\ell$ (longueur de la corde déformée de l'élément $\left(\Sigma\right)$). On
écrit donc : $\mathrm{d}E_{p} = \delta
W\left(\overrightarrow{F}_{\mathrm{op}}\right) = T\left(\mathrm{d}\ell -\mathrm{d}x\right)$. \\
Or, on a montré qu'à l'ordre le plus bas : $\mathrm{d}\ell \approx \mathrm{d}x$, donc
$\mathrm{d}E_{p} = 0$ ... Etrange ! \\
En fait, comme toujours en physique, ce résultat aberrant provient
du "degré" de finesse des approximations faites. A l'ordre le plus,
on a écrit que : $\mathrm{d}\ell \approx \mathrm{d}x$ ce qui nous a permis notamment
de démontrer l'équation des cordes vibrantes.
\\
Or, cette approximation est trop "grossière" pour introduire la
notion d'énergie potentielle de déformation de la corde. Aussi, on
écrit (Cf. question 2 du I.4)) : $\mathrm{d}\ell = \sqrt{\mathrm{d}x^{2}+\mathrm{d}y^{2}} =
\mathrm{d}x\left(1+(\frac{\partial y}{\partial x}\right)^{2})^{\frac{1}{2}}$ avec
$\left(\frac{\partial y}{\partial x}\right)^{2}$ $\ll$ $1$ un infiniment petit
du second ordre. On doit donc faire un DL à "l'ordre 1" pour avoir :
$\ell \approx \mathrm{d}x\left[1+\frac{1}{2}\left(\frac{\partial y}{\partial x}\right)^{2}\right]$.
\\
On en déduit alors l'énergie potentielle de $\left(\Sigma\right)$ : \\
\centerline{\mathcolorbox{gray!20}{\mathrm{d}E_{p} = \frac{1}{2}T\left(\frac{\partial
y}{\partial x}\right)^{2}\mathrm{d}x}}. \\
On définit l'énergie potentielle linéique de déformation par :
\mathcolorbox{gray!20}{e_{p} = \frac{\mathrm{d}E_{p}}{\mathrm{d}x} =  \frac{1}{2}T\left(\frac{\partial
y}{\partial x}\right)^{2}}. \\
L'énergie mécanique de $\left(\Sigma\right)$ est donc : \\
\centerline{\mathcolorbox{gray!20}{\mathrm{d}E = \left[
\frac{1}{2}\mu_{0}\left(\frac{\partial y}{\partial
t}\right)^{2}+\frac{1}{2}T\left(\frac{\partial y}{\partial x}\right)^{2}\right]\mathrm{d}x}}. \\
L'énergie de tout élément de corde est non quantifiée (contrairement
à l'étude comparative menée dans le confinement d'un quanton dans un
puits de potentiel de profondeur infinie). \\
L'énergie mécanique de la corde est : $E_{m} = \int_{0}^{L} \left[
\frac{1}{2}\mu_{0}\left(\frac{\partial y}{\partial
t}\right)^{2}+\frac{1}{2}T\left(\frac{\partial y}{\partial x}\right)^{2}\right]\mathrm{d}x$.
\end{itemize}

\section*{II. Modes propres de vibration}
\subsection*{3. Résolution}\begin{itemize}
\item \underline{Question n$^{\circ}1$ :}
On injecte dans l'équation des cordes vibrantes des solutions
de la forme : $y(x,t) = f(x)\cos\omega t$. \\
On obtient $\forall x$; $\forall t$, $\cos\omega
t\left[\frac{\mathrm{d}^{2}f}{\mathrm{d}x^{2}}+\left(\frac{\omega}{v}\right)^{2}f(x)\right]=0$.
\\
On en déduit l'équation différentielle suivante satisfaite par $f$ :
\begin{eqnarray}\label{Eq.7}
\frac{\mathrm{d}^{2}f}{\mathrm{d}x^{2}}+\left(\frac{\omega}{v}\right)^{2}f(x)=0.
\end{eqnarray}
\item \underline{Question n$^{\circ}2$ :} La solution de (\ref{Eq.7}) sont "oscillantes". Aussi, on pose $k
= \sqrt{\frac{\omega}{v}}$ pulsation spatiale ou nombre d'onde. \\
La solution est de la forme : $f(x)  =A\cos(kx)+B\sin(kx)$. \\
Exploitons les CL du problème. On a deux noeuds de vibration en
$x=0$ et $x=L$, soit :\\
CL1 : $\forall t$, $y(0,t) = 0$ ; \\
CL2 : $\forall t$, $y(L,t) = 0$. \\
La CL1 conduit à : $f(0)=0$, d'où : $A=0$ ; \\
la CL2 donne : $f(L) = 0$, soit ($B\neq 0$) : $\sin(kL) = 0$. Il y a
donc quantification du nombre d'onde : $k_{n} = \frac{n\pi}{L}$ avec
$n$ un entier naturel non nul\footnote{Si $n$ est nul, le nombre
d'onde serait nul ce qui conduirait à $f(x) = 0$ en tout point de la
corde : un résultat hautement absurde car la corde est belle et bien en vibration !}. \\
Les pulsations, les longueurs d'onde et les fréquences sont
également quantifiées\footnote{La relation $\omega_{n} = k_{n}v =
\frac{n\pi v}{L}$ démontre au passage le caractère non dispersif de
la propagation d'ondes mécaniques le long de la corde.} :
$\omega_{n} = k_{n}v = \frac{n\pi v}{L}$, $\lambda_{n} =
\frac{2\pi}{k_{n}} = \frac{2L}{n}$ et $f_{n} =
\frac{\omega_{n}}{2\pi} = \frac{nv}{2L}$ avec $n \in
\mathbb{N}^{*}$.
\item \underline{Question n$^{\circ}3$ :} Pour $n=1$, on a le mode propre dit fondamental, de fréquence $f_{1}
= \frac{v}{2L}$. Les modes pour $n \geq 2$ sont appelés modes
harmoniques avec $f_{n}  =nf_{1}$.
\end{itemize}
Les solutions \mathcolorbox{gray!20}{y_{n}(x,t) =
B_{n}\sin\left(k_{n}x\right)\cos\left(\omega_{n}t\right)} sont les solutions sinusoïdales
de l'équation de propagation : on les qualifie de modes propres de
vibration.

\subsection*{5. Décomposition en série de Fourier}
\subsubsection*{a) Cas de la corde de Melde}
L'équation de d'Alembert étant linéaire,
toute somme de solution est encore solution de cette équation. On
écrit la solution la plus générale de l'équation sous la forme d'une
série (dite de Fourier) "somme" de tous les modes propres de
vibration (selon l'usage, les $B_{n}$ sont rebaptisés $b_{n}$) :
\begin{eqnarray}\label{Eq.10}
y(x,t) = \sum_{n=1}^{+\infty}b_{n}\sin\left(\frac{n\pi x}{L}\right)\cos\left(\frac{n
\pi v}{L} t\right).
\end{eqnarray}
Cette approche est confortée par le fait que l'ébranlement
$y(x,t)=f(x)\cos\left(\omega t\right)$ étant une fonction $2L$-périodique
spatialement (de la variable $x$), elle est décomposable en série de
Fourier selon le développement donné par
(\ref{Eq.10}). \\
\subsubsection*{b) Généralisation (en lien avec le kit de survie
n$^{\circ}$3)}
\\
Toute fonction $f$ $2L$-périodique (ici
spatialement\footnote{L'approche temporelle est présentée dans le
kit de survie n$^{\circ}$3.}) est décomposable en série de Fourier:
\begin{eqnarray}\label{Eq.Fourier}
f(x) = a_{0}+\sum_{n=1}^{+\infty}f_{n}(x) , \notag \\
f(x) = a_{0}+\sum_{n=1}^{+\infty}a_{n}\cos\left(\frac{n\pi x}{L}\right)+
\sum_{n=1}^{+\infty}b_{n}\sin\left(\frac{n\pi x}{L}\right).
\end{eqnarray}
Dans (\ref{Eq.Fourier}), les coefficients  $a_{n}$ et $b_{n}$ sont
les coefficients de la série. Le terme $a_{0}$ représente
physiquement la valeur moyenne (ici moyenne spatiale) de $f$. Les
coefficients $a_{n}$ et $b_{n}$ pour $n \in \mathbb{N}^{*}$ se
calculent comme suit (Cf. le somptueux kit de survie n$^{\circ}$3):
\begin{eqnarray}undefined
a_{0} = \frac{1}{2L}\int_{0}^{2L}f(x)\mathrm{d}x, \label{Eq.Fourier2} \\
a_{n} = \frac{1}{L}\int_{0}^{2L}f(x)\cos\left(k_{n}x\right)\mathrm{d}x, \label{Eq.Fourier3} \\
b_{n} = \frac{1}{L}\int_{0}^{2L}f(x)\sin\left(k_{n}x\right)\mathrm{d}x.
\label{Eq.Fourier4}
\end{eqnarray}
En étudiant la parité de $f$, on peut faciliter la détermination des
coefficients de la série comme suit :
\begin{itemize}
\item Si la fonction est impaire, tous les $a_{n}$ ($n \in \mathbb{N}$) sont nuls (en particulier, $a_{0} = 0$ car la valeur moyenne
de $f$ est alors nulle) ;
\item Si la fonction est paire, tous les $b_{n}$ ($n \in \mathbb{N}^{*}$) sont
nuls.
\end{itemize}
Les modes propres de vibration s'écrivent de façon générale comme
suit (la quantification de $k$ induit celle de $\omega$):
\begin{eqnarray}\label{Eq.1000}
y(x,t) &=&
a_{0}+\sum_{n=1}^{+\infty}a_{n}\cos\left(k_{n}x\right)\cos\left(\omega_{n}t\right)
\notag \\
&+&\sum_{n=1}^{+\infty}b_{n}\sin\left(k_{n}x\right)\cos\left(\omega_{n}t\right),
\end{eqnarray}
avec $k_{n} = \frac{n \pi}{L}$ et $\omega_{n}  =\frac{n \pi v}{L}$.
\\

\underline{Nano-exercice :} On donne l'ébranlement d'un
quelconque point de la corde à $t=0$ par : $y(x,0) = f(x) =
4b\sin^{3}\left(\frac{\pi x}{L}\right)$. \\
La fonction $y$ étant impaire, tous les $a_{n}$ sont nuls. On peut
bien sûr calculer les différents $b_{n}$ par la formule
(\ref{Eq.Fourier4}). Il est en fait plus simple ici de linéariser le
$\sin^{3}$ pour obtenir les différents modes présents. On a :
$y(x,0) = f(x) = 3b\sin\left(\frac{\pi
x}{L}\right)-b\sin\left(\frac{3\pi x}{L}\right)$. \\
Cet ébranlement est la somme de deux modes propres : le fondamental
$n=1$ et le mode de troisième harmonique ($n=3$) qui sont :
$y_{1}(x,t) = 3b\sin\left(\frac{\pi x}{L}\right)\cos\left(\frac{\pi v}{L}t\right)$ et
$y_{3}(x,t) = -b\sin\left(\frac{3\pi x}{L}\right)\cos\left(\frac{3\pi v}{L}t\right)$, d'où
:
\begin{eqnarray*}
y(x,t) &=& y_{1}(x,t)+y_{3}(x,t) \\
&=& 3b\sin(\frac{\pi x}{L})\cos(\frac{\pi v}{L}t) -b\sin(\frac{3\pi
x}{L})\cos(\frac{3\pi v}{L}t).
\end{eqnarray*}


\section*{III. Ondes stationnaires résonnantes}
\subsection*{2. Etablissement d'ondes stationnaires}
\subsubsection*{b) Mise en oeuvre de la démarche}\begin{itemize}
\item \underline{Question n$^{\circ}1$ :} Pour toute pulsation de la corde, la superposition de deux ondes planes progressives donne naissance à une onde
stationnaire. Aussi, on choisit maintenant des solutions de la forme
(Cf. EM7) : $y(x,t) = f(x)g(t)$.
\item \underline{Question n$^{\circ}2$ :} On injecte cette solution dans l'équation de d'Alembert et on arrive
à :
$g(t)\frac{\mathrm{d}^{2}f}{\mathrm{d}x^{2}}-\frac{1}{v^{2}}f(x)\frac{\mathrm{d}^{2}g}{\mathrm{d}t^{2}}(t)
= 0$. \\
On divise cette équation par le produit $f(x)g(t)$ et on a :
\begin{eqnarray}\label{Eq.20}
\frac{1}{f(x)}\frac{\mathrm{d}^{2}f}{\mathrm{d}x^{2}}(x)=\frac{1}{v^{2}}\frac{1}{g(t)}\frac{\mathrm{d}^{2}g}{\mathrm{d}t^{2}}(t).
\end{eqnarray}
Cette équation devant satisfaite pour tout $x$ et $t$, le terme de
gauche n'est fonction que de $x$, celui de droite que de $t$ ;
mathématiquement, cela impose que ces deux termes sont égaux à une
même constante notée $\mathrm{cste}$ ; on "découple" ainsi les variables
d'espace et de temps ; en outre, on recherche des solutions
oscillantes de sorte que l'on pose cette constante égale à $\mathrm{cste} =
-k^{2}$ avec $k$ une grandeur homogène à l'inverse du carré d'une
longueur que l'on pose comme étant le carré de la pulsation spatiale
: $\mathrm{cste} = -k^{2}$. L'équation (\ref{Eq.20}) donne naissance aux deux
équations suivantes:
\begin{eqnarray}undefined
\frac{\mathrm{d}^{2}f}{\mathrm{d}x^{2}}(x)+k^{2}f(x) &= &0, \label{Eq.21} \\
\frac{\mathrm{d}^{2}g}{\mathrm{d}t^{2}}(t)+k^{2}v^{2}g(t) &= &0. \label{Eq.22}
\end{eqnarray}
Dans (\ref{Eq.22}), on pose : $\omega = kv$ (pulsation du problème).
Les solutions de ces deux équations sont : $f(x)
=A_{1}\cos(kx)+A_{2}\sin(kx)$ et\footnote{On aurait pu choisir $g$
sous la forme : $g(t) =B_{1}\cos\left(\omega t\right)+B_{2}\sin\left(\omega t\right)$.} ou
encore: $g(t) =B_{1}\cos\left(\omega t +\varphi\right)$ : $y(x,t) = f(x)g(t) =
A\cos(kx)\cos\left(\omega t
+\varphi\right)+B\sin(kx)\cos\left(\omega t+\varphi\right)$. \\
La CL1 est l'exploitation d'un noeud en $x=L$ pour tout $t$ : \\
CL 1 : $\forall t$, $y(L,t) = f(L)g(t) = 0$, soit $f(L) = 0$. \\
La CL2 s'obtient en écrivant qu'en $x=0$ pour tout $t$ l'ébranlement
vibratoire doit être égal à celui immposé par le vibreur : $y(0,t) =
b\cos\left(\omega t\right)$.
\item \underline{Question n$^{\circ}3$ :} La CL 2 donne : $f(0) = b = A_{1}$ ; la CL 1 : $f(L) = 0 =
b\cos(kL)+A_{2}\sin(kL) = 0$; soit : $A_{2} =
-b\frac{\cos(kL)}{\sin(kL)}$. Il s'ensuit que : $f(x) =
b\cos(kx)-\frac{b\cos(kL)}{\sin(kL)}\sin(kx) =
\frac{b}{\sin(kL)}\left[\cos(kx)\sin(kL)-\cos(kL)\sin(kx)\right]$, ou encore :
\\
\centerline{\mathcolorbox{gray!20}{f(x) = \frac{b}{\sin(kL)}\sin[k(L-x)\right]}}. \\
L'ébranlement vibratoire d'un point $x$ de la corde à la date $t$
est : \\
$y(x,t) = B_{1}\frac{b}{\sin(kL)}\sin[k(L-x)\right]\cos\left(\omega t
+\varphi\right)$. \\
On exploite la CI en $x=0$ : $y\left(x=0,t=0\right) = b$, soit
$B_{1}b\cos\left(\varphi\right) = b$, d'où : $B_{1}\cos\left(\varphi\right) = 1$. \\
L'amplitude de l'ébranlement étant par définition une quantité
postive, on pose : $B_{1} = 1$ et $\varphi=0$.  \\
Au final, on retiendra que (au dénominateur $\sin(kL) \neq 0$) :
\begin{eqnarray}\label{Eq.200}
y(x,t) = b\frac{\sin[k(L-x)\right]}{\sin(kL)}\cos\left(\omega t\right).
\end{eqnarray}
\item \underline{Question n$^{\circ}4$ :} On peut rechercher les positions $x_{i}$ des noeuds de vibration par
: $\forall t$, $y\left(x_{i},t\right) = 0$, soit ($\sin[k\left(L-x_{i}\right)\right] = 0$) :
$k\left(L-x_{i}\right) = i\pi$ avec $i$ un entier, d'où :\\
\centerline{\mathcolorbox{gray!20}{x_{i} =
L-i\frac{\pi}{k}}}. \\
En particulier, pour $i=0$, on retrouve la CL1 : la position $x_{0}
= L$ est un noeud de vibration. \\
la distance $\Delta x_{N}$ entre deux noeuds consécutifs vaut
$\frac{\pi}{k} = \frac{\lambda}{2}$. \\
On peut également rechercher les positions $x_{\mathrm{\ell}}$ des ventres de
vibration par : $\forall t$, $y\left(x_{\mathrm{\ell}},t\right)$ est un extremum, soit
($\sin[k\left(L-x_{\mathrm{\ell}}\right)\right] = \pm 1$) :
$k\left(L-x_{\mathrm{\ell}}\right) = \left(2\ell +1\right)\frac{\pi}{2}$ avec $\ell$ un entier, d'où :\\
\centerline{\mathcolorbox{gray!20}{x_{\mathrm{\ell}} =
L-\left(\ell+\frac{1}{2}\right)\frac{\pi}{k}}}. \\
la distance $\Delta x_{V}$ entre deux ventres consécutifs vaut
également $\frac{\pi}{k} = \frac{\lambda}{2}$.
\item \underline{Question n$^{\circ}5$ :} Pour établir la relation (\ref{Eq.200}), aucune hypothèse n'a été
faite sur la valeur de $\omega$ : quelque soit la valeur de la
pulsation imposée par le générateur, une onde stationnaire se forme
donc le long de la corde.
\item \underline{Question n$^{\circ}6$ :} Montrons que l'ébranlement de l'onde stationnaire donné par
(\ref{Eq.200}) est la somme de deux OPPH se propageant selon (Ox) en
sens opposés (on utilise la formule $\sin(a)\cos(b) =
\frac{1}{2}\left[\sin(a+b)+\sin(a-b)\right]$ et le caractère impair de la
fonction sinus) :
\begin{eqnarray}\label{Eq.201}
y(x,t) &=& \frac{b}{2\sin(kL)}\left[\sin\left(\omega t-kx+kL\right) \notag
\\
&-& \sin\left(\omega t +kx -kL\right)\right].
\end{eqnarray}
\end{itemize}
\subsection*{3. Etablissement d'ondes stationnaires résonnantes}
\subsubsection*{b) Condition d'obtention d'ondes stationnaires résonnantes}\begin{itemize}
\item \underline{Question n$^{\circ}1$ :} Il apparaît des ondes stationnaires dites résonnantes
lorsque dans l'équation (\ref{Eq.200}), $\sin(kL) = 0$, soit $k_{n}
= \frac{n\pi}{L}$ avec $n \in \mathbb{N}^{*}$. \\
En d'autres termes, les ondes stationnaires deviennent résonnantes
lorsque la pulsation délivrée par le générateur est égale à la
pulsation d'un des modes propres de vibration de la corde :
\mathcolorbox{gray!20}{\omega_{n} = n\frac{\pi v}{L}} et \mathcolorbox{gray!20}{f_{n} =
\frac{\omega_{n}}{2\pi} = \frac{nv}{2L}}. \\
Ces fréquences sont celles des modes propres de vibration. \\
Pour $n=1$, l'onde stationnaire résonnante qui se forme est celle
associée à la fréquence du mode propre fondamental : $f_{1} =
\frac{v}{2L}$.
\item \underline{Question n$^{\circ}2$ :} On a : \mathcolorbox{gray!20}{f_{n} = nf_{1}}.
\item \underline{Question n$^{\circ}3$ :} Dans ce modèle, si on veut conserver un ébranlement
vibratoire qui ne diverge pas, il faut en fait travailler au
voisinage d'une des pulsations propres de la corde. L'amplitude des
ventres de vibration est alors quasi-infinie (Cf. annexe
n$^{\circ}$6 ; l'amplitude des ventres proche de la résonance font
plusieurs km\footnote{Pour une corde dont on ne tient pas compte de
l'élasticité, ce résultat est quelque peu paradoxale ! En fait la
corde sans élasticité se romprait à la résonance ... Exit la corde
de notre ami Franz Melde !}). \\
La position des noeuds de vibration
est là encore donnée par l'expression : $x_{i} =
L-i\frac{\pi}{k}$ avec $i$ entier. \\
Pour l'onde stationnaire quasi-résonnante selon le mode $n=N$ (il y
a $N$ ventres de vibration et $N+1$ noeuds), les positions
successives des noeuds sont : $x_{0} = L$, $x_{1} =
L-\frac{\pi}{k_{N}}$, ..., $x_{N} = L-N\frac{\pi}{k_{N}}$. Or,
$k_{N} = \frac{N\pi}{L}-\epsilon$ avec $0< \epsilon \ll1$ (si on
choisit $\omega = \omega_{N}-\epsilon'$), donc $x_{N} =
L-\frac{L}{1-\frac{L}{\pi N}\epsilon} < 0$. \\
Le N+1$^{\grave{e}me}$ noeud est donc situé à gauche de la position
du vibreur mais proche de ce dernier.
\item \underline{Question n$^{\circ}4$ :} Pour améliorer le modèle, il faudrait tenir compte des frottements
fluide exercés par l'air sur la corde, prendre en compte le
caractère élastique de la corde. Ces hypothèses permettent de
limiter l'amplitude des ventres de vibration à une valeur finie.
\end{itemize}












\end{document}
