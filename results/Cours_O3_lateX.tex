\documentclass{article}
\usepackage{geometry}
\geometry{a4paper,margin=2cm}

\usepackage{wrapfig}

\everymath{\displaystyle}
\renewcommand{\epsilon}{\varepsilon}
%\renewcommand{\epsilon}{\mathchar"122}

\usepackage{esvect}
\usepackage{wrapfig}
\usepackage{mathrsfs}

\usepackage{physics}

% (?<!\\Bigg)\)\_
% <(.*?)>
% \_\{(?=em)(.*?)\}
% \\fbox{\$\left((.|\n\right)*?)\$}
% : ?\\\\ ?\n\$\left((.(?!(\\mathcolorbox\right))|\n)*?)\$ ?\.? ?\n?\\\\

\usepackage{xcolor}
\usepackage{soul}
\newcommand{\mathcolorbox}[2]{\fcolorbox{black}{#1}{$#2$}}

\DeclareSymbolFont{legacyletters}{OML}{cmm}{m}{it}
\let\j\relax
\DeclareMathSymbol{\j}{\mathord}{legacyletters}{"7C}


\newcommand{\oneast}{\bigskip\par{\large\centerline{*\medskip}}\par}


\usepackage[T1]{fontenc}
%% \usepackage[french]{babel}
\usepackage{epsfig}
\usepackage{graphicx}
\usepackage{amsmath}
%\setlength{\mathindent}{0cm}
\usepackage{amsfonts}
\usepackage{amssymb}
\usepackage{float}
\usepackage{esint}
\usepackage{enumitem}
\usepackage{frcursive}
%\usepackage{fourier}
%% \usepackage{amsrefs}
\reversemarginpar
\newcommand{\asterism}{%
\leavevmode\marginpar{\makebox[10em][c]{$10^{500}$\makebox[1em][c]{%
\makebox[0pt][c]{\raisebox{-0.3ex}{\smash{$\star\star$}}}%
\makebox[0pt][c]{\raisebox{0.8ex}{\smash{$\star$}}}%
}}}}

\setlength\parindent{0pt}

\let\oldiint\iint
\renewcommand{\iint}{\oldiint\limits}

\let\oldiiint\iiint
\renewcommand{\iiint}{\oldiiint\limits}

\let\oldoint\oint
\renewcommand{\oint}{\oldoint\limits}

\let\oldoiint\oiint
\renewcommand{\oiint}{\oldoiint\limits}


\renewcommand\overrightarrow{\vv}
\let\oldexp\exp
\renewcommand{\exp}[1]{\oldexp\left(#1\right)}

\newcommand{\ext}{\text{ext}}
\newcommand{\cste}{\text{cste}}

%\renewcommand{\div}{\mathrm{div}}
\let\div\relax
\DeclareMathOperator{\div}{\mathrm{div}}
\let\rot\relax
\DeclareMathOperator{\rot}{\overrightarrow{\mathrm{rot}}}
\let\grad\relax
\DeclareMathOperator{\grad}{\overrightarrow{\mathrm{grad}}}

\title{\huge{\textbf{Interférences lumineuses en lumière non polarisée - Division d'amplitude - Interféromètre de Michelson (O3)}}}
\author{par Guillaume Saget, professeur de Sciences Physiques en MP, Lycée Champollion}
\date{}

\begin{document}
\maketitle


\begin{abstract}
Je propose ici le cours O3 qui vient apporter les réponses au
support de cours éponyme.
\end{abstract}


\section*{III. Interféromètre réglé en lame d'air avec une source monochromatique étendue}
\subsection*{4.  Calcul de la différence de marche géométrique (sans ajout d'une LMC en sortie de l'interféromètre)}
\underline{Première méthode :}\\
La différence de marche entre le rayon immédiatement réfléchi par la
lame d'air équivalente (rayon 2 en vert sur la figure n$^{\circ}$6)
et le rayon 1 en rouge (ayant subi deux "réfractions" et une
réflexion dans la lame d'air) est : \\
$\delta_{\mathrm{1/2}} = \delta = \left(SLIKJM_{\mathrm{\infty}}\right)-\left(SLIHM_{\mathrm{\infty}}\right)$. \\
Or, $\left(SLIKJM_{\mathrm{\infty}}\right) = (SL)+(LI)+(IKJ)+\left(JM_{\mathrm{\infty}}\right)$ et
$\left(SLIHM_{\mathrm{\infty}}\right) = (SL)+(LI)+(IH)+\left(HM_{\mathrm{\infty}}\right)$. Il vient :
$\delta = (IKJ)+\left(JM_{\mathrm{\infty}}\right) - (IH)-\left(HM_{\mathrm{\infty}}\right)$. \\
Or, $M_{\mathrm{\infty}}$ est le point de concours des rayons interférant à
l'infini. En vertu du principe du retour inverse de la lumière,
c'est une source ponctuelle dont les surfaces d'ondes loin de ce
point sont des plans d'ondes. En vertu du théorème de Malus-Dupin,
les rayons lumineux sont orthogonaux à ces plans d'ondes. \\
Ainsi, $\left(\Sigma\right)$ est un plan d'ondes issu de $M_{\mathrm{\infty}}$. Comme
$H$ (projeté orthogonal de J sur le rayon immédiatement réfléchi par
la lame d'air) et $J \in \left(\Sigma\right)$, $\left(M_{\mathrm{\infty}}H\right) = \left(M_{\mathrm{\infty}}J\right)$
ou encore à l'aide du principe du retour inverse de la lumière : $\left(HM_{\mathrm{\infty}}\right) = \left(JM_{\mathrm{\infty}}\right)$. \\
On en déduit avec une joie évidente et non dissimulée que : \\
\centerline{\mathcolorbox{gray!20}{\delta = (IKJ)-(IH)=(IK)+(KJ)-(IH)}}. \\
On travaille dans l'air assimilé au vide, donc $(IH) = IH$, $(IK) =
IK$ et $(KJ) = KJ$. \\
Explicitons ces différentes distances en fonction de l'angle
d'incidence $i$. On a dans le triangle IKN : $\cos i =
\frac{e}{IK}$. Or, $IK = KJ$, d'où : $\delta = 2KJ-IH =
\frac{2e}{\cos i}-IH$. \\
En outre, dans le triangle IHJ, $\sin i = \frac{IH}{IJ}$. \\
Enfin, dans le triangle IKN ou JKN, $\tan i = \frac{IJ}{2e}$, soit :
$IJ = 2e\tan i$, puis $IH = IJ\sin i = 2e\tan i\sin i$. \\
On en déduit que : $\delta = \frac{2e}{\cos i} - 2e\tan i\sin i = \frac{2e}{\cos i}\left[1-\sin^{2}i\right] =\frac{2e}{\cos i}\left[\cos^{2}i\right] =
2e\cos i$.
\\
\underline{Deuxième méthode :}
\begin{itemize}
\item On s'aide de la figure de l'Annexe 3 (figure de gauche). \\
On calcule différence de marche entre les deux émergents (différence
de marche du rayon 1 par rapport au rayon 2) interférant à l'infini
issu d'un même incident émis par un point $S$ (via le théorème de
localisation des franges) de la source étendue (on fait le calcul
avec $S_{0}$ mais cela ne restreint en rien la généralité du
problème) : $\delta_{\mathrm{1/2}} = \delta =
\left(S_{0}LPQM_{\mathrm{\infty}}\right)-\left(S_{0}LIHM_{\mathrm{\infty}}\right)$. \\
On a : $\left(S_{0}LPQM_{\mathrm{\infty}}\right) = \left(S_{0}LP\right)+(PQ)+\left(QM_{\mathrm{\infty}}\right)$.
\\
Or, $\left(S_{0}LP\right) = \left(S_{1}'P\right)$ car $S_{1}'$ est le symétrique de
$S_{0}$ par le miroir $M_{1}$. Il vient : \\
$\left(S_{0}LPQM_{\mathrm{\infty}}\right) = \left(S_{0}LP\right)+(PQ)+\left(QM_{\mathrm{\infty}}\right) =
\left(S_{1}'P\right)+(PQ)+\left(QM_{\mathrm{\infty}}\right) = \left(S_{1}'Q\right)+\left(QM_{\mathrm{\infty}}\right)$. \\
Enfin, $S_{1}$ étant le symétrique de $S_{1}'$ par la séparatrice,
$\left(S_{1}'Q\right) = \left(S_{1}Q\right)$. On en déduit que : $\left(S_{0}LPQM_{\mathrm{\infty}}\right) =
\left(S_{1}'Q\right)+\left(QM_{\mathrm{\infty}}\right) = \left(S_{1}Q\right)+\left(QM_{\mathrm{\infty}}\right) =\left(S_{1}M_{\mathrm{\infty}}\right)$. \\
De même, $\left(S_{0}LIHM_{\mathrm{\infty}}\right)=\left(S_{0}L\right)+(LI)+\left(IHM_{\mathrm{\infty}}\right)$. \\
Or, $\left(S_{0}L\right) = \left(S_{2}'L\right)$ car $S_{2}'$ est le symétrique de
$S_{0}$ par la séparatrice. Il vient : \\
$\left(S_{0}LIHM_{\mathrm{\infty}}\right) = \left(S_{0}L\right)+(LI)+\left(IHM_{\mathrm{\infty}}\right) = \left(S_{2}'L\right)+(LI)+\left(IHM_{\mathrm{\infty}}\right) = \left(S_{2}'I\right) + \left(IHM_{\mathrm{\infty}}\right)$. \\
Or, $\left(S_{2}'I\right) = \left(S_{2}I\right)$ car $S_{2}$ est le symétrique de $S_{2}'$
par le miroir $M_{2}$. \\
On en déduit que : $\left(S_{0}LIHM_{\mathrm{\infty}}\right) = \left(S_{2}'I\right) +
\left(IHM_{\mathrm{\infty}}\right) =  \left(S_{2}I\right)+\left(IHM_{\mathrm{\infty}}\right) = \left(S_{2}M_{\mathrm{\infty}}\right)$.
\item Montrons que la différence de marche vaut $\delta = 2e\cos i$. \\
La différence de marche entre les deux émergents parallèles vaut :
$\delta = \left(S_{1}M\right)-\left(S_{2}M\right)$. \\
Or, $\left(S_{1}M\right) = \left(S_{1}H\right)+\left(HM_{\mathrm{\infty}}\right)$. $M_{\mathrm{\infty}}$ est le point
de concours des rayons interférant à l'infini. En vertu du principe
du retour inverse de la lumière, c'est une source ponctuelle dont
les surfaces d'ondes loin de ce point sont des plans d'ondes. En
vertu du théorème de Malus-Dupin,
les rayons lumineux sont orthogonaux à ces plans d'ondes. \\
Ainsi, $\left(\Sigma\right)$ est un plan d'ondes issu de $M_{\mathrm{\infty}}$. Comme
$H$ (projeté orthogonal de $S_{2}$ sur le rayon issu de $S_{1}$) et
$S_{2} \in \left(\Sigma\right)$, $\left(M_{\mathrm{\infty}}H\right) = \left(M_{\mathrm{\infty}}S_{2}\right)$
ou encore à l'aide du principe du retour inverse de la lumière : $\left(HM_{\mathrm{\infty}}\right) = \left(S_{2}M_{\mathrm{\infty}}\right)$. \\
Il vient : $\delta = \left(S_{1}H\right)$. Dans l'air, $\delta = S_{1}H$. Or,
dans le triangle $S_{1}S_{2}H$, $\cos i = \frac{S_{1}H}{2e}$, d'où :
\mathcolorbox{gray!20}{\delta = 2e\cos i}.
\item Explication de texte concernant la figure n$^{\circ}$7. \\
$S_{2}$ est le symétrique du point source $S$ par le miroir $M_{2}$.
$S_{1}$ est le symétrique du point source $S$ par le miroir $M_{1}$.
\\
On aura remarqué que le point source $S$ est en fait $S_{2}'$...
\end{itemize}


\subsection*{5. Calcul de l'intensité en un point M de l'écran}
\begin{itemize}
\item Décomposons la source étendue quasi-monochromatique en une
infinité de sources ponctuelles S quasi-monochro-matiques
incohérentes entre elles.
\item Chaque source ponctuelle S crée un
couple de sources secondaires $S_{1}$ et $S_{2}$ de même intensité
élémentaire $\mathrm{d}I_{0}$. A partir d'un incident (à $i$ fixé) issu d'un
quelconque point source $S \in \left[AB]$, la différence de marche entre
deux émergents parallèlement en sortie de l'inter-féromètre vaut
invariablement $\delta = \left(S_{1}M_{\mathrm{\infty}}\right)-\left(S_{2}M_{\mathrm{\infty}}\right) =
2e\cos i$.
\item Les sources secondaires $S_{1}$ et $S_{2}$ sont
mutuellement cohérentes (elles sont formées par division d'amplitude
à partir de $S$). La formule de Fresnel à deux ondes ponctuelles
cohérentes est licite. \\
En notant $\mathrm{d}I_{0}\left(S_{1}\right) =\mathrm{d}I_{0}\left(S_{2}\right)$ l'intensité d'une des
sources secondaires, on a : $\mathrm{d}I\left(M_{\mathrm{\infty}}\right) =
2\mathrm{d}I_{0}\left(S_{1}\right)\left[1+\cos\left(\Delta \varphi\right)\right]$ avec $\forall S_{1}\in
\left[S_{\mathrm{A1}}S_{\mathrm{B1}}\right]$ et $S_{2}  \in \left[S_{\mathrm{A2}}S_{\mathrm{B2}}\right]$, $\Delta \varphi =
\frac{2\pi}{\lambda_{0}}\delta$.
\item Les différentes sources
ponctuelles $S$ formant la source étendue sont incohérentes entre
elles : les intensités qu'elles produisent en $M_{\mathrm{\infty}}$
s'ajoutent. \\
L'intensité produite en $M_{\mathrm{\infty}}$ de la part des sources
secondaires étendues est ainsi : \\
$I\left(M_{\mathrm{\infty}}\right) = \int_{S_{1} \in \left[S_{\mathrm{A1}}S_{\mathrm{B1}}\right]}
2\mathrm{d}I_{0}\left(S_{1}\right)\left[1+\cos\left(\Delta \varphi\right)\right]$, ou encore : $I\left(M_{\mathrm{\infty}}\right)
= \left[\int_{S_{1} \in \left[S_{\mathrm{A1}}S_{\mathrm{B1}}\right]} 2\mathrm{d}I_{0}\left(S_{1}\right)\right]\left[1+\cos\left(\Delta
\varphi\right)\right]$, d'où : \mathcolorbox{gray!20}{I\left(M_{\mathrm{\infty}}\right) = 2I_{0}\left[1+\cos\left(\Delta
\varphi\right)\right]}. \\
On retrouve la formule de Fresnel à deux ondes présentée dans le
cours O2 pour des sources ponctuelles cohérentes ; ici les sources
secondaires sont cohérentes mais étendues !
\end{itemize}

\subsection*{6.  Nature des franges : anneaux d'interférences}
\begin{itemize}
\item Montrons que l'ajout de la lentille mince convergente en
sortie de l'interféromètre ne modifie pas la différence de marche
(Cf. Annexe n$^{\circ}$3). \\
Sur la figure de gauche, la différence de marche équivalente entre
les deux émergents issus d'un incident émis par $S_{0}$ est :
$\delta = (IKJ) - (IH)$.\\
Sur la figure de droite, la différence de marche \textit{ad hoc} est
: $\delta = \left(S_{0}LIKJBM\right)-\left(S_{0}LIHAM\right)$. \\
Or, $\left(S_{0}LIKJBM\right) = \left(S_{0}L\right)+(LI)+\left(IKJBM\right)$ et $\left(S_{0}LIHAM\right) =
\left(S_{0}L\right)+(LI)+(IHAM)$, d'où : $\delta = \left(IKJBM\right) - (IHAM)$.
\\
Or, $\left(IKJBM\right) = (IKJ)+(JBM)$ et $(IHAM) = (IH)+(HAM)$. Or, le point
$M$ sur l'écran est conjugué à un point source ponctuel à l'infini
dans la direction des sources secondaires. Dans le cours O1, nous
montrâmes que $\left(\Sigma\right)$ est un plan d'ondes issu de $M$ ; il vient
que : $(HAM) = (JBM)$. \\
On en déduit que la différence de marche se réduit à $\delta = (IKJ)
- (IH)$. CQFD !
\item Montrons que les franges sont des anneaux. \\
\underline{Première méthode :} On peut remarquer que l'écran est
placé perpendiculairement à la droite passant par les sources
secondaires $S_{1}$ et $S_{2}$ : les franges sont des anneaux
concentriques d'interférences ou franges
circulaires de centre $F'$, de rayon $r$. \\
\underline{Seconde méthode :} la formule de Fresnel est : $I(M) =
2I_{0}\left[1+\cos\left(\frac{2\pi}{\lambda_{0}}\times 2e\cos i\right)\right]$. \\
Les franges d'égale intensité sont donc les franges d'égale
inclinaison $i$. Or, il y a invariance de la figure d'interférences
à $i$ donné autour de la droite $(O'F')$ (Cf. Annexe n$^{\circ}$3,
figure de droite) : les franges sont des anneaux concentriques, de
centre $F'$.
\item Ecrivons que $\delta = 2e\cos i \approx 2e\left[1 -
\frac{i^{2}}{2}\right]$ (DL à l'ordre 2 sur $\cos i$ car $i \ll 1$). \\
Par ailleurs (Cf. Annexe n$^{\circ}$3, figure de droite), $\tan i
\approx i = \frac{r}{f'}$, d'où : $\delta \approx 2e\left[1-
\frac{r^{2}}{2f'^{2}}\right]=p\lambda_{0}$ avec $p$ l'ordre
d'interférences en $M$. Il vient : $p = \frac{2e}{\lambda_{0}}\left[1-\frac{r^{2}}{2f'^{2}}\right]$. \\
Notons $p_{0}$ l'ordre au centre (acquis pour $r = 0$) ; on a :
$p_{0} = \frac{2e}{\lambda_{0}}$. On en déduit que : \mathcolorbox{gray!20}{p =
p_{0}\left[1-\frac{r^{2}}{2f'^{2}}\right]}. \\
Lorsqu'on s'éloigne du centre de la figure, le rayon des anneaux
augmente, donc d'après la formule précédente, l'ordre $p$ diminue à
partir du centre de la figure.
\item Indiçons les rayons des anneaux en fonction de l'ordre
d'interférences : $p = p_{0}\left[1-\frac{r_{p}^{2}}{2f'^{2}}\right]$. Il vient
: $1 - \frac{p}{p_{0}} = \frac{r_{p}^{2}}{2f'^{2}}$ ; on en déduit
que : \mathcolorbox{gray!20}{r_{p} = f'\sqrt{2\left[1-\frac{p}{p_{0}}\right]}}.
\item Indiçons maintenant les anneaux brillants en fonction de leur
comptage à partir du centre. Dans le raisonnement proposé dans ce
cours (Cf. TD O3 pour un autre comptage possible), un rayon brillant
d'anneau nul est comptabilisé. \\
Soit $p_{0} \in \mathbb{N}^{\ast}$ (nous ne sommes pas à la teinte
plate : $e \neq 0$). Comme l'ordre diminue à partir du centre, il
existe un premier entier naturel $p_{1} = E\left(p_{0}\right) = \llcorner p_{0}
\lrcorner < p_{0}$ correspondant au premier anneau brillant de rayon
$r_{1}$. \\
Le second anneau brillant est de rayon $r_{2}$ et d'ordre $p_{2} =
p_{1} - 1$, le troisième anneau brillant est de rayon $r_{3}$ et
d'ordre $p_{3} = p_{2} - 1 = p_{1}-2$, ..., le $k^{\grave{e}me}$
anneau brillant est de rayon $r_{k}$ et d'ordre $p_{k} = p_{1} -
\left[k-1\right]$, ... \\
Pour le $k^{\grave{e}me}$ anneau brillant, la formule établie
précédemment permet d'écrire : $r_{k} =
f'\sqrt{2\left[1-\frac{p_{k}}{p_{0}}\right]}$ \\
$=$ $f'\sqrt{2\left[1-\frac{p_{1}-\left[k-1\right]}{p_{0}}\right]} = f'\sqrt{\frac{2}{p_{0}}\left[p_{0}-p_{1}+\left[k-1\right])\right]}$. \\
Posons $\epsilon = p_{0}-p_{1}=p_{0}-n$ l'excédent fractionnaire
($n$ entier naturel non nul) avec $0 \leq \epsilon < 1$. $\epsilon$
est nul si l'anneau au centre est brillant (rayon $r_{1}$ nul). \\
Or, $p_{0} = \frac{2e}{\lambda_{0}}$ ; on en déduit que :
\mathcolorbox{gray!20}{r_{k} = f'\sqrt{\frac{\lambda_{0}}{e}\left[\epsilon +k-1\right]}}.
\end{itemize}

\subsection*{7.  Applications numériques}
L'ordre au centre vaut : $p_{0} = \frac{2e}{\lambda_{0}} = 1733,1$.
Le résidu vaut $\epsilon = 0,109$. Comme l'ordre diminue au fur et à
mesure que l'on s'éloigne du centre de la figure, le premier anneau
sera brillant, d'ordre $p_{1} = 1733,0$ et de rayon $r_{1} = 1,09$
cm ; les ordres et rayons des  deuxième, troisième et quatrième
anneaux brillants seront $p_{2} = 1732,0$, $r_{2} = 3,57$ cm, $p_{3}
= 1731,0$, $r_{3} = 4,92$ cm et $p_{4} = 1730,0$, $r_{4}
= 5,98$ cm. \\
De même, l'ordre du premier anneau sombre sera $p_{1}' = 1732,5$, de
rayon $r_{1}' = 2,64$ cm ; les ordres et rayons des  deuxième,
troisième et quatrième anneaux sombres seront $p_{2}' = 1731,5$,
$r_{2}' = 4,30$ cm, $p_{3}' = 1730,5$, $r_{3}' = 5,48$ cm et $p_{4}'
= 1729,5$, $r_{4}' = 6,45$ cm.

\section*{IV. Interféromètre réglé en coin d'air avec une source
quasi-monochromatique étendue}
\subsection*{2.  Calcul de la différence de marche géométrique}
Sur la figure de gauche, $\cos\left(2\alpha\right) =
\frac{e(x)}{PK}$. Or, $\alpha \ll 1$, donc à l'ordre le plus bas,
$\cos\left(2\alpha\right) \approx 1$. A cet ordre d'approximation (DL à l'ordre
0), $e(x)=IP \approx PK$ ; les points $I$ et $K$ sont confondus à
cet ordre d'approximation. \\
Calculons la différence de marche géométrique entre le rayon
transmis dans le coin d'air et le rayon immédiatement réfléchi (en
l'absence de lentille mince convergente en sortie de
l'interféromètre). \\
On cherche à calculer : $\delta =
\left(S_{0}CLIPIM_{\mathrm{\infty}}\right)-\left(S_{0}CLILM_{\mathrm{\infty}}\right)$. Or,
$\left(S_{0}CLILM_{\mathrm{\infty}}\right) = \left(S_{0}C\right) +(CL)+(LI)+\left(ILM_{\mathrm{\infty}}\right)$ et
$\left(S_{0}CLIPIM_{\mathrm{\infty}}\right) =
\left(S_{0}C\right)+(CL)+(LI)+(IP)+(PI)+\left(ILM_{\mathrm{\infty}}\right)$. D'où (on travaille
dans l'air) : $\delta = (IP)+(PI) = 2e(x)$. \\
Or, $\tan\left(\alpha\right) \approx \alpha = \frac{e(x)}{x}$, d'où : $e(x)
\approx \alpha x$. On en déduit la différence de marche :
\mathcolorbox{gray!20}{\delta = 2\alpha x}.
\\
Calculons maintenant la différence de marche entre les rayons 1 et 2
en présence d'une lentille mince convergente en sortie de
l'interféromètre (Figure du bas de l'Annexe n$^{\circ}$5).
\begin{itemize}
\item On calcule $\delta_{\mathrm{1/2}} =
\left(S_{0}CLIPKQM\right)_{\mathrm{rayon\,1}}$\\ $-\left(S_{0}CLILRM\right)_{\mathrm{rayon\,2}}$.
\item Or,
$\left(S_{0}CLIPKQM\right)_{\mathrm{rayon\,1}} =
\left(S_{0}C\right)+(CL)+(LI)+(IP)+(PKQM)_{\mathrm{rayon\,1}}$ et
$\left(S_{0}CLILRM\right)_{\mathrm{rayon\,2}} = \left(S_{0}C\right)+(CL)+(LI)+(ILRM)_{\mathrm{rayon\,2}}$.
\\ Il reste à calculer : $\delta = \delta_{\mathrm{1/2}} =
(IP)+(PKQM)_{\mathrm{rayon\,1}} - (ILRM)_{\mathrm{rayon\,2}}$.
\item Exploitons le fait que $M$ est le conjugué de $P$ par la
seconde lentille mince convergente : $P
\underleftrightarrow{(LMC2)}M$, soit : $(PKQM)_{\mathrm{rayon\,1}} =
\left(PILRM\right)_{\mathrm{rayon\,2}}$. \\
Or, $\left(PILRM\right)_{\mathrm{rayon\,2}} = (PI)+(ILRM)_{\mathrm{rayon\,2}}$. On en déduit que
: $\delta = \delta_{\mathrm{1/2}} = (IP)+\left(PILRM\right)_{\mathrm{rayon\,2}} -
(ILRM)_{\mathrm{rayon\,2}} = (IP)+(PI) = 2IP = 2e(x) \approx 2\alpha x$.
\end{itemize}

\subsection*{3. Calcul de l'intensité}
Soit $I_{0}$ l'intensité du point source $S_{0} \in \left[AB]$.
Le point de concours des rayons interférant est un point $P \in
\left(M_{1}'\right)$. Or, les rayons émergents de l'interféromètre ont subi une
réflexion et une transmission par le coin d'air équivalent : chaque
rayon
émergent transporte l'intensité $\frac{I_{0}}{4}$. \\
L'intensité lumineuse en un point $P$ du miroir $M_{1}'$ est donnée
par la sempiternelle formule de Fresnel : $I(P)=2\times
\frac{I_{0}}{4}\left[1+\cos\left(\Delta \varphi(P\right))\right]$ avec $\Delta \varphi(P)
= \frac{2\pi}{\lambda_{0}}\times 2\alpha x$, soit : \\
$I(P) = I(x) = \frac{I_{0}}{2}\left[1+\cos\left(\frac{2\pi}{\lambda_{0}}\times
2\alpha x\right)\right]$.

\subsection*{4. Nature des franges}
\begin{itemize}
\item Les franges d'égale intensité sont les franges d'égale $x$ : les
franges sont donc rectilignes (d'équation $x = \mathrm{cste}$), parallèles à
l'arête du coin d'air (d'équation $x = 0$). \\
En $O$, la différence de marche vaut : $\delta(O) = p\lambda_{0} =
2\alpha x$ avec $x  =0$ ($e = 0$), donc l'ordre au centre est nul
(contrairement à la configuration lame d'air en dehors du contact
optique).
\item Calculons l'interfrange $i$. \\
On calcule la distance entre les centres de deux franges
consécutives de même nature, par exemple, deux franges brillantes
consécutives. Ainsi, pour les franges brillantes d'ordre $m$ et
$m+1$, on a : $\delta_{m} = 2\alpha x_{m} = m\lambda_{0}$ et
$\delta_{\mathrm{m+1}} = 2\alpha x_{\mathrm{m+1}} = (m+1)\lambda_{0}$. On en déduit
que : $i = x_{\mathrm{m+1}}-x_{m} = \frac{\lambda_{0}\left[m+1 -m\right]}{2\alpha} =
\frac{\lambda_{0}}{2\alpha}$.
\\
\underline{Autre méthode :} L'interfrange est la période spatiale de
l'intensité, soit : $I(x) =
\frac{I_{0}}{2}\left[1+\cos\left(\frac{2\pi x}{i}\right)\right]$. \\
Par comparaison avec la formule de Fresnel, on obtient immédiatement
que : \mathcolorbox{gray!20}{i = \frac{\lambda_{0}}{2\alpha}}.
\end{itemize}
Calcul d'ordre de grandeur de $i$ : pour $\alpha = 10^{-3}$ rad,
$\lambda_{0} = 6\times 10^{-7}$ m, \\
A.N. \underline{$i = \frac{6\times 10^{-7}}{2\times 10^{-3}} =
3\times 10^{-4}$ m $=$ $0,3$ mm}.

\section*{V.  Interféromètre réglé en lame d'air avec une source de lumière
blanche étendue}
\subsection*{2.  Estimation de la longueur de cohérence de la source de
lumière blanche via le critère semi-quantitatif de brouillage des
franges (critère de cohérence temporelle)}
Au centre du
profil spectral de lumière blanche, on peut écrire que : $\delta(M)
= p_{1}\lambda_{0}$ avec $\lambda_{0} =
\frac{\lambda_{1}+\lambda_{2}}{2}$; pour la radiation du profil à
$\lambda_{2} = \lambda_{0}+\frac{\Delta \lambda}{2}$ avec $\Delta
\lambda = \lambda_{2}-\lambda_{1}$, on a : $\delta(M) =
p_{2}\left[\lambda_{0}+\frac{\Delta \lambda}{2}\right]$. \\
On en déduit en $M$
la différence de l'ordre d'interférences : $\Delta p = p_{1}-p_{2} =
\frac{\delta}{\lambda_{0}}-\frac{\delta}{\lambda_{0}+\frac{\Delta
\lambda}{2}} =
\delta[\frac{1}{\lambda_{0}}-\frac{1}{\lambda_{0}+\frac{\Delta
\lambda}{2}}\right] = \delta \times \frac{\frac{\Delta
\lambda}{2}}{\lambda_{0}\left[\lambda_{0}+\frac{\Delta \lambda}{2}\right]}$. \\
D'après le critère, il y a brouillage des franges à la condition que
$\Delta p > \frac{1}{2}$, soit : $\delta \times \frac{\frac{\Delta
\lambda}{2}}{\lambda_{0}\left[\lambda_{0}+\frac{\Delta \lambda}{2}\right]} >
\frac{1}{2}$ ou encore : $\delta
>  \ell_{c} = \frac{\lambda_{0}\left[\lambda_{0}+\frac{\Delta \lambda}{2}\right]}{\Delta \lambda}$. \\
A.N. : $\lambda_{0} = 600$ nm ; $\Delta \lambda = 400$ nm, \\
\underline{$\ell_{c} = \frac{600\times 800}{400} = 1,2\times 10^{3}$
nm = 1,2 $\mu$m}.

\subsection*{3.  Exemples de calcul}
\subsubsection*{a. Au centre de la figure pour une épaisseur $e$
nulle} Pour tout $\lambda \in \left[\lambda_{1},\,
\lambda_{2}\right]$, au centre de la figure, pour $e = 0$, $p_{0} =
\frac{2e}{\lambda} = 0$, donc $\delta = 2e = p_{0}\lambda = 0$ ; au
centre, de la figure, par synthèse additive de toutes les radiations
composant la lumière
blanche, la frange centrale est blanche. \\
En fait, en l'absence des défauts de planéité des miroirs, la
différence de marche est nulle en tout point de l'écran : la teinte
plate est blanche.

\subsubsection*{b. Au centre de la figure pour une épaisseur $e$
"faible"} Au centre de la figure, $\delta = 2e$. Il existe
des longueurs d'onde $\lambda \in \left[\lambda_{1},\, \lambda_{2}\right]$
éteintes. En effet, l'ordre au centre varie entre les valeurs
$p_{\mathrm{min}} = \frac{2e}{\lambda_{2}} = \frac{1125}{800} = 1,4$ et
$p_{\mathrm{max}} = \frac{2e}{\lambda_{1}} = \frac{1125}{400} = 2,8$. \\
Les longueurs d'onde éteintes au centre de la figure sont celles
pour lesquelles l'ordre au centre est demi-entier, soit : $p_{1} =
1,5$ et $p_{2} = 2,5$ ; elles correspondent aux longueurs d'onde
$\lambda_{1}' = \frac{2e}{p_{1}}$ et $\lambda_{2}' =
\frac{2e}{p_{2}}$. \\
A.N. \underline{$\lambda_{1}' = 750$ nm} ; \underline{$\lambda_{2}'
= 450$ nm}. \\
Au centre, il existe une radiation dont la couleur est renforcée ;
c'est celle pour laquelle l'ordre est entier, soit $p_{3} = 2,0$ ;
elle correspond à la longueur d'onde $\lambda_{3}' =
\frac{2e}{p_{3}}$. \\
A.N. \underline{$\lambda_{3}' = 562,5$ nm}.
\end{document}
